\documentclass[a4paper, landscape]{extarticle}
\usepackage{etoolbox}
\usepackage{Alegreya}
\usepackage{amssymb, mathrsfs}
\usepackage[left=1cm,top=2cm,right=1cm,bottom=1cm]{geometry}
\usepackage{multicol}
\usepackage{MnSymbol}
\usepackage{fancyhdr}
\makeatletter
\fancypagestyle{mypagestyle}
{\newpage \fancyfoot[C]{} \renewcommand{\footrulewidth}{0pt}}
\makeatother
\pagestyle{mypagestyle}
\headsep 5pt         %% put this outside
\usepackage{lastpage}
\usepackage[usenames,dvipsnames]{color}
\usepackage{xcolor}

\newcommand{\C}{\mathbb{C}}
\newcommand{\N}{\mathbb N}
\newcommand{\R}{\mathbb{R}}
\newcommand{\Q}{\mathbb{Q}}
\newcommand{\Z}{\mathbb{Z}}

\newcommand{\expected}{\mathbb{E}}
\newcommand{\variance}{\mathbb{V}}

\usepackage{graphicx}
\relpenalty=10000
\binoppenalty=10000
\interlinepenalty=10000

\newenvironment{enumx}{\begin{enumerate} \setlength{\itemsep}{0pt} \setlength{\parskip}{0pt} \setlength{\parsep}{0pt}}{\end{enumerate}}
\newenvironment{itemx}{\begin{itemize} \setlength{\itemsep}{0pt} \setlength{\parskip}{0pt} \setlength{\parsep}{0pt}}{\end{itemize}}

\usepackage{float}

\definecolor{zolty}{HTML}{BFBF00}

\renewcommand{\arraystretch}{3}

\usepackage[polish]{babel}
\usepackage[utf8]{inputenc}
\usepackage[T1]{fontenc}
\selectlanguage{polish}

\newenvironment{Figure}
{\par\medskip\noindent\minipage{\linewidth}}
{\endminipage\par\medskip}

\begin{document}
\setlength{\belowdisplayskip}{2pt}
\setlength{\belowdisplayshortskip}{2pt}
\setlength{\abovedisplayskip}{2pt}
\setlength{\abovedisplayshortskip}{2pt}


\renewcommand{\footrulewidth}{0.4pt}
\fancyhead[LE,LO]{Rozkłady prawdopodobieństwa (MSC 60)}
\fancyhead[RO,RE]{almawauch@yandex.com, sierpień 2016}

\begin{multicols*}{3}
Niech $X$ będzie zmienną losową na przestrzeni $(\Omega, \mathcal F, P)$.
\begin{itemx}
\item \textbf{Wartość oczekiwana $X$} to $\expected[X] := \int_\Omega X\,\textrm{d}P$
\item $n$-ty \textbf{moment centralny} to $\mu_n=\expected[(X - \expected[X])^n]$.
\item \textbf{Wariancja} $\variance$ jest drugim momentem, pierwiastek z niej to \textbf{odchylenie standardowe}, $\sigma$.
\item \textbf{Skośność} $\gamma = \mu_3/\sigma^3$ i \textbf{kurtoza} $\kappa = \mu_4 / \sigma^4 -3$.
\item \textbf{Funkcja charakterystyczna}: $\expected[e^{itX}]$
%\item \textbf{Tworząca momenty} $M_X(t)$ to jej wartość w $-it$.
\item \textbf{Informacja Fishera} $\mathcal I(\theta) := \int (\partial_\theta \log f(x, \theta))^2 f(x, \theta)\,\textrm{d}x$.
\end{itemx}

Wszystkie rozkłady w wersji ,,probabilistycznej''.

\section*{Rozkłady dyskretne}
\begin{enumx}
\item \textbf{Borela} ($0 < \mu < 1$).
	Opisuje potomstwo wątrobowca, gdy dzietność w kolonii podlega $\lambda$-rozkładowi Poissona.
	Jego gęstość to $(\lambda k)^{k-1} : (k! \exp \lambda k)$.
	\[
		\expected = \frac{1}{1 - \lambda} \,\bullet\,
		\variance = \frac{\lambda}{(1-\lambda)^3}% \,\bullet\,
	\]

\item \textbf{Dwumianowy} (z $n \in \N$ i $0 < p < 1$).
	Liczba sukcesów w $n$ próbach Bernoulliego.
	Gęstość ${n \choose k} p^k q^{n-k}$, gdzie	 $q = 1-p$.
	Moda $\lfloor (n+1)p\rfloor$ lub $\lceil (n+1)p \rceil -1$.
	\[
		\expected = np \,\bullet\,
		\variance = npq \,\bullet\,
		\gamma = \frac{1-2p}{\sqrt{npq}} \,\bullet\,
		\kappa = \frac{1-6pq}{npq}.
	\]

	F. charakterystyczna $(q + p\exp(it))^n$, entropia: \[H = \frac 12 \log (2\pi e npq) + O(1/n).\]

\item \textbf{Beta-dwumianowy} (z $n \in \N$ i $\alpha, \beta > 0$).
	W urnie jest $\alpha$ kul czarnych i $\beta$ zielonych.
	Ciągniemy $n$-krotnie, za każdym razem dokładając zobaczoną kulę. Ile będzie czarnych?
	Gęstość ${n \choose k} B(k + \alpha, n - k + \beta) / B(\alpha, \beta)$.
	\[
		\expected = \frac{n\alpha}{\alpha + \beta} \,\bullet\,
		\variance = \frac{n \alpha \beta (\alpha + \beta + n)}{(\alpha + \beta)^2 (\alpha + \beta + 1)}
	\]

\item \textbf{Ujemny dwumianowy} ($r > 0$, $0 < p < 1$).
	Ile sukcesów w próbie Bernoulliego, zanim poniesiemy $r$ porażek.
	Ma gęstość ${k + r - 1 \choose k} q^r p^k$.
	$\mathcal I = r / (p^2q)$, gdzie $q = 1 - p$.
	\[
		\expected = \frac{pr}q \,\bullet\,
		\variance = \frac{pr}{q^2} \,\bullet\,
		\gamma = \frac{1+p}{\sqrt{pr}} \,\bullet\,
		\kappa = \frac{6p + q^2}{pr}
	\]

Funkcja charakterystyczna: \[\left(\frac{q}{1-p\exp (it)}\right)^r.\]

\item \textbf{Geometryczny} (z $0 < p < 1$).
	Proces Bernoulliego dopiero w $k$-tej próbie ($k \ge 1$) zakończy się sukcesem.
	Niech $q$ to $1-p$.
	Gęstość $q^{k-1} p$, dystrybuanta $1 - q^k$, moda $1$.
	\[
		\expected = \frac 1 p \,\bullet\,
		\variance = \frac{q}{p^2} \,\bullet\,
		\gamma = \frac{2-p}{\sqrt{q}} \,\bullet\,
		\kappa = 6 + \frac{p^2}{q}
	\]

	Funkcja charakterystyczna i entropia:
	\[
		\varphi(t) = \frac{p \exp (it)}{1 - q\exp(it)} \,\bullet\,
		H = \frac{q \log q + p \log p}{-p}
	\]

\item \textbf{Hipergeometryczny} (z $0 \le K, n \le N$).
	W stawie pływa $N$ ryb, $K$ spośród nich jest jadalna.
	Ile z $n$ wyciągniętych sztuk nie będzie trujących?
	Gęstość ${K \choose k}{N - K \choose n-k} / {N \choose n}$, zaś moda to $\lfloor(n+1)(K+1) / (N+2) \rfloor$.
	%Choć skośność i kurtoza istnieją, nie zmieściły się tu.
	\[
		\expected = \frac {nK}N \,\bullet\,
		\variance = \frac{nK(N-K)(N-n)}{N^3-N^2}
	\]

\item \textbf{Jednostajny} (na $[a,b] \cap \Z$ z $n = b - a +1$).
	Gęstość $1/n$, entropia $\log n$.
	\[
		\expected = \frac 12 (a+b) \,\bullet\,
		\variance = \frac {n^2-1}{12} \,\bullet\,
		\gamma = 0 \,\bullet\,
		\kappa = - \frac{6(n^2+1)}{5(n^2-1)}
	\]

	Funkcja charakterystyczna: \[\frac{\exp(iat) - \exp(i(b+1)t)}{n(1- \exp(it))}\]

\item Odwrotny \textbf{Markova-Polyi} ($\alpha, \beta, r > 0$).
	Ile porażek trzeba ponieść przed $r$ sukcesami w próbach Bernoulliego, kiedy stałe p-stwo sukcesu $p$ pochodzi z rozkładu beta?

	Gęstość $\Gamma(r+k) B(a+r, \beta + k) / (k! \Gamma(r) B(\alpha, \beta))$, dla $\alpha \le 1$, $\expected$ nie istnieje, zaś dla $\alpha \le 2$: $\variance$ nie istnieje.
	\[
		\expected = \frac{r \beta}{\alpha - 1} \,\bullet\,
		\variance = \frac{r (\alpha + r -1) \beta (\alpha + \beta - 1)}{(\alpha-2)(\alpha-1)^2}
	\]

	Skośność dla $\alpha >3$: \[\frac{(\alpha + 2r -1)(\alpha + 2\beta - 1) \sqrt{\alpha-2}}{(\alpha - 3) \sqrt{r(\alpha + r-1)\beta (\alpha + \beta - 1)}}\]

\item \textbf{Poissona} (z $\lambda > 0$).
	P-stwo danej liczby zdarzeń w stałym przedziale, kiedy znana jest ich średnia, a wystąpienie nie zależy od czasu, jaki upłynął od poprzedniego.

	Gęstość $\lambda^k \exp(-\lambda) / k!$, moda $\lceil\lambda\rceil - 1$ lub $\lfloor \lambda \rfloor$.
	Informacja Fishera $1 / \lambda$
	F. charakterystyczna: $\exp(\lambda(\exp(it) - 1))$.
	\[
		\expected = \lambda \,\bullet\,
		\variance = \lambda \,\bullet\,
		\gamma = \lambda^{-1/2} \,\bullet\,
		\kappa = \lambda^{-1}
	\]

	Przybliżona entropia: \[\frac{\log (2 \pi e \lambda)}2 - \frac{1}{12 \lambda} - \frac{1}{24\lambda^2} - \frac{19}{360\lambda^3} + O(\lambda^{-4}).\]

\item \textbf{Skellama} (z $\lambda_1, \lambda_2 > 0$).
	Różnica zmiennych losowych z rozkładu Poissona (niezależnych).
	\[
		\expected = \lambda_1 - \lambda_2 \,\bullet\,
		\variance = \lambda_1 + \lambda_2 \,\bullet\,
		\gamma = \frac{\expected}{(\variance)^{3/2}} \,\bullet\,
		\kappa = \frac{1}{\variance	}
	\]
	Gęstość: \[\exp(-\lambda_1-\lambda_2) \lambda_1^k \cdot \sum_{m =0}^\infty \frac{(\lambda_1\lambda_2)^m}{m! (m+k)!}\]
	Funkcja charakterystyczna: \[\varphi(t) = \exp(-(\lambda_1+\lambda_2) + \lambda_1 \exp(it) + \lambda_2 \exp(-it))\]

\item \textbf{Delaporte} ($\alpha, \beta, \lambda > 0$): Poissona, z losowym parametrem $\lambda$ + Gamma($\alpha, \beta$).
	$\expected	= \lambda + \alpha \beta$, $\variance = \lambda + \alpha \beta (1 + \beta)$.
	Mamy (dla $z = 1+6\lambda+6\lambda\beta+7\beta+12\beta^2+6\beta^3+3\alpha\beta(1+\beta)^2$):
	\[
		\gamma = \frac{\lambda + \alpha \beta(1 + 3 \beta + 2 \beta^2)}{(\lambda + \alpha \beta ( 1 + \beta))^{3/2}}
\,\bullet\,
		\kappa = \frac{\lambda + 3 \lambda^2 + \alpha \beta \cdot z}{(\lambda + \alpha \beta (1+ \beta))^2}
	\]

\item \textbf{Zeta} ($s > 1$).
	Gęstość $k^{-s} / \zeta(s)$.
	$\expected$ istnieje dla $s > 2$, $\variance$: dla $s >3$.
	Jest nieskończenie podzielny.
	\[
		\expected = \frac{\zeta(s-1)}{\zeta(s)} \,\bullet\,
		\variance = \frac{\zeta(s)\zeta(s-2) - \zeta(s-1)^2}{\zeta(s)^2}
	\]
\end{enumx}

\end{multicols*}
\end{document}
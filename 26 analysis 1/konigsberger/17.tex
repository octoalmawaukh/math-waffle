Punktowa \prawo{17.1} granica $G \colon \C \to \C$ funkcji $G_n(z)$ jest \prawo{17.2} ciągła \prawo{17.3} i ma zera w $-n$ ($n \in \N_0$).
Dla $k \in \N$, $z \neq 0$ jest $z G(z+1) = G(z)$, $(k-1)! G(k) = 1$.
\wcht{Funkcja Gamma}, $\Gamma(z) = 1 : G(z)$ (poza zerami), spełnia \textbf{prawo uzupełnień}: $\Gamma(x)\Gamma (1-x) = \pi / \sin(\pi x)$ ($x \in \R \setminus \N$) i jest \wcht{logarytmicznie wypukła} na $(0, \infty)$ (ma wypukły logarytm).
\wcht{Tw. Bohra-Mollerupa} (\datum{1922}): jeśli logarytmicznie wypukła $F: (0, \infty) \to \R_+$ spełnia $F(1) = 1$, $F(x+1) = xF(x)$, to $F \equiv \Gamma$.
Khm-2: całkowe przedstawienie Eulera, $x > 0$.
Khm-4: prawo \wcht{podwajania Legendre'a}, $x > 0$.
\wcht{Wzór Stirlinga} dla $x > 0$ oraz $0 < 12 \mu(x) x < 1$: khm-3.
Funkcja beta: $\Gamma(x) \Gamma(y):\Gamma(x+y)$, całka (,,$\textrm{d}t$'') z $t^{x-1} (1-t)^{y-1}$ nad $[0,1]$, pojawia się w stochastyce.
\[
	G_n(z) = \frac{z^{\overline{n+1}}}{n! n^z} \spk
	\Gamma(x+1) = \int_0^\infty \frac{t^{x}}{e^t} \,\textrm{d}t = \sqrt{2 \pi x} \cdot \frac{x^x}{e^x} \left(1 + \frac{n^{-1}}{12} + \frac{n^{-2}}{288} - \frac{139n^{-3}}{51840} - \ldots\right) \spk
	\frac{\Gamma(x)}{\Gamma(2x)} \Gamma(x + 1/2) = \frac{\sqrt \pi}{2^{2x-1}} 
\]
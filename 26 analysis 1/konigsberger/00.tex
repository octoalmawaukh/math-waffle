\wcht{Równanie kwadratowe}: \prawo{Wi} jeżeli $ax^2 + bx + c = 0$, to $2ax = -b \pm (b^2 - 4ac)^{1/2}$.
\wcht{Sześcienne} (metoda Viete'a): w równaniu $z^3 +az^2 + bz + c = 0$ można podstawić $z = x - a/3$ (kasuje wyraz z $z^2$, do $x^3 + px = q$) i $x = w - p / 3w$, co po przemnożeniu przez $t = w^3$ daje $t^2 - qt - (p/3)^3 = 0$.
Istnieją też mniej wygodne rachunkowo wzory Cardano.
Dla równania stopnia czwartego: wzory Ferrari, wyżej brak ogólnych rozwiązań (ze względu na teorię Galois i kłopotliwe $x^5 + x + 1 = 0$).
\wcht{Ciągłość} \prawo{7.1} funkcji $f \colon \C \to \C$: \wcht{zwykła} (każdym $\varepsilon > 0$ i $x$ \prawo{7.2} odpowiada $\delta > 0$, że $|x-y| < \delta \Ra |f(x) - f(y)| < \varepsilon$), \wcht{jednostajna} (każdemu $\varepsilon > 0$ odpowiada $\delta > 0$, że $|x-y| < \delta \Ra |f(x)-f(y)|<\varepsilon$), \wcht{Lipschitza} (istnieje $L > 0$, że $|f(x) - f(y)| \le L |x-y|$; dla $L < 1$: $f$ to \wcht{kontrakcja}) oraz \prawo{7.4} \wcht{Höldera} (istnieją $C > 0$ i $0 < a < 1$, że $|f(x) - f(y)| \le C |x-y|^a$).
Suma, produkt i złożenie ciągłych funkcji są ciągłe.
Odwrotna do injekcji z przedziału $[a,b]$ też.
\wcht{Tw. o wartości średniej}: ciągła funkcja $f \colon [a,b] \to \R$ przyjmuje wszystkie wartości między $f(a)$ i $f(b)$ co najmniej raz. %Pochodna ma własność Darboux.

Ciągła \prawo{7.5} funkcja ($(K \subseteq_k \C) \to \R$) jest ograniczona (Weierstraß), jednostajnie ciągła (Heine, Cantor) i osiąga swoje kresy.
\wcht{Tw. szlabanowe}: różalna funkcja na przedziale o ograniczonej pochodnej jest lipschitzowska.
\wcht{Tw. Aleksandrowa}: jeśli $(U \subset \R^n) \to \R^m$ jest wypukła, to jej druga pochodna istnieje p.w.
\wcht{Tw. Rademachera}: jeśli $(U \subseteq_o \R^n ) \to \R^m$ jest Lipschitza, to nie jest różalna na Le-zerowym zbiorze.

\wcht{Ciąg funkcyjny} \prawo{7.3} $f_n$ zbiega do $f$ (punktowo), gdy $f_n(x)$ dąży do $f(x)$ dla wszystkich argumentów z dziedziny.
Ograniczone funkcje mają skończoną \wcht{normę supremum}, $\|f\| := \sup\{|f(x)| : x \in D\}$.
Szereg $\sum_{n=1}^\infty f_n$ zbiega \wcht{normalnie}, gdy $f_n$ są ograniczone oraz $\sum_{n=1}^\infty \|f_n\| < \infty$.
Jeśli składniki $f_n$ są ciągłe, to cała suma także; zatem szeregi potęgowe definiują ciągłe funkcje w kole zbieżności.
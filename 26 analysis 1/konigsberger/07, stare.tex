\wcht{Ciągłość funkcji}: Cauchy, jednostajna, Lipschitz ($L < 1$: kontrakcja), Hölder. \prawo{7.1}
\wcht{Schrankensatz}: r--lna na przedziale z ograniczoną pochodną jest lipschitzowska (9.4).
Ciągła \prawo{7.5} na zwartym jest ograniczona (Weierstrass) i jednostajnie ciągła (Heine, Cantor).
Tw. o wartości średniej: dla ciągłej $f \colon [a,b] \to \R$, $f([a,b]) \supseteq (f(a), f(b))$.
%Ciągła $f:[a,b]\to \R$ przyjmuje wszystkie wartości między $f(a)$ i $f(b)$ (tw. o wartości średniej).
%Pochodna ma własność Darboux.
\wcht{Tw. Aleksandrowa}: jeśli $(U \subset \R^n) \to \R^m$ jest wypukła, to jej druga pochodna istnieje prawie wszędzie.
\wcht{Tw. Rademachera}: jeśli $(U \subseteq_o \R^n ) \to \R^m$ jest Lipschitza, to nie jest r--lna na zbiorze Le-miary zero.
\wcht{Ciąg funkcyjny} \prawo{7.3} $f_n$ zbiega do $f$ (punktowo): $f_n(x) \to f(x)$.
Szereg $\sum_{n=1}^\infty f_n$ zbiega \wcht{normalnie}: $\sum_{n=1}^\infty \sup |f(x)| < \infty$.
Jeśli $f_n$ są ciągłe, to także ich suma jest ciągła.

\begin{minipage}{.5\linewidth}
\vspace{1mm}
\begin{itemx}
\item[$\star_1$] $(\forall x, \varepsilon > 0) (\exists \delta > 0) (\forall y)(|x-y| < \delta \Ra |f(x) - f(y)| < \varepsilon)$ 
\item[$\star_2$] $(\forall \varepsilon>0) (\exists {\delta>0}) (\forall x, y) (|x-y| < \delta \Ra |f(x)-f(y)|<\varepsilon )$
\end{itemx}
\vspace{0.01mm}
\end{minipage}
\begin{minipage}{.5\linewidth}
\vspace{1mm}
\begin{itemx}
\item[$\star_3$] $(\exists L > 0)(\forall x, y)(|f(x) - f(y)| \le L |x-y|)$
\item[$\star_4$] $(\exists C > 0, a \in(0,1))(\forall x, y)(|f(x) - f(y)| \le C |x-y|^a)$
\end{itemx}
\vspace{0.01mm}
\end{minipage}
\wcht{Suma kompleksowa}: \prawo{8.1} $A + B = \{a+b : a \in A, b \in B\}$.
%\wcht{Problem Waringa}: dla jakiego $g_k$ jest $g_k \{n^k : n \in \N_0\} = \N_0$
\wcht{Fermat/Cauchy}: każda naturalna jest sumą co najwyżej $k$ liczb $k$-kątnych.
Chen pokazał, że każda odpowiednio duża parzysta jest sumą pierwszej $p$ i $a$ takiej, że $\Omega(a) \le 2$.
\wcht{Hipoteza Goldbacha}: nieparzyste $\ge 5$ są sumą co najwyżej trzech pierwszych.
Winogradow (\datum{1937}): każda nieparzysta $\ge x_0$ jest sumą trzech pierwszych; Chen, Wang: $x_0 \le \exp(\exp(9.715))$.
Jeśli $L$-funkcje Dirichleta nie mają zer w $\Re s> 1/2$ (\wcht{GRH}), to $x_0 = 9$.

Każdy $A \subset N_0$ ma \wcht{gęstość Schnirelmana}: $\delta(A) = \inf_{n} |A \cap [1, n]| : n$, a przy tym: $\delta(A) > 0 \Lra 1 \in A$, $d_*(A) \neq 0$ oraz $A = \N_0 \Lra \delta(A) = 1$.
Jeśli $1 \in A$ i $0 \in B$, to $\delta(A + B) \ge \delta(A) + \delta(B) - \delta(A)\delta(B)$.
Jeśli $0, 1 \in A$, to dla $k \in \N$ jest $\delta(kA) \ge 1- (1- \delta(A))^k$.
Jeśli $0 \in A$ i $\delta(A)>0$, to $A$ jest \wcht{bazą} $\N$ ($kA = \N$ dla pewnego $k$).
\wcht{Tw. Schnirelmana}: $c \mathbb P = \N_{\ge 2}$ dla pewnej $c$.
Ramaré: wystarczy $c = 7$.
Jeżeli zbiór $A$ zawiera $0$ i $\delta(A) > 0$, to $kA = \N$ dla $k \ge [\log 4] : [\log (1- \delta(A))]$.
Wzmocnienie: jeżeli $1 \in A \cup B$ i $0 \in A \cap B$, to $\delta(A+B) \ge \min\{1, \delta(A) + \delta(B)\}$.

% {
% Nam \prawo{8.2} dui \color{gray} ligula, fringilla a, euismod sodales, sollicitudin vel, wisi. Morbi auctor lorem non justo. Nam lacus libero, pretium at, lobortis
% vitae,ultricieset,tellus. Donecaliquet,tortorsedaccumsanbibendum,eratligulaaliquetmagna,vitaeornareodiometusami. Morbiacorci
% et nisl hendrerit mollis. Suspendisse ut massa. Cras nec ante. Pellentesque a nulla. Cum sociis natoque penatibus et magnis dis parturient
% montes, nascetur ridiculus mus. Aliquam tincidunt urna. Nulla ullamcorper vestibulum turpis. Pellentesque cursus luctus mauris.
% }

% {
% Nam \prawo{8.3} dui \color{gray} ligula, fringilla a, euismod sodales, sollicitudin vel, wisi. Morbi auctor lorem non justo. Nam lacus libero, pretium at, lobortis
% vitae,ultricieset,tellus. Donecaliquet,tortorsedaccumsanbibendum,eratligulaaliquetmagna,vitaeornareodiometusami. Morbiacorci
% et nisl hendrerit mollis. Suspendisse ut massa. Cras nec ante. Pellentesque a nulla. Cum sociis natoque penatibus et magnis dis parturient
% montes, nascetur ridiculus mus. Aliquam tincidunt urna. Nulla ullamcorper vestibulum turpis. Pellentesque cursus luctus mauris.
% }





% % \wcht{Problem Waringa}: dla pewnego $g_k$ jest $g_k A_k = \N_0$ ($A_k=\{0, 1, 2^k, 3^k,\dots\}$).
% % Każdy $A \subset N_0$ ma \wcht{Gęstość Schnirelmana}: $\delta(A) = \inf_{n \in \N} |A \cap [1, n]| / n$.
% % Fakty: $\delta(A) > 0$ $\Lra$ $1 \in A$, $d_*(A) \neq 0$; $A = \N_0$ $\Lra$ $\delta(A) = 1$; jeśli $1 \in A$ i $0 \in B$, to $\delta(A + B) \ge \delta(A) + \delta(B) - \delta(A)\delta(B)$.
% % Zatem jeśli $0, 1 \in A$, to dla $k \in \N$ jest $\delta(kA) \ge 1- (1- \delta(A))^k$.
% % Jeśli $0 \in A$ i $\delta(A)>0$, to $A$ jest \wcht{bazą} $\N$ ($A \subset N_0$ bazą naturalnych, jeśli $N_0 \subset kA$ dla pewnego $k$, rzędu bazy).
% % \wcht{Tw. Schnirelmana}: istnieje stała $C$, że naturalne $n \neq 1$ są sumą mniej niż $C$ l. pierwszych.
% % Jeżeli $1 \in A \cup B$ i $0 \in A \cap B$, to $\delta(A+B) \ge \min\{1, \delta(A) + \delta(B)\}$.

% % \wcht{Tw. Waringa-Hilberta}: \prawo{8.2} dla każdej $k \ge 2$ istnieje $s(k)$, że każda $n \in \N$ jest sumą $s(k)$ $k$-tych potęg, $g_k$ to ,,$\min$'' z $s(k)$.
% % Fakt: jest naturalna $k \ge 2$ i $p/q \not\in \Z$, że $p \perp q$; dla $H > 1$ definiujemy $S_k(H)$; wtedy istnieje $\delta_k > 0$, że $\sqrt H \le q \le H^{k-1/2}$ pociąga $|S_k(H)| = O(H^{1-\delta})$.
% % Dla $k \ge 7$ i $3^k - 2^k + 2 < (2^k-1)[(3/2)^k]$ jest $g_k = A(k)$; jeśli nie to $g_k = B(k)$ lub $B(k) - 1$ (to zależy od khm).
% % Dla $k \ge 2$ jest oczywiście $A(k) \le g(k)$, równość dla 3 (Wieferich), 4 (?), 5 (Chen), 6 (Pillai), 7 -- 400 (Dickson), 401 -- 200 000 (Stemmler), ... -- 471 600 000 (Kubina, Wunderlich), $k_0$ -- $\infty$ (Mahler).
% % Najmniejsze $G$, że odpowiednio duże są sumą $k$-tych potęg (?): dla $k \ge 3$ jest $G < g$, $G(2) = 4$ i $G(4) = 16$, dalej tylko oszacowania: $4 \le G_3 \le 7$, $G_5 \le 17$ i tak dalej.
% % Dla $k > 4000$ jest $G_k < 2k \log k + 2k\log \log k + 12 k$, dla $k > 14$ zaś $G_k < 3k \log k + 4.7 k$.
% % \[
% % 	S_k(H) = \sum_{j=1}^H \exp\left( 2 \pi ij^k \cdot \frac p q \right) \spk
% % 	A(k) = [(3/2)^k] + 2^k - 2 \spk
% % 	B(k) = A(k) + [(4/3)^k] \spk
% % 	\left( \left[\left(3/2\right)^k\right] + 1 \right)  \cdot \left( \left[\left(4/3\right)^k\right] + 1 \right) = 2^k+1
% % \]

% % Zagadkowo: prawy szereg jest zbieżny niemal jednostajnie w $|z| < 1$, absolutnie (partycje!).
% % Hardy, Ramanujan.
% % \[
% % 	1 + \sum_{n=1}^\infty p(n) z^n = \prod_{k=1}^\infty \frac{1}{1-z^k} \spk
% % 	p(n) = \frac 1 n \left( \frac {1}{4 \sqrt 3} + o(1) \right) \cdot \exp \left( \pi \frac{\sqrt{2n}}{\sqrt{3}} \right)
% % \]
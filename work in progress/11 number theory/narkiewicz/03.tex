% Równanie \prawo{3.1} $ax^2+bxy+cy^2+dx+ey=k$ (współczynniki z $\Z$) dla $|a|+|b|+|c|>0$ ma wyróżnik $\Delta=b^2-4ac$.
% Jeśli $\Delta=0$, to sprowadza się do liniowego lub kwadratowej kongruencji; jeśli nie, to do $AX^2+BY^2=C$ z $A,B,C\in\Z$.
% Jeśli $ABC\neq0$ i znamy rozwiązanie $(x_0,\pm y_0)\in\Q^2$, to pozostałe dostajemy z khm-1 ($\lambda\in\Q$); brak rozwiązań w $\Q\Lra$ w $\R$ lub ,,któregoś spośród: $a \mid x^2-bc$, $b \mid x^2-ca$, $c \mid x^2-ab$''.
% \wcht{Euklides}: wszystkie naturalne rozwiązania $x^2+y^2=z^2$ z $(x,y)=1$, $2\mid y$, są dane przez $(m^2-n^2,2mn,m^2+n^2)$ dla względnie pierwszych $m>n$.
% \wcht{Pell}: $x^2-dy^2=1$, ciekawe dla niekwadratu $d>0$.
% \wcht{Lagrange}: jeśli $(x_1,y_1)$ jest rozwiązaniem z $y_1>0$ i minimalnym $x_1>0$, to pozostałe spełniają $x_n+y_nd^{1/2}=(x_1+y_1d^{1/2})^n$; jest ich $\infty$-wiele oraz $x_{n+1}=x_1x_n+dy_1y_n$, $y_{n+1}=x_1y_n+y_1x_n$, $x_{n+1}=2x_1x_n-x_{n-1}$, $y_{n+1}=2x_1y_n-y_{n-1}$.
% \wcht{Lemat Dirichleta}: każdej $a\in\R$ i $N\in\N$ odpowiada $m/n\in\Q$ (skrócona), że $1\le n\le N$ i $|a-m/n| \le 1/(nN)$.
% \[
% 	\left(x_0-2\cdot\frac{Ax_0+\lambda By_0}{A+B\lambda^2},y_0-2\lambda\cdot\frac{Ax_0+\lambda By_0}{A+B\lambda^2}\right)
% \]
% {\color{Red} Stosunkowo szybką metodę testowania... strona 88.}

% %Równanie \prawo{3.2} $x^n+y^n = z^n$ nie ma rozwiązań dla $n \ge 3$ (Wiles, \datum{1995}). \\
% %Krzywe eliptyczne.
% {
% Nam \prawo{3.2} dui \color{gray} ligula, fringilla a, euismod sodales, sollicitudin vel, wisi. Morbi auctor lorem non justo. Nam lacus libero, pretium at, lobortis
% vitae,ultricieset,tellus. Donecaliquet,tortorsedaccumsanbibendum,eratligulaaliquetmagna,vitaeornareodiometusami. Morbiacorci
% et nisl hendrerit mollis. Suspendisse ut massa. Cras nec ante. Pellentesque a nulla. Cum sociis natoque penatibus et magnis dis parturient
% montes, nascetur ridiculus mus. Aliquam tincidunt urna. Nulla ullamcorper vestibulum turpis. Pellentesque cursus luctus mauris.
% }
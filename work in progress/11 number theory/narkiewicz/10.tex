% % \wcht{Ułamek łańcuchowy} \prawo{10.1} ma \wcht{redukty} ($a_1, \dots, a_n$: mianowniki).
% % Określamy $P_1=1$, $P_0(x_0) = x_0$ oraz $Q_{-1} = 0$ i $Q_0(x_0) = 1$; dalej tak, jak niżej.
% % Wtedy ułamkom łańcuchowym odpowiadają ilorazy; jest $r_{2m} \le K^{\infty}_{i=0} a_i  \le r_{2n+1}$.
% % Każdy $x \in \R$ ma co najmniej jeden, dokładnie $x \in \Q$ mają dokładnie dwa.
% % \wcht{tw 10.10}
% % \wcht{Tw. Hurwitza}: jeśli $\alpha \in \R \setminus \Q$, to dla $\infty$-wielu $p/q$ (z $p \perp q$) jest $|\alpha - p/q| < 1/(\sqrt 5 q^2)$, każda z nich jest reduktem $r_k = p_k/q_k$, zaś stała jest nie do poprawienia.
% % \wcht{Tw. Borela}: jeśli $k \ge 2$, to $|\alpha - r_j| < 1/ (\sqrt 5 q_k^2)$ co najmniej dla $j = k-2$, $k-1$ lub $k$.
% % \wcht{10.4}: reell-quadratisch : periodisch.
% % \[
% % 	\underset{i=0}{\overset{n}{\mathbf{K}}} a_i = a_0 + \left[\underset{i=1}{\overset{n}{\mathbf{K}}} a_i\right]^{-1} =_? \frac{P_n(a_0, \dots, a_n)}{Q_n(a_0, \dots, a_n)} \spk
% % 	(P, Q)_{k+1}(x_0, \dots, x_k) = x_{k+1} (P,Q)_k(x_0, \dots, x_k) + (P, Q)_{k-1}(x_0, \dots, x_{k-1}) \spk
% % \]


% Ciąg \prawo{10.2} $(a_n)$ w $[0,1)$ ma \wcht{ekwipartycję}, jeśli dla $0 \le a \le b \le 1$ jest khm-zbieżność.
% \wcht{Kryterium Weyla}: $\R$-ciąg $a_n$ ma ekwipartycję mod 1 $\Lra$ dla każdej $m \in \Z \setminus \{0\}$ jest khm-2.
% \wcht{Tw. van der Corputa}: jeśli ciąg $a_n$ ma tę własność, że wszystkie $b_n = a_{n+m} - a_n$ mają ekwipartycję, to sam ją ma.
% Ciąg $[nx]$ ma ekwipartycję $\Lra 1/x \in (\R \setminus \Q) \cup \Z$ (Niven).
% \[
%  	\frac{|\{k : k \le N, a\le a_k \le b\}|} N \to b-a \spk
% 	\lim_{N \to \infty} \frac 1 N \sum_{k=1}^N \exp (2\pi i m a_k) = 0
% \]
% \wcht{Podzielność}, \prawo{1.1} \wcht{kgV} \prawo{1.2} oraz \wcht{ggT}. 
% \wcht{Algorytm Euklid\wcht{Podzielność}, \prawo{1.1} \wcht{kgV} \prawo{1.2} oraz \wcht{ggT}. 
% \wcht{Algorytm Euklidesa}.
% Dzielenie z resztą: $a = bq+r$ (quotient, rest) dla $0 \le r < |b|$.
% Binarne znajdowanie ggT (Silver, Terzian \datum{1962}): jeśli obie liczby są parzyste, to połowimy je i zapamiętujemy czynnik $2$, gdy tylko jedna jest parzysta, to połowimy ją, jeśli żadna: odejmujemy mniejszą od większej. 
% Jeśli $n_1, n_2, \dots$ dzielą $n$, to $(n_1, \dots, n_k)[n/n_1, \dots, n / n_k] = n$.
% Losowe liczby są kopierwsze z p-stwem $6/\pi^2$. 
% \wcht{Równanie diofantyczne}: \prawo{1.3} rozwiązania w $\Z$.
% Jeśli $a_i \in \Z \setminus \{0\}$, to $\sum_{j=1}^n a_j X_j = b$ ma rozwiązanie w $\Z$ $\Lra$ $(a_1, \dots, a_n) \mid b$; wtedy można wybrać takie, że $|X_i| \le |b/H| + (n-1)H/2$ (i $H = \max_i |a_i|$).
% \wcht{Problem Frobeniusa}: jeśli $(a,b)=1$, to wszystkie większe od $ab-a-b$ zapisują się jako $ax+by$ dla $x,y \ge 0$.esa}.
% Dzielenie z resztą: $a = bq+r$ (quotient, rest) dla $0 \le r < |b|$.
% Binarne znajdowanie ggT (Silver, Terzian \datum{1962}): jeśli obie liczby są parzyste, to połowimy je i zapamiętujemy czynnik $2$, gdy tylko jedna jest parzysta, to połowimy ją, jeśli żadna: odejmujemy mniejszą od większej. 
% Jeśli $n_1, n_2, \dots$ dzielą $n$, to $(n_1, \dots, n_k)[n/n_1, \dots, n / n_k] = n$.
% Losowe liczby są kopierwsze z p-stwem $6/\pi^2$. 
% \wcht{Równanie diofantyczne}: \prawo{1.3} rozwiązania w $\Z$.
% Jeśli $a_i \in \Z \setminus \{0\}$, to $\sum_{j=1}^n a_j X_j = b$ ma rozwiązanie w $\Z$ $\Lra$ $(a_1, \dots, a_n) \mid b$; wtedy można wybrać takie, że $|X_i| \le |b/H| + (n-1)H/2$ (i $H = \max_i |a_i|$).
% \wcht{Problem Frobeniusa}: jeśli $(a,b)=1$, to wszystkie większe od $ab-a-b$ zapisują się jako $ax+by$ dla $x,y \ge 0$.
\wcht{Skojarzenie}: \prawo{1.1} zbiór niezależnych krawędzi, $k$-\wcht{faktor}: $k$-regularny podgraf rozpinający.
$H \subseteq G$ to 1-faktor $\Lra$ krawędzie $H$ kojarzą każdy wierzchołek z $V$.
Ustalmy dwudzielny $G$ z rozbiciem wierzchołków $\{A, B\}$. 
\wcht{Droga alternująca}: droga w $G$, zaczyna się od niesparowanego w $A$ i skacze między $M$ oraz $E \setminus M$.
\wcht{Droga poprawcza}: alternująca, kończy się w niesparowanym z $B$.
\wcht{Tw. Königa} (\datum{1931}): najmniejszna moc z pokryć wierzchołkowych ($U \subseteq V$, który sąsiaduje z każdą krawędzią $G$) to największa moc skojarzenia.
\wcht{Tw. Halla} (\datum{1935}): w grafie $G$ istnieje skojarzenie dla $A$ $\Lra$ $|N(S)| \ge |S|$ dla wierzchozbiorów $S \subseteq A$ ($N(S)$ to punkty sąsiadujące z $S$). 
Wniosek: $[k \ge 1]$-regularny graf ma $1$-faktor.
\wcht{Wniosek Petersena} (\datum{1891}): regularny graf parzystego stopnia $> 0$ ma $2$-faktor.
\wcht{Tw. Slutskiego}: \prawo{5.2} $X_n \wgrozkladu X$, $A_n \wgpstwa a$, $B_n \wgpstwa b$ pociągają $A_n + B_nX_n \wgpstwa a + bX$.
Jeśli $X_n \wgrozkladu X$, to $X_n$ są \wcht{ograniczone wg p-stwa}: każde $\varepsilon > 0$ ma $B$, że $\pstwo (|X_n| \le B) \ge 1 - \varepsilon$ dla dużych $n$.
Jeśli jeszcze $Y_n \wgpstwa 0$, to $X_nY_n \wgpstwa 0$.
\wcht{$\Delta$-metoda}: jeżeli $n^{1/2} (X_n - \theta) \wgrozkladu \mathcal N(0, \sigma^2)$, zaś $g'(\theta) \neq 0$ istnieje, to $n^{1/2} (g(X_n) - g(\theta)) \wgrozkladu \mathcal N(0, \sigma^2 (g'(\theta))^2)$.
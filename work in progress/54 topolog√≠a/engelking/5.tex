Zwarta, \prawo{5.1} $\mathfrak L$ lub metryzowalna $\Ra$ \wcht{parazwarta} (Dieudonné, \datum{1944}): $\mathcal T_2$, w pokrycie można wpisać otwarte, lokalnie skończone $\Ra$ \wcht{kolektywnie normalna} ($\mathcal T_1$, dyskretna rodzina $\{F_s\}$ domkniętych ma rodzinę otwartych $V_s \supseteq F_s$) i $\mathcal T_4$.
\wcht{Rozkład jedności}: rodzina ciągłych $f_s \colon X \to [0,1]$, że $\sum_s f_s \equiv 1$, \wcht{lokalnie skończony}: pokrycie $\{f^{-1}_s(0,1]\}$ lokalnie skończone; \wcht{drobniejszy} od pokrycia $\mathcal A$: rodzina wcześniej wpisana w $\mathcal A$.
Dla $M \subseteq X$ \wcht{gwiazdą} względem pokrycia $\mathcal A = \{A_s\}$ jest $\operatorname{Gw}(M, \mathcal A)  = \bigcup\{A_s : A_s \cap M \neq \varnothing\}$.
Pokrycie $\{A_s\}$ wpisane \wcht{gwiaździście} w inne, $\{B_t\}$: $(\forall s)(\exists t)(\operatorname{Gw}(A_s, \mathcal A) \subseteq B_t)$.
\wcht{Punktowo gwiaździście}: $(\forall x \in X)(\exists s)(\operatorname{Gw}(x, \mathcal A) \subseteq B_t)$.
\wcht{Pokrycie normalne} $\mathcal W$: jest otwarte, istnieje ciąg $\mathcal W_i$ pokryć otwartych, że $\mathcal W_1 = \mathcal W$, a każde jest wpisane gwiaździście w poprzednie.

Dla $\mathcal T_1$ NWSR: para-; otwarte pokrycia mają drobniejsze od siebie (lokalnie skończone/dowolne) rozkłady jedności; w otwarte pokrycia wpisują się (punktowo) gwiaździście otwarte; $\mathcal T_3$ i w pokrycia otwarte wpisują się otwarte, $\sigma$-dyskretne.
Dla $\mathcal T_3$: w pokrycia otwarte można wpisać (otwarte, $\sigma$-lokalnie skończone) $\Lra$ (lokalnie skończone) $\Lra$ (domknięte lokalnie skończone), przestrzeń jest para-.
Zwarta $\Lra$ zwarta przeliczalnie (dla para-).
Para- z gęstym podzbiorem $\mathfrak L$ sama jest $\mathfrak L$.
Lokalnie zwarte, para- jest sumą rozłącznych otwarniętych podprzestrzeni, które są $\mathfrak L$.
Para- dziedziczy się na $F_\sigma$; suma jest para- $\Lra$ składniki są.
Para- to niezmiennik domkniętych (\wcht{tw. Michaela}, \datum{1957}).
Jest także dla doskonałych przeciwniezmiennikiem.
Produkt para- ze zwartą jest para-; klasa para- jest doskonałą.
\wcht{Tw. Tamano}: dla $\mathcal T_{3.5}$ NWSR: $X$ jest parazwarta, produkt $X$ przez uzwarcenie $rX$ (każde, pewne lub $\beta$) jest $\mathcal T_4$.
NWSR: $X$ parazwarta; $X \times Y$ jest $\mathcal T_4$ dla każdej $Y$: (zwartej), (zwartej, że $\operatorname{w}(Y) \le \operatorname{w}(X)$), ($ Y=I^{\operatorname{w}(X)}$).

Parazwarta, przeliczalnie zwarta lub $\mathcal T_6$ \prawo{5.2} $\Ra$ \wcht{przeliczalnie parazwarta}: $\mathcal T_2$, w każde przeliczalne pokrycie otwarte można wpisać otwarte, lokalnie skończone $\Lra$ wpisane pokrycie może być postaci $V_i \subseteq U_i$ $\Lra$ jeżeli $W_i \subseteq_o X$ wstępują do $X$, to istnieją $F_i \subseteq^a X$, $F_i \subseteq U_i$, których to wnętrza sumują się do $X$ $\Lra$ prawa de Morgana dla poprzednika [wszędzie $\mathcal T_2$!] (Dowker/Katetov \datum{1951}).
Rodzina $A_s$ \wcht{gwiaździście skończona} [przeliczalna]: każdy zbiór kroi skończenie [przeliczalnie] wiele innych.
Dla $\mathcal T_4$: przeliczalnie parazwarta $\Lra$ w przeliczalne pokrycie otwarte można wpisać pokrycie otwarte gwiaździście / punktowo skończone (Iseki, \datum{1954}).
Produkt przeliczalnie parazwartej, która jest $\mathcal T_4$ ze zwartą, dwa-przeliczalną jest znowu $\mathcal T_4$.
Produkt $X \times [0,1]$ jest $\mathcal T_4$ $\Lra$ $X$ jest przeliczalnie parazwarta, $\mathcal T_4$.
(Fleissner \datum{1978}): jeżeli $A \subseteq^a X$ jest dyskretna w przeliczalnie parazwartej $X$, to $|A| < \exp d(X)$. %, więc kwadrat strzałki, płaszczyzna Niemyckiego i $\N^{\mathfrak c}$ nie są przeliczalnie parazwarte.
(Aull \datum{1965}): raz-przeliczalna przeliczalnie parazwarta jest $\mathcal T_3$.

$\mathfrak L$ \prawo{5.3} $\Ra$ mocno parazwarta (\wcht{hypo-}: $\mathcal T_2$, w każde otwarte pokrycie wpisuje się otwarte gwiaździście skończone; Dowker \datum{1947}) $\Ra$ para- $\Ra$ słabo parazwarta (\wcht{meta-}: $\mathcal T_2$, w każde otwarte pokrycie wpisuje się otwarte, punktowo skończone; Arens, Dugundji \datum{1950}).
Meta-, $\mathcal T_4$ $\Ra$ parazwarta przeliczalnie.
Każde punktowo skończone ma podpokrycie \wcht{nieprzywiedlne} (nie można niczego pominąć).
Dla meta-: zwarta $\Lra$ przeliczalnie zwarta.
\wcht{Tw. Michaela-Nagamiego} (\datum{1955}): kolektywnie normalna, meta- $\Ra$ para-.
\wcht{Tw. Worrella} (\datum{1966}): jeśli istnieje domknięta surjekcja $f \colon X$ (meta-) $\to Y$ ($\mathcal T_2$), to $Y$ też jest meta-.
Dla $\mathcal T_3$ NWSR: hypozwartość $\Lra$ w pokrycie otwarte $X$ można wpisać (domknięte, lokalnie i gwiaździście skończone) / (lokalnie skończone, domknięte, gwiaździście przeliczalne) / (otwarte, gwiaździście przeliczalne). (Smirnow \datum{1956}).

Traylor, Hodel (\datum{1964}/\datum{1969}): lokalnie ośrodkowa, $\mathcal T_3$, meta- $\Ra$ hypo-
Stone (\datum{1962}): jeśli w każde otwarte pokrycie można wpisać otwarte, punktowo przeliczalne i każdy $x$ ma $U_x$, że $w(U_x) \le \mathfrak m$, to $X = \bigoplus_s X_s$, $w(X_s )\le \mathfrak m$.
%Charlesworth (\datum{1976}): 
Meta- i hypo- są addytywne.
Meta- dziedziczy się na $F_\sigma$ (Czoban, \datum{1970}), hypo- na domknięte (ale $F_\sigma$ nie!).
Żadne z nich nie jest skończenie multiplikatywne.
Meta- nie są zamknięte na branie ,,gso''.
Hanai (\datum{1956}): meta- i hypo- to przeciwniezmienniki doskonałych.
Ponomariow (\datum{1962}): hypo- to niezmiennik otwartych doskonałych (ale nie: doskonałych).
Worrell (\datum{1966}): parazwartość jest niezmiennikiem domkniętych.
Morita (\datum{1954}): metryzowalna, hypo-, ciężaru $\mathfrak m \ge \aleph_0$ zanurza się w iloczynie $[0,1]^{\aleph_0} \times \mathfrak B(\mathfrak m)$.

{\color{gray}
Baza \prawo{5.4} $\mathcal B$ dla $X$ jest \wcht{punktowo regularna}, gdy dla każdych $x \in X$, $U_x$ skończenie wiele bazowych otoczeń $x$ kroi $X \setminus U$ (Aleksandrow, \datum{1960}).
\wcht{Regularna}: skończenie wiele bazowych kroi (ustalone) $V_x \subseteq U_x$ i $X \setminus U$.
Ciąg $\mathcal W_k$ otwartych pokryć $X$ \wcht{miałki punktowo}: każde $x$, $U_x$ mają $n$, że $\operatorname{Gw}(x, \mathcal W_n) \subseteq U_x$.
\wcht{Miałki}: $n$ i $V_x \ni x$, że $\operatorname{Gw}(V_x, \mathcal W_n) \subseteq U_x$ (\datum{1919}, Chittenden, Pitcher).
\wcht{Aleksandrow-Urysohn} (\datum{1923}): $\mathcal T_0$ oraz ma ,,pmcp'' $\mathcal W_k$, że gdy dwa elementy $\mathcal W_{n+1}$ się kroją, to oba są zawarte w pewnym z $\mathcal W_n$.
\wcht{Moore} (\datum{1935}): $\mathcal T_0$ i ma ,,mcp''.
\wcht{Bing} (\datum{1951}): ma ,,pmcp'' oraz jest kolektywnie normalna.
\wcht{Archangielski} (\datum{1960}): $\mathcal T_1$ i ma ,,rb''.
\wcht{Aleksandrow} (\datum{1960}): kolektywnie normalna i ma ,,prb''.
% Dla $T_2$, $X$, NWSR: $X$ ma punk-reg-bazę; ma punk-miał-ciąg pokryć i jest parazwarta; ma punk-miał-ciąg pokryć punk-skończonych.
Smirnow (\datum{1951}): lokalnie metryzowalna oraz parazwarta $\Ra$ metryzowalna.
,,Twierdzenia o metryzacji II''.
Dla ,,prb'': $|\mathcal B| \le \aleph_0 \cdot d(X)$.
}
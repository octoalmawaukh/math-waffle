\wcht{Pierścień}: \prawo{0.0} niepusty $R$ z działaniami $+$ i $\times$, że $(R,+)$ to grupa abelowa, $\times$ jest rozdzielne względem $+$ i łączne.
\wcht{Dziedzina}: bez \wcht{dzielników zera} ($a$, $b \neq 0$, $ab = 0$); \wcht{całkowitości} (IB): przemienna.
\wcht{HIR}: IB, każdy ideał jest \wcht{główny} (jeden generator).
\wcht{EPZ}: każdy $a \neq 0$ ma jednoznaczny rozkład jako iloczyn pierwszych.
\wcht{Dziedzina Euklidesa}: IB z funkcją $f \colon R \setminus \{0\} \to \N_0$, gdy dla $b \neq 0$ i $a \in R$ istnieją $q, r \in R$, że $a = bq+r$ oraz $f(r) < f(b)$ lub $r = 0$.
\wcht{Ciało}: pierścień $R$, że $R \setminus \{0\}$ jest abelową grupą $\Ra$ ER $\Ra$ HIR $\Ra$ EPZ.

\wcht{Ideał} $I \trk R$: podgrupa zamknięta na mnożenie przez $r \in R$.
\wcht{Centrum} ($c \in R$, że $cr = rc$ dla $r \in R$) jest podpierścieniem, lecz niekoniecznie ideałem.
Ideały można dodawać i mnożyć kompleksowo.
\wcht{Pierścień ilorazowy}: $R/I$ (przez analogię do grup).
Element $p$ w przemiennym jest \wcht{pierwszy}: niezero, niejedność, ,,$p \mid ab \Ra p \mid a$ lub $p \mid b$'' $\Lra$ $(p)$ jest \wcht{ideałem pierwszym} ($I \neq R$, $ab \in I \Ra \{a,b\} \cap I \neq \varnothing$), niezerowym.
Ideał maksymalny: względem $\subseteq$, różny od $R$.
Fakt: $R/I$ to ciało $\Lra$ $I$ maksymalny, $R/I$ IB $\Lra$ $I$ pierwszy.

ER są: $\Z$ i $\Z[i] \subseteq \C$ z $x \mapsto |x|$, $\Z[\omega]$ z $a + b\omega \mapsto a^2-ab+b^2$, $\mathfrak K[x]$ ze stopniem, $\mathfrak K[[x]]$ ze stopniem najmniejszej potęgi $x$. 
$\Z[x]$ i $\mathfrak K[x,y]$ nie są HIR.
W każdym HIR jest \wcht{największy wspólny dzielnik}: generator $(a,b)$.
W pierścieniach z jedynką ideał maksymalny jest pierwszy; w HIR niezerowy pierwszy jest maksymalny.
EPZ to IB, w którym niezerowe elementy są iloczynem jedności i elementów pierwszych.
$R$ jest EPZ $\Ra R[x]$ jest EPZ.
$\mathfrak K[[x_1, \ldots, x_n]]$ jest EPZ (zamiast $\mathfrak K$ można wziąć HIR).
Formalny pierścień $\C[x]$ jest EPZ, ale podpierścień zbieżnych wszędzie (entiére) nie: $\sin \pi z = \pi z \prod_{n \ge 1} (1 - z^2/n^2)$.
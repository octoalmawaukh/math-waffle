Jeśli $f \colon A \to B$ \prawo{2.4} jest homo- pierścieni, zaś $S \subseteq A$ \wcht{multiplikatywny} (zawiera jedynkę i jest zamknięty na iloczyny) oraz każdy homo- $g \colon A \to C$ przenoszący elementy $S$ na jedności ma dokładnie jeden homo- $h \colon B \to C$, że $g = hf$, to $B$ jest \wcht{lokalizacją} $A$, $B = S^{-1}A = A_S$ z kanoniczną $f$.
Dokładniej, $B$ to zbiór par $(a,s)\in A \times S$ utożsamionych przez: $(a, s) \sim (b, t) \Lra \exists u \in S$, że $u(ta-sb) = 0$.
Jeżeli $S$ to niedzielniki zera, to $A_S$ jest ,,\wcht{ciałem ułamków}''.
Ideał $J \trk B$ jest \wcht{prymarny}, gdy nie zawiera jedynki, zaś dzielniki zera $B/J$ są nilpotentami.
Ideały $A_S$ to dokładnie $IA_S$, gdzie $I \trk A$.
Pierwsze (prymarne) ideały $A_S$ to dokładnie $\mathfrak p A_S$, przy czym $\mathfrak p \trk A$ jest pierwszy (prymarny) i rozłączny z $S$.
Efekt uboczny: bycie Noether (Artina) dziedziczy się z $A$ na $A_S$.
Lokalizacja ,,komutuje'' z dzieleniem: jeśli $I \trk A$, $S \subseteq A$ jest multiplikatywny i $T$ to obraz $S$ w $A/I$, to $A_S/IA_S \cong (A/I)_S$.
\wcht{Ciało residuów}: iloraz przez maksymalny ideał.
\wcht{Widmo}: rodzina ideałów pierwszych (Spec) albo maksymalnych (m-Spec).
Jeśli $I \trk A$, to $V(I) := \{\mathfrak p \in \operatorname{Spec} A : \mathfrak p \supseteq I\}$; uznając $V(I)$ za zbiory domknięte, dostajemy \wcht{topologię Zaryskiego}.
Dla $M$, $A$-modułu, również można zbudować lokalizację; $M_S \cong M \otimes_A A_S$.
%Funktor $M \mapsto A_S$ jest dokładny i kowariantny (z kategorii $A$-modułów w $A_S$-modułów).
Spektrum jest zwarte bez $\mathcal T_2$; jego niespójność pociąga istnienie trzeciego (różnego od $0, 1$) idempotentu, czyli $e$ takiego, że $e^2 = e$.
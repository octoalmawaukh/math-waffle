Mamy \prawo{GF} $\dot x = A(t)x$, macierz $n \times n$ z okresem $T$.
Jeśli $\Phi(t)$ jest macierzą fundamentalną, to $\Phi(t+T) = \Phi(t) \Phi^{-1}(0)\Phi(T)$.
Dalej, jeśli $\exp(TB) = \Phi^{-1}(0)\Phi(T)$ (to \wcht{macierz monodromii}), to istnieje $T$-okresowa macierz $P(t)$ ($n \times n$), że $\Phi(t) = P(t) \exp(tB)$: tutaj $B$ może być zespolona.
Do tego jest rzeczywista macierz $R$ i rzeczywista $2T$-okresowa $Q(t)$, że $\Phi(t) = Q(t)\exp(tR)$ (to było \wcht{tw. Floqueta}, \datum{1883}).

% http://mathworld.wolfram.com/FloquetsTheorem.html
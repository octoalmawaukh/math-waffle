\documentclass[a4paper, fleqn, 9pt]{extarticle}
\usepackage{etoolbox}
\usepackage{Alegreya, euler}

\usepackage{amssymb, mathrsfs}
\usepackage[left=1.5cm,top=2.5cm,right=1.5cm,bottom=1.5cm]{geometry}
\usepackage{multicol}
\usepackage{MnSymbol}

\usepackage{fancyhdr}
\makeatletter
\fancypagestyle{mypagestyle}
{\newpage \fancyfoot[C]{} \renewcommand{\footrulewidth}{0pt}}
\makeatother
\pagestyle{mypagestyle}
\headsep 5pt                 %% put this outside
\usepackage{lastpage}

\usepackage[usenames,dvipsnames]{color}
\usepackage{xcolor}

\newcommand{\prawo}[1]{\marginpar{\textbf{\underline{#1}}}}
\newcommand{\wcht}[1]{{\bf {#1}}}
\newcommand{\spk}{\, \bullet \,}
\newcommand{\datum}[1]{{\color{Purple} \bf {#1}}}

% ? - pożyteczne zbiory
\newcommand{\K}{\mathbb{K}}
\newcommand{\C}{\mathbb{C}}
\newcommand{\N}{\mathbb{N}}
\newcommand{\R}{\mathbb{R}}
\newcommand{\Q}{\mathbb{Q}}
\newcommand{\Z}{\mathbb{Z}}
\newcommand{\Qp}{\mathbb{Q}_p}
\newcommand{\Zp}{\mathbb{Z}_p}

% Logika
\newcommand{\Lra}{\Leftrightarrow}
\newcommand{\Ra}{\Rightarrow}

% Algebra
\newcommand{\trk}{\trianglelefteq}
\newcommand{\im}{\operatorname{im}}
\newcommand{\autgrp}{\operatorname{Aut}}
\newcommand{\inngrp}{\operatorname{Inn}}
\newcommand{\outgrp}{\operatorname{Out}}
\newcommand{\holmrf}{\operatorname{Hol}}























% Analiza
%\newcommand{\D}{\textrm{d}}
%\DeclareMathOperator{\grad}{grad}
%\DeclareMathOperator{\rot}{rot}
%\DeclareMathOperator{\dvrg}{div}

% Miara
%\newcommand{\Rr}{\overline \R}

% Algebra abstrakcyjna

%\newcommand{\krt}{\trianglerighteq}


% Topologia
%\DeclareMathOperator{\cl}{cl}
%\DeclareMathOperator{\interior}{int}

% Pstwo
%\DeclareMathOperator{\pstwo}{\mathcal{P}}
%\DeclareMathOperator{\esssup}{ess\,sup}
%\newcommand{\expected}{\mathbb{E}}
%\newcommand{\variance}{\mathbb{V}}
%\newcommand{\prawie}{\leadsto}
%\newcommand{\wgpstwa}{\rightarrowtail}
%\newcommand{\wgrozkladu}{\twoheadrightarrow}

% Liniowa
%\newcommand{\rank}{\operatorname{rank}}
%\newcommand{\chara}{\operatorname{char}}
%\newcommand{\pf}{\operatorname{Pf}}
%\newcommand{\ann}{\operatorname{Ann}}

% \DeclareMathOperator{\autom}{\mathcal A}
% \DeclareMathOperator{\centror}{\mathcal C}
% \DeclareMathOperator{\chara}{char}
% \DeclareMathOperator{\cl}{cl}
% \DeclareMathOperator{\covariance}{cov}
% \DeclareMathOperator{\diam}{diam}
% \DeclareMathOperator{\disexp}{Exp}
% \DeclareMathOperator{\esssup}{ess\,sup}
% \DeclareMathOperator{\grupapsl}{\mathrm{PSL}}
% \DeclareMathOperator{\holomorf}{\mathcal H}
% \DeclareMathOperator{\innerm}{\mathcal I}
% \DeclareMathOperator{\interior}{int}
% \DeclareMathOperator{\normor}{\mathcal N}
% \DeclareMathOperator{\outerm}{\mathcal O}
% \DeclareMathOperator{\rad}{rad}
% \DeclareMathOperator{\rot}{rot}
% \DeclareMathOperator{\zenter}{\mathcal Z}
% \newcommand{\Ab}{\catname{Ab}}
% \newcommand{\Aut}{{\textrm {Aut}}}
% \newcommand{\bigslant}[2]{{\raisebox{.2em}{$#1$}\left/\raisebox{-.2em}{$#2$}\right.}}
% \newcommand{\calka}[4] {\int_{\farbea{#1}}^{\farbeb{#2}} #3 \, \textrm{d}{#4}}
% \newcommand{\catname}[1]{{\normalfont\textbf{#1}}}
% \newcommand{\cf}{\text{cf }}
% \newcommand{\Ci}{\operatorname{Ci}}
% \newcommand{\col}{\mathit{Col}}
% \newcommand{\cov}{\text{cov }}
% \newcommand{\Cp}{\mathbb{C}_p}
% \newcommand{\CRng}{\catname{CRng}}
% \newcommand{\dotr}[1]{#1^{\cdot}} 
% \newcommand{\dvrg}{\textrm{div }}
% \newcommand{\dwumk}[2] {\left[{#1 \atop #2}\right]}
% \newcommand{\dwumo}[2] {\left\{{#1 \atop #2}\right\}}
% \newcommand{\dwump}[2] {\left\{\begin{matrix} #1\\ #2\\ \end{matrix} \right\}}
% \newcommand{\dwum}[2] {\left({#1 \atop #2}\right)}
% \newcommand{\Ei}{\operatorname{Ei}}
% \newcommand{\ex}{{\textrm E}}
% \newcommand{\Fp}{\mathbb{F}_p}
% \newcommand{\grad}{\text{grad }}
% \newcommand{\Grp}{\catname{Grp}}
% \newcommand{\hipergeo}[3] {\left(\left.{#1 \atope #2}\,\right|\,#3\right)}
% \newcommand{\igr}{{\textrm Y}}
% \newcommand{\iks}{{\textrm X}}
% \newcommand{\imaz}{\mathfrak{Im } }
% \newcommand{\imp}{\bf}
% \newcommand{\inj}{\hookrightarrow}
% \newcommand{\kalendarz}{\color{wordblu} \bf}
% \newcommand{\kcod}{\textrm{cod }}
% \newcommand{\kdom}{\textrm{dom }}
% \newcommand{\khom}{\textrm{hom}}
% \newcommand{\kid}[1]{\textrm{id}_{#1} }
% \newcommand{\kkcod}{\textrm{cod }}
% \newcommand{\kkdom}{\textrm{dom }}
% \newcommand{\kkid}[1]{\textrm{id}_{#1} }
% \newcommand{\krt}{\trianglerighteq}
% \newcommand{\K}{\mathbb{K}}
% \newcommand{\li}{\operatorname{li}}
% \newcommand{\lk}{\mathit{lk}}
% \newcommand{\mydelta}{{\color{wordred}\delta}}
% \newcommand{\mysigma}{{\color{wordred}\sigma}}
% \newcommand{\oprot}{\text{rot }}
% \newcommand{\pf}{\text{Pf }}
% \newcommand{\pocho}[2] {{#1}^{\overline{#2}}}
% \newcommand{\pochu}[2] {{#1}^{\underline{#2}}}
% \newcommand{\PSL}{\text{PSL}}
% \newcommand{\rank}{\text{rank }}
% \newcommand{\reel}{\mathfrak{Re }\, }
% \newcommand{\rhn}[1] {\left \lsem {#1} \right \rsem}
% \newcommand{\Span}{\mathit{span}}
% \newcommand{\sqrr}[1] {#1^{1/2}}
% \newcommand{\stda}{\color{wordblu}}
% \newcommand{\stdb}{\color{wordblu}}
% \newcommand{\summ}{\sum_{m=1}^\infty}
% \newcommand{\sumn}{\sum_{n=1}^\infty}
% \newcommand{\surj}{\twoheadrightarrow}
% \newcommand{\Top}{\catname{Top}}
% \newcommand{\tostar}{\ensuremath{\mathaccent\star\to}}
% \newcommand{\unif}{\rightrightarrows}

\usepackage{graphicx}
\relpenalty=10000
\binoppenalty=10000
\interlinepenalty=10000

\newenvironment{enumx}{\begin{enumerate} \setlength{\itemsep}{0pt} \setlength{\parskip}{0pt} \setlength{\parsep}{0pt}}{\end{enumerate}}
\newenvironment{itemx}{\begin{itemize} \setlength{\itemsep}{0pt} \setlength{\parskip}{0pt} \setlength{\parsep}{0pt}}{\end{itemize}}


\usepackage[polish]{babel}
\usepackage[utf8]{inputenc}
\usepackage[T1]{fontenc}
\selectlanguage{polish}

\begin{document}
\setlength{\belowdisplayskip}{2pt}
\setlength{\belowdisplayshortskip}{2pt}
\setlength{\abovedisplayskip}{2pt}
\setlength{\abovedisplayshortskip}{2pt}

\renewcommand{\footrulewidth}{0.4pt}
\fancyhead[LE,LO]{Zwyczajne równania różniczkowe (MSC 34)}
\fancyhead[RO,RE]{leon.aragones@gmail.com, listopad 2015}
\fancyfoot[RF]{Aulbach, czerwiec 1997}

Obiekty \prawo{1.5} $a, b$ \wcht{izomorficzne}: strzałka $e \colon a \to b$ jest \wcht{odwracalna} w $C$  (istnieje $e' \colon b \to a$ w $C$, że $e'e = 1_a$, $ee' = 1_b$, ,,$e^{-1}$''). 
Strzałka $m \colon a \to b$ \wcht{mono-} w $C$: dla równoległych $f_1, f_2 \colon d \to a$ równość $m \circ f_1 = m \circ f_2$ pociąga $f_1 = f_2$.
W \textsc{Set} i \textsc{Grp} mono- to injekcje.
Strzałka $h \colon a \to b$ \wcht{epi-} w $C$: dla strzałek $g_1, g_2 \colon b \to c$ równość $g_1 \circ h = g_2 \circ h$ pociąga $g_1 = g_2$.
W \textsc{Set} są to surjekcje.
\wcht{Prawa odwrotność} dla $h$: $r \colon b \to a$, że $hr = 1_b$, sekcja.
\wcht{Lewa odwrotność}: analogicznie, retrakcja.
Strzałki z sekcjami $\Ra$ epi-, $\Leftarrow$ dla \textsc{Set}, ale nie \textsc{Grp}.
Strzałki z retrakcjami są mono-.
Jeżeli $gh=1_a$, to $g$ jest \wcht{rozdartym epi-}, $h$ rozdartym mono, zaś $f=hg$ jest idempotentna.
Do obiektu \wcht{terminalnego} (z \wcht{inicjalnego}) prowadzi po jednej strzałce z (do) każdego.
\wcht{Zerowy}: taki i taki.
Epi- mono- może się nie odwracać!
\wcht{Grupoid}: kategoria bez nieodwracalnych strzałek.

\fancyfoot[LF]{strona \thepage { }z \pageref{LastPage} [od 1 do 212]}

Każdy ciąg zstępujących \prawo{2.X} przedziałów $[a_n, b_n]$ długości dążącej do zera (\wcht{gnieżdżący się}) wyznacza pewną liczbę rzeczywistą.
Przykłady: pierwiastek z $a_0b_0$ i $a_{n+1} = H(a_n, b_n)$, $b_{n+1} = A(a_n, b_n)$; średnia arytmetyczno-geometryczna.
\wcht{Ciąg liczbowy}: odwzorowanie $\N \to \C$.
Jeśli $a_n > 0$ i $a_{n+1}/a_n \to a$, to $n$-ty pierwiastek z $a_n$ też dąży do $a$.
\wcht{Dzielenie mnożeniem}: $x_{n+1} = x_n(2-ax_n)$ zbiega kwadratowo do $1/a$ dla $0 < a x_0 < 2$.
Podobnie można szukać pierwiastka z $a$ ($x_n$ zbiega kwadratowo, $y_n$: sześciennie).
\wcht{Tw. o kanapce}: jeśli $a_n\le x_n\le b_n$, a przy tym $a_n$ oraz $b_n$ mają wspólną granicę $s$, to również $x_n$ dąży do tej liczby.
\[
	x_{n+1} = \frac{1}{2}\left(x_n+\frac{a}{x_n}\right) \spk
	y_{n+1} = \frac{y_n^3 + 3ay_n}{3y_n^2+a} \hfill
	b_{n+1}-a_{n+1} \le_{\text{HA}} \frac{(b_n-a_n)^2}{4a} \spk
	b_{n+1}-a_{n+1} \le_{\text{GA}} \frac{(b_n-a_n)^2}{8a}
\]


% Równanie \prawo{3.1} $ax^2+bxy+cy^2+dx+ey=k$ (współczynniki z $\Z$) dla $|a|+|b|+|c|>0$ ma wyróżnik $\Delta=b^2-4ac$.
% Jeśli $\Delta=0$, to sprowadza się do liniowego lub kwadratowej kongruencji; jeśli nie, to do $AX^2+BY^2=C$ z $A,B,C\in\Z$.
% Jeśli $ABC\neq0$ i znamy rozwiązanie $(x_0,\pm y_0)\in\Q^2$, to pozostałe dostajemy z khm-1 ($\lambda\in\Q$); brak rozwiązań w $\Q\Lra$ w $\R$ lub ,,któregoś spośród: $a \mid x^2-bc$, $b \mid x^2-ca$, $c \mid x^2-ab$''.
% \wcht{Euklides}: wszystkie naturalne rozwiązania $x^2+y^2=z^2$ z $(x,y)=1$, $2\mid y$, są dane przez $(m^2-n^2,2mn,m^2+n^2)$ dla względnie pierwszych $m>n$.
% \wcht{Pell}: $x^2-dy^2=1$, ciekawe dla niekwadratu $d>0$.
% \wcht{Lagrange}: jeśli $(x_1,y_1)$ jest rozwiązaniem z $y_1>0$ i minimalnym $x_1>0$, to pozostałe spełniają $x_n+y_nd^{1/2}=(x_1+y_1d^{1/2})^n$; jest ich $\infty$-wiele oraz $x_{n+1}=x_1x_n+dy_1y_n$, $y_{n+1}=x_1y_n+y_1x_n$, $x_{n+1}=2x_1x_n-x_{n-1}$, $y_{n+1}=2x_1y_n-y_{n-1}$.
% \wcht{Lemat Dirichleta}: każdej $a\in\R$ i $N\in\N$ odpowiada $m/n\in\Q$ (skrócona), że $1\le n\le N$ i $|a-m/n| \le 1/(nN)$.
% \[
% 	\left(x_0-2\cdot\frac{Ax_0+\lambda By_0}{A+B\lambda^2},y_0-2\lambda\cdot\frac{Ax_0+\lambda By_0}{A+B\lambda^2}\right)
% \]
% {\color{Red} Stosunkowo szybką metodę testowania... strona 88.}

% %Równanie \prawo{3.2} $x^n+y^n = z^n$ nie ma rozwiązań dla $n \ge 3$ (Wiles, \datum{1995}). \\
% %Krzywe eliptyczne.
% {
% Nam \prawo{3.2} dui \color{gray} ligula, fringilla a, euismod sodales, sollicitudin vel, wisi. Morbi auctor lorem non justo. Nam lacus libero, pretium at, lobortis
% vitae,ultricieset,tellus. Donecaliquet,tortorsedaccumsanbibendum,eratligulaaliquetmagna,vitaeornareodiometusami. Morbiacorci
% et nisl hendrerit mollis. Suspendisse ut massa. Cras nec ante. Pellentesque a nulla. Cum sociis natoque penatibus et magnis dis parturient
% montes, nascetur ridiculus mus. Aliquam tincidunt urna. Nulla ullamcorper vestibulum turpis. Pellentesque cursus luctus mauris.
% }

Zbiór \wcht{funkcji arytmetycznych} \prawo{4.1} ($\N \to \C$) %($\varphi$, $\omega$, $\sigma_k$, $\Omega$, $\mu$).
z dodawaniem i \wcht{splotem Dirichleta} $f*g \colon n \mapsto \sum_{d \mid n} f(d) g(n : d)$ jest przemiennym pierścieniem bez dzielników zera, $n \mapsto [n=1]$ jest jedynką.
Funkcja $f$ odwraca się $\Lra f(1) \neq 0$.
Zbieżność absolutna khm-1-a dla $f, g \in \mathbb A$ oraz pewnego $z \in \C$ pociąga to samo dla khm-1-b.
\wcht{Splot Abela}: $f \times g \colon n \mapsto \sum_{k=0}^n f(k)g(n-k)$, określony na $\mathbb A_0$ ($\N_0 \to \C$).
Ma jedynkę $n \mapsto [n = 0]$; $f$ się odwraca $\Lra f(0) \neq 0$.
\wcht{Splot unitarny}: $f \circ g \colon n \mapsto \sum_* f(d) g(n/d)$, sumowanie po $d \mid n$, że $(d, n/d) = 1$.
\wcht{Sumowanie Abela}: gdy $a_i, b_i \in \C$, $A(m) = a_1 + \ldots + a_m$ i $c_m = b_{m+1} - b_m$, to $\sum_{i=1}^n a_i b_i = A(n) b_n - \sum_{m=1}^{n-1} A(m) c_m$.
\wcht{Wzór Eulera-MacLaurina}: \emph{Analiza 1}.
Dla $x \ge 2$ i $c > -1$, $\{f(n)\}$ jest skrótem $\sum_{n \le x} f(n)$ (tylko tu!): $\{n^c\} = x^{1+c}/(1+c) + O(x^c)$, $\{1/n\} = \log x + \gamma + O(1/x)$, wzór Stirlinga (\emph{Kombinatoryka}).
\wcht{Tw. Césaro}: jeśli $g = 1 * f$ i szeregi $\sum_{n=1}^\infty g(n) x^n$, $\sum_{n=1}^\infty f(n) x^n / (1-x)$ są absolutnie zbieżne w $x$, to mają równe sumy.
Pierścień arytmetycznych funkcji ze splotem Cauchy'ego jest izo- z ,,formalnym'' $\C[X]$, ze splotem Dirichleta: $\C[X_1, X_2, \ldots]$.
\[
	\left[\sum_{n=1}^\infty \frac{f(n)}{n^z} \right] \cdot \left[\sum_{n=1}^\infty \frac{g(n)}{n^z} \right] = \sum_{n=1}^\infty \frac{(f*g)(n)}{n^z} \spk
	\left\{\frac{1}{n \log n}\right\} = \log \log x + C_1 + O \left( \frac{1}{x \log x}\right) \spk
	\left\{\frac{\log n}{n}\right\} = \frac{\log^2 x}{2} + O \left(\frac{\log x}{x}\right)
\]

Funkcja \wcht{addytywna} \prawo{4.2} spełnia $f(mn) = f(m) + f(n)$ dla $(m,n) = 1$, \wcht{w pełni}: dla wszystkich; \wcht{multiplikatywna}: $f(mn) = f(m) f(n)$; te ostatnie ze splotem Dirichleta są grupą.
Wniosek Bella: splot multi- z niemulti- nie jest multi-.
Przykłady: $d$, $\sigma$, $\sigma_k$.
\wcht{Wzór Möbiusa}: $f$, $g$ są arytmetyczne, $\sum_{d \mid n} f(d) = g(n) \Lra \sum_{d\mid n} \mu (d) g(n/d) = f(n)$ (Dedekind, Liouville \datum{1857}).
Khm-3: Walfisz (\datum{1963}).
Jeśli funkcja $f$ spełnia ,,$\lim_n f(2n+1) - f(n)$ istnieje'' lub ,,$f(n+1) - f(n) \to 0$, $f$ addytywna'', to $f(n) = c \log n$ (Mauclaire, Erdös).

Dla każdego $\delta > 0$ prawdą jest $d(n) = o(n^\delta)$.
Dla $x \ge 2$, $\sum_{n \le x} d(n) = x \log x + (2\gamma - 1) x + O(x^{1:2})$.
Jeżeli funkcja $f$ jest multiplikatywna, $S = \sum_{n \ge 1} f8n)$ zbiega absolutnie, to $\prod_p \sum_{k \ge 0} f(p^k)$ też, do tego samego.
\wcht{Tw. von Sternecka}: $(1 * (f \circ g)) = (1 * f)(1 * g)$, przy czym $(f \circ g)(n) = \sum f(r)g(s)$, $[r, s] = n$.
\[
	\limsup_n \frac{\varphi(n)}{n} = 1 \spk
	\liminf_n \frac{\varphi(n)}{n} = 0 \spk
	\sum_{n \le N} \varphi(n) = 3 \frac{N^2}{\pi^2} + O(N [\log N]^{2:3} [\log \log N]^{4:3}) \spk
	\sum_{n=1}^\infty \frac{\mu(n)}{n} = 0
\]


% % \wcht{Gęstość górna, dolna}: \prawo{4.3} $d^* = \limsup_{x \to \infty} |A \cap [1,x]| / x$ ($d_* = \liminf \dots$).
% % Gęstość unii rozłącznej dwóch jest sumą gęstości, ale nie jest miarą.
% % Jeśli $a_k$ jest ciągiem naturalnych parami względnie pierwszych, zaś $A$ zbiorem naturalnych niepodzielnych przez żadną z $a_k$, to $A$ ma gęstość: gdy $\sum_{i=1}^\infty 1/a_i$ rozbiega, to $d(A) = 0$, jesli nie, to $d(A) = \prod_{i=1}^\infty (1-1/a_i)$.
% % Wniosek: $\mathbb P$ ma gęstość zero.


% % \wcht{Górna/dolna wartość średnia}: $M^*(f) = \limsup_{x \to \infty} \sum_{n \le x} f(n) /x$ (inferior).
% % \wcht{Tw. van der Corputa}: jeśli $g$ ma wartość średnią, zaś $H = \sum_{n=1}^\infty h(n) / n$ abso-zbiega, to $g*h$ też ma wartość średnią, $M(g) \cdot H$.
% % \wcht{Aproksymanta} $g$ dla $f$: dla każdego $\varepsilon > 0$ $\{n \colon |f(n) - g(n)| > \varepsilon g(n)\}$ jest gęstości zero.
% % \wcht{Tw. Bircha}: jeśli $f$ jest multiplikatywna, dodatnia i nieograniczona, ma aproksymantę niemalejącą $\Ra$ $f(n) = n^c$.
% % \wcht{Wniosek Erdösa}: jeśli $f$ jest multiplikatywna, dodatnia i monotoniczna, to $f(n) = n^c$; jeśli jest addytywna i monotoniczna, to $g(n) = c \log n$.

% \wcht{Charakter} \prawo{4.4} grupy $G$ (skończonej, abelowej): homo- $G \to \{z \in \C^\times : |z| = 1\}$, tworzą \wcht{grupą dualną}, $\widehat G$.
% Jeżeli $g$ generuje $\Z_n$, to $\widehat G \cong G$, zaś jej elementy to $\varphi_j(g^r) = \exp(2 \pi i r j / n)$ dla $0 \le j, r \le n-1$.
% Gdy $H = G_1 \oplus G_2$, to $\widehat H \cong \widehat G_1 \oplus \widehat G_2$.
% Jeśli $g \in \Z_n$, dla każdego $\phi \in \widehat \Z_n$ jest $\phi(g) = 1$, to $g$ jest jednością, zaś $|\widehat G| = \varphi(n)$.
% \wcht{Charakter Dirichleta}: $\chi(n) = \psi(n \textrm{ mod } k)$, gdy $(n, k) = 1$ i $0$ w.p.p. ($\psi$ to charakter $\Z/k\Z$). 
% Własności: $\chi(n+k) = \chi(n)$, $\chi(n) = 0 \Lra (n, k) \neq 1$ oraz $\chi(mn) = \chi(m)\chi(n)$.
% Funkcja arytmetyczna z tymi własnościami jest charakterem mod $k$.
% Charakter pochodzący od $\psi = 1$: $\chi_0$, \wcht{główny}.
% Symbol Legendre'a, Jacobiego ($(\frac{x}{k}) = \prod_{j=1}^r (\frac{x}{p_j})^{\alpha_j}$ dla $x \in \N$, gdy $2 \nmid k = \prod_{j=1}^r p_i^{\alpha_i}$).

% Jeśli $m \mid n$ i $\chi$ jest charakterem mod $m$, to wzór $\chi_1(x) = \chi(x \textrm{ mod } m)$ (gdy $(x, n) = 1$) lub $0$ (w.p.p) określa charakter mod $n$.
% Mówi się, że $\chi_1$ jest indukowany przez $\chi$.
% \wcht{Przewodnik} charakteru $\chi$ mod $k$: minimalne $d$, że istnieje charakter $\chi_1$ mod $d$ indukujący $\chi$.
% \wcht{Pierwotny} charakter: $d = k$.
% Charakter $\chi$ mod $k$ jest pierwotny $\Lra$ dla każdego $d \mid k$, $d < k$, istnieje $n \in 1 + d\Z$, że $(n, k) = 1$ i $\chi(n) \neq 1$.

% Jeśli $a \in \Z$, $\zeta_k$ to pierwotny $k$-ty pierwiastek jedności, to $\tau_a (\chi) = \sum_{n=1}^k \chi(n) \zeta_k^{an}$ (gdzie $\chi$ jest charakterem mod $k$) jest \wcht{sumą Gaußa}.
% \wcht{Tw. Polyi-Winogradowa}: jeśli $k \ge 3$, zaś $\chi$ pierwotnym charakterem modulo $k$, to dla $x \ge 1$ jest: $\left|\sum_{n \le x} \chi (n) \right| \le \sqrt{k} \log k$.

% \newpage








% Wniosek: jeśli $p > 2$ jest pierwsza, zaś $n_2(p)$ najmniejszą dodatnią nieresztą kwadratową modulo $p$, to $n_2(p) \le 1 + \sqrt{p} \log p$.

\wcht{Forma Pfaffa} \prawo{5.1} (1-forma) odwzorowanie $\omega \colon (U \subseteq_o \R^n) \to \textrm{L}(\R^n, \C)$; \wcht{rzeczywista}: zawsze $\omega(x)[\R^n] \subseteq \R$.
Dyferencjały różalnych $f \colon U \to \C$, $\sum_{i=1}^n \partial_i f(x)h_i = \D f (x) h$, to przykłady 1-form.
Rzeczywiste 1-formy odpowiadają polom wektorów, polu $v \colon U \to \R^n$ przypisujemy formę $\omega_v$, $\omega_v(x) h = \langle v(x) \mid h \rangle$.
Każda forma liniowa $\omega(x)$ ,,pochodzi'' od $v_\omega(x)$: $\omega(x)h = \langle v_\omega(x) \mid h \rangle$, $x \mapsto v_w(x)$ jest szukanym polem.
Dyferencjał dostanie gradient: $v = \grad f$ gdy $\omega_v = \textrm{d}f$.
Niech $a_i(\xi) = \omega(\xi) e_i$, wtedy $\omega(\xi) h = \sum_{i=1}^n a_i(\xi) \textrm{d}x_i (\xi)h$, w skrócie $\omega = \sum_i a_i \textrm{d}x_i$.
Funkcje $a_i$ to \wcht{współczynniki} względem $\D x_i$.

Forma \prawo{5.2} $\omega$ na $U$ jest \wcht{całkowalna wzdłuż $\gamma$}, gdy istnieje $I$, że każdy $\varepsilon > 0$ ma $\delta > 0$, że rozkład $Z$ odcinka $[a, b]$ drobniejszy niż $\delta$ pociąga dla każdego $Z'$ nierówność $|S(Z, Z') - I| < \varepsilon$; rozkład $Z$: $a = t_0 < \dots < t_r = b$; wtedy $Z' = \{t_k' \in [t_{k-1}, t_k] : 1 \le k \le r\}$.
Nie można całkować wzdłuż każdej drogi, $\gamma = (\gamma_1, \ldots, \gamma_n) \colon [a, b] \to \R^n$ jest \wcht{całkodrogą}, gdy istnieją ,,Regel-'' $\dot \gamma_i$, których $\gamma_i$ to pierwotne.
Wzdłuż całkodrogi $\gamma$ każda ciągła 1-forma $\omega = \sum_{i=1}^n a_i \D x_i$, $\omega$ jest całkowalna i khm-3.
\[
	S(Z, Z') = \sum_{k=1}^r \omega(\gamma(t_k'))(\gamma(t_k) - \gamma(t_{k-1})) \spk
	I = \int_\gamma \omega \spk
	\int_\gamma \omega = \int_a^b \omega(\gamma(t)) \dot \gamma (t) \,\D t = \int_a^b \sum_{i=1}^n a_i(\gamma(t)) \cdot \dot{\gamma}_i (t) \,\D t
\]

Forma \prawo{5.3} Pfaffa $\omega = \sum_{i = 1}^n f_i \D x_i$, która ma na $U \subseteq_o \R^n$ \wcht{potencjał} (\wcht{f. pierwotną}: różalną $f \colon U \to \C$, że $\omega = \D f$, tzn. $f_k = \partial_k f$) jest \wcht{dokładna}.
Jeśli $f$ to potencjał na $U$ ciągłej 1-formy $\omega$, to całka z $\omega$ wzdłuż każdej całkodrogi $\gamma$ w $U$ od $A$ do $B$ wynosi $f(B) - f(A)$.
Ciągła 1-forma $\omega$ na obszarze $U \subseteq_o \R^n$, którą można całkować niezależnie od drogi, ma pierwotną: $f(x) = \int_a^x \omega$ dla $x \in U$.

Ciągła \prawo{5.4} 1-forma $\omega$ na $U \subseteq_o \R^n$ jest \wcht{lokalnie dokładna} (\wcht{zamknięta}): każdy $x$ ma $U_x$, gdzie istnieje pierwotna $f$ dla $\omega$: $\omega \mid_{U_x} = \D f$.
Dla formy $\omega = \sum_{i=1}^n f_i \D x_i$ (ciągle różalnej) tw. Schwarza daje \wcht{warunek całkowalności} (konieczny dla zamkniętości): $\partial_i f_k = \partial_k f_i$ dla $i, k \le n$.
\wcht{Gwiezdny zbiór}: $X \subseteq \R^n$, o ile istnieje $a \in X$, że wszystkie odcinki $[a,x]$ leżą w $X$.
\wcht{Lemat Poincarégo}: ciągle różalna 1-forma na gwiezdnym zbiorze spełniająca warunek całkowalności
 ma pierwotną.
Niech $n = 3$.
Warunek całkowalności zmienia się w \wcht{bezrotacyjność}: $\rot v = (\partial_2 v_3 - \partial_3 v_2,$ $\partial_3 v_1 - \partial_1 v_3, \partial_1 v_2 - \partial_2 v_1)$, symbolicznie $\nabla \times u$, ma się zerować (dla różalnych pól).
Na gwiezdnych zbiorach jest też wystarczający.

\wcht{Homotopia} \prawo{5.5} dwóch krzywych $\gamma_i \colon [a,b] \to X$ o wspólnych końcach $A, B$: ciągłe $H \colon [a,b] \times [0,1] \to X$, że $H(\cdot, i) = \gamma_i$ dla $i \in \{0, 1\}$ oraz $H(a, s) = A$, $H(b,s) = B$.
Całka z lokalnie dokładnej 1-formy $\omega$ w $U \subseteq_o \R^n$ po homotopijnych drogach o tych samych końcach ma tę samą wartość.
\wcht{Wolna homotopia} zamkniętych krzywych: nie musi trzymać końców.
Indeks zaczepienia nie zależy wolnohomotopijnie od krzywej.
%\wcht{Tw. Brouwera o punkcie stałym}: każda ciągła z $B_\ge(0,1)$ w siebie ma punkt stały.
\wcht{Jednospójność}: łukowo spójny $X \subseteq \R^n$, gdzie każda zamknięta krzywa ściąŋa się do punktu.
Zamknięta 1-forma na jednospójnym obszarze $U \subseteq_o \R^n$ ma tam całki niezależne od drogi oraz pierwotną.
Dla $n = 3$: ciągle różalne pole wektorów bez rotacji jest gradientowe.

W \prawo{6.2} tym rozdziale żyjemy w $(\Omega, \mathcal M, \pstwo)$, wszystkie $\sigma$-ciaua zawierają się w $M$.
Mamy \wcht{warunkowa wartość oczekiwaną} (khm-1) dla $\pstwo_A(B)$ równego $\pstwo (B \mid A)$ i $X$ o skończonej nadziei.
Jeżeli $\pstwo(A) > 0$, to khm-2.
Jeżeli $\{A_i\}$ to przeliczalne rozbicie $\Omega$ i $\pstwo(A_i) > 0$, zaś z-losowa $X$ jest caukowalna, to khm-3.
Jeśli $\Omega = \bigcup_i B_i$, $\pstwo(B_i) > 0$ i $\mathcal G = \sigma(B_i : i \in I)$, to $\expected (X \mid \mathcal G)(\omega) = \sum_{i \in I} \expected (X \mid B_i) [\omega \in B_i]$. 
Tak zefiniowana z-losowa jest $\mathcal G$-mierzalna; dla $B \in \mathcal G$ mamy khm-4.
\[
	\expected (X \mid A) := \int_\Omega X \,\D \pstwo_A = \int_A \frac{X}{\pstwo(A)} \,\D \pstwo \spk
	\expected X = \sum_{i=0}^\infty \expected (X \mid A_i) \cdot \pstwo (A_i) \spk
	\int_B X \,\D\pstwo = \int_B \expected (X \mid \mathcal G) \,\D\pstwo
\]

\wcht{Warunkowa wartość oczekiwana} \prawo{6.3} całkowalnej z-losowej $X$ pod warunkiem $\sigma$-ciaua $\mathcal F \subseteq \mathcal M$ to $\mathcal F$-mierzalna z-losowa $\expected (X \mid \mathcal F)$, że dla $A \in \mathcal F$ całki z ,,$X \, \D \pstwo$'', ,,$\expected (X \mid \mathcal F) \, \D \pstwo$'' nad $A$ są równe.
Zawsze istnieje, jednoznacznie z dokładnością do zdarzeń o p-stwie zero.
\wcht{Wielkie twierdzenie}: z-losowe $X, X_i$ mają skończoną nadzieję, $\mathcal G \subseteq \mathcal F \subseteq \mathcal M$ to $\sigma$-ciaua.
\wcht{Nierówność Jensena}: dla wypukłej $\varphi \colon \R \to \R$, z-losowych $X$, $\varphi(X)$ z $L^1(\Omega, \mathcal M, \pstwo)$ i $\sigma$-ciaua $\mathcal F \subseteq \mathcal M$ mamy $\varphi (\expected(X \mid \mathcal F)) \le \expected (\varphi(X) \mid \mathcal F)$ p.n.
Jeżeli z-losowa $X$ spełnia $\expected |X| < \infty$, zaś $Y$ ma wartości w $\R^n$, to istnieje borelowska $h \colon \R^n \to \R$, że $\expected (X \mid Y) = h(Y)$.
Warwaroczem z-losowej $X$ pod warunkiem $\{Y = y\}$ nazywamy $h(y)$.

\wcht{4.A}: dla $\mathcal F$-mierzalnej $X$: $\expected (X\mid \mathcal F) = X$ p.n.
\wcht{4.B}: dla $X \ge 0$: $\expected (X \mid \mathcal F) \ge 0$ p.n.
\wcht{4.C}: $|\expected(X \mid \mathcal F)| \le \expected (|X| \mid \mathcal F)$ p.n.
\wcht{4.D}: $\expected (\alpha X_1 + \beta X_2 \mid \mathcal F)$ jest równe $\alpha \cdot \expected (X_1 \mid \mathcal F) + \beta \cdot \expected (X_2 \mid \mathcal F)$ p.n.
\wcht{4.E}: $X_n \uparrow X$ implikuje $\expected (X_n \mid \mathcal F) \uparrow \expected(X \mid \mathcal F)$ p.n. 
\wcht{4.G}: $\expected X = \expected( \expected (X \mid \mathcal F))$ p.n.
\wcht{4.F}: $\expected (X \mid \mathcal G)$, $\expected (\expected (X \mid  \mathcal F) \mid  \mathcal G)$ oraz $\expected (\expected (X \mid  \mathcal G) \mid \mathcal F)$ są równe sobie p.n.
\wcht{4.H}: dla niezależnych $\mathcal F$ i $\sigma(X)$: $\expected(X \mid \mathcal F) = \expected X$ p.n.
\wcht{4.I}: dla ograniczonej oraz $\mathcal F$-mierzalnej z-losowej $Y$, $\expected (XY \mid \mathcal F) = Y \expected (X \mid \mathcal F)$.

\wcht{Warunkowy Fatou}: dla $X_n \ge 0$ mamy $\expected (\liminf X_n \mid \mathcal F) \le \liminf \expected (X_n \mid \mathcal F)$.
\wcht{Levi}: gdy $|X_n (\omega)| \le Y (\omega)$, $\expected Y < \infty$ oraz $X_n \to X$ p.n., to $\lim_n \expected (X_n \mid \mathcal F) = \expected (X \mid \mathcal F)$ p.n.
\wcht{Wariancja}: $\variance (X \mid \mathcal F) := \expected ((X - \expected (X \mid \mathcal F))^2 \mid \mathcal F)$, gdy $\expected X^2 < \infty$, wtedy $\variance X = \expected \variance (X \mid \mathcal F) + \variance \expected (X \mid \mathcal F)$.
\wcht{Fubini}: $\sigma$-ciauo $\mathcal F \subseteq \mathcal M$, p. mierzalna $(E, \Sigma, \mu)$, $X \in L^1 (E \times \Omega, \Sigma \times \mathcal F, \mu \times \pstwo)$.
Wtedy khm-1, khm-2.
\wcht{Niezależność} $\sigma$-ciau $\mathcal F_1, \ldots, \mathcal F_n; \mathcal G \subseteq M$: $\pstwo (\bigcap_{i=1}^n A_i \mid \mathcal G) = \prod_{i=1}^n \pstwo (A_i \mid \mathcal G)$ dla każdego $A_i \in \mathcal F_i$.
$\mathcal F, \mathcal H$ są wnz względem $\mathcal G$ $\Lra$ dla każdego $H \in \mathcal H$, $\pstwo (H \mid \mathcal F \vee \mathcal G) = \pstwo (H \mid \mathcal G)$ p.n.
\[
	\expected \left | \int_E \expected (X_s \mid \mathcal F) \mu(\D s) \right| < \infty \spk
	\expected \left[\left. \int_E X_s \mu(\D s) \right\mid \mathcal F \right] = \int_E \expected(X_S \mid \mathcal F) \mu (\D s)
\]

\wcht{P-stwo warunkowe} \prawo{6.4} $A \in \mathcal M$ pod warunkiem $Y = y$: $\pstwo(A \mid Y = y) := \expected (1_A \mid Y= y)$. 
Gdy $(X,Y)$ ma ciągły rozkład o gęstości $g$, to khm-1 i khm-2 dla tych borelowskich $\varphi$, że $\expected |\varphi(x)| < \infty$ (gdy mianownik się zeruje, kładziemy $0$ po prawej).
\wcht{Uogólniony Bayes} $\mathcal G \subseteq \mathcal F$: $\sigma$-ciało, $B \in \mathcal G$, $A \in \mathcal F$, $\pstwo(A) > 0$ i ,,$\pstwo(A\mid \mathcal G) = \expected(\mathbb I_A \mid \mathcal G)$ dają khm-3.
\wcht{Abstrakcyjny}: $P, Q$ miarami probabilistycznymi na $(\Omega, \mathcal F)$, że gęstość $\D Q / \D P = Z >0$ istnieje, $\mathcal G \subseteq \mathcal F$, $X$: z-losowa $Q$-caukowalna; wtedy: $\expected_Q X = \expected_P XZ$ i khm-4 jest równe $\expected_Q (X \mid \mathcal G)$.
\[
	\pstwo (X\in B \mid Y) = \frac{\int_B g(x, Y) \, \D x}{\int_\R g(x,Y) \, \D x} \spk
	\expected (\varphi(x) \mid Y) = \frac{\int_\R \varphi(x) g(x, Y) \, \D x}{\int_\R g(x,Y) \, \D x} \spk
	\pstwo (B \mid A) = \frac{\int_B \pstwo (A \mid \mathcal G) \, \D \pstwo}{\int_\Omega \pstwo(A \mid \mathcal G) \, \D \pstwo} \spk
	\frac{\expected_P(XZ \mid \mathcal G)}{\expected_P (Z \mathcal G)}
 \]

\wcht{P-stwo} $B$ \prawo{6.5} pod warunkiem $\sigma$-ciaua $\mathcal F$: $\mathcal F$-mierzalna z-losowa $\pstwo (B \mid \mathcal F) := \expected (1_B \mid \mathcal F)$ o wartościach w $[0,1]$.
Khm-1 ($A \in \mathcal F$); jeśli $B_n$ są rozłączne parami, to khm-2.
\wcht{Regularny rozkład warunkowy} względem $\mathcal F$: funkcja $\pstwo_{\mathcal F} \colon \mathcal M \times \Omega \to [0,1]$, że $\pstwo_{\mathcal F} (B, \cdot)$ jest wersją $\expected(1_B, \mathcal F)$, zaś $\pstwo_{\mathcal F}(\cdot, \omega)$ to rozkłady p-stwa na $\mathcal M$.
Dla całkowalnej z-losowej $X$: khm-3. 
,,Regrowar'' nie musi istnieć, lecz istnieje on dla z-losowej $X$ pod warunkiem $\sigma$-ciaua $\mathcal F$ (funkcja $\pstwo_{X \mid \mathcal F}$ na $\mathfrak B(\R) \times \Omega$, że $\pstwo_{X \mid \mathcal F}(B, \omega) = \pstwo (X \in B \mid \mathcal F)(\omega)$, zaś $\pstwo_{X \mid \mathcal F}(\cdot, \omega)$ to rozkłady p-stwa na $\mathfrak B(\R)$).
\[
	\pstwo (A \cap B) = \int_A \pstwo (B \mid \mathcal F) \, \D \pstwo \spk
	\pstwo \left(\left. \bigcup_{n=1}^\infty B_n \right \mid \mathcal F \right) = \sum_{n=1}^\infty \pstwo (B_n \mid \mathcal F) \textrm{ p.n.} \spk
	\expected (X \mid \mathcal F)(\omega) = \int_{\Omega} X (\tilde{\omega}) \pstwo_{\mathcal F} (\D{} \tilde{\omega}, \omega) \textrm{ p.n.}
\]

\fancyfoot[LF]{strona \thepage { }z \pageref{LastPage} [od 213 do 355]}

\wcht{Rozbicie} zbioru $A \subseteq \R^n$ o subtelności $\delta$: rodzina rozłącznych i mierzalnych $A_1, \dots, A_r$, że $\bigcup_k A_k = A$ i ich średnica jest $\le \delta$.
% 7-8
Fakt: dla ciągłej $f \colon A \to \C$ na zwartym $A \subset \R^n$ i każdego $\varepsilon > 0$ istnieje $\delta > 0$, że dla rozbić $A_1, \dots, A_r$ zbioru $A$ o subtelności $\delta$ i dowolnego wyboru $\xi_k \in A_k$ zachodzi khm.
\[
	\left| \int_A f(x) \,\D x - \sum_{k=1}^r f(\xi_k) \cdot v(A_k) \right| \le \varepsilon
\]

\fancyfoot[LF]{strona \thepage { }z \pageref{LastPage} [od 356 do 417]}

\newpage

\wcht{Równanie/problem Sturma-Liouville'a}: \prawo{SL} znamy $p, q$ oraz $w$, dla jakich $\lambda$ istnieją nietrywialne rozwiązania $y$?
Takiej postaci są np. Bessela, Legendre'a.
\[
	\frac{d}{dx} \left[ p(x) \frac{dy}{dx}\right] + [\lambda w(x) - q(x)] y = 0
\]

% https://en.wikipedia.org/wiki/Sturm%E2%80%93Liouville_theory

Mamy \prawo{GF} $\dot x = A(t)x$, macierz $n \times n$ z okresem $T$.
Jeśli $\Phi(t)$ jest macierzą fundamentalną, to $\Phi(t+T) = \Phi(t) \Phi^{-1}(0)\Phi(T)$.
Dalej, jeśli $\exp(TB) = \Phi^{-1}(0)\Phi(T)$ (to \wcht{macierz monodromii}), to istnieje $T$-okresowa macierz $P(t)$ ($n \times n$), że $\Phi(t) = P(t) \exp(tB)$: tutaj $B$ może być zespolona.
Do tego jest rzeczywista macierz $R$ i rzeczywista $2T$-okresowa $Q(t)$, że $\Phi(t) = Q(t)\exp(tR)$ (to było \wcht{tw. Floqueta}, \datum{1883}).

% http://mathworld.wolfram.com/FloquetsTheorem.html

Jeśli \prawo{PSL} $f(t) \colon [0, \infty)\to \R$ jest przedziałami ciągła i spełnia $|f(t)| \le Me^{ct}$, to dla $s > c$ ma \wcht{transformatę Laplace'a}.
Różne funkcje mają różne transformaty (\wcht{tw. Lercha}).
%AWP $a y'' + by' + cy = f(t)$, $y(0) = y_0$ i $y'(0) = y_0'$ upraszcza się przez nałożenie transformaty na obie strony, uporządkowanie i użycie odwrotnej.
Fakt: $\mathcal L\{(f*g)(t)\} = \dots = \mathcal L\{f(t)\} \mathcal L\{g(t)\}$. 
\wcht{Całka Bromwicha}: khm-4, przy czym pionowy kontur $\gamma$ nie dotyka osobliwości $F(s)$.
Dalej, $\mathcal L\{e^{at}f(t)\} (s) = F(s-a)$ oraz $\mathcal L\{-t f(t)\}(s) = F'(s)$.
\[
	\mathcal L\{f(t)\} = \int_0^\infty \frac{f(t)}{e^{st}} \,\textrm{d}t \spk
	\mathcal L\left\{\int_0^t f(\tau) g(t-\tau) \textrm{d}\tau\right\} \spk
	\mathcal L[f^{(n)}(t)] = s^n \mathcal L[f(t)] - \sum_{k=0}^{n-1} s^kf^{(n-k-1)}(0) \spk
	f(t) = \int_{\gamma-i \infty}^{\gamma + i \infty} \frac{e^{st} F(s)}{2\pi i}\,\textrm{d}s
\]

Tablice transformat Laplace'a:
\begin{align*}
\exp(at) & \mapsto 1/(s-a) &
t^{n} & \mapsto {n!}/{s^{n+1}} \\
%\sqrt{t} & \mapsto \sqrt{\pi/4s^3} \\
\sin at & \mapsto {a} / (s^2+a^2) &
\cos at & \mapsto {s} / (s^2+a^2) \\
\sinh at & \mapsto a/ (s^2-a^2) &
\cosh at & \mapsto s / (s^2-a^2) \\ 
t \sin at & \mapsto {2as}/(s^2+a^2)^2 &
t \cos at & \mapsto (s^2-a^2)/(s^2+a^2)^2 \\
e^{at} [\sin bt] & \mapsto {b}/{[(s-a)^2 + b^2]} &
e^{at} [\cos bt] & \mapsto {(s-a)}/{[(s-a)^2 + b^2]} \\
e^{at} [\sinh  bt] & \mapsto {b}/[(s-a)^2 - b^2] &
e^{at} [\cosh bt] & \mapsto {(s-a)}/[(s-a)^2 - b^2]
\end{align*}

Dane \prawo{??} jest zwyczajne $p(x) y''(x) + q(x) y'(x) + r(x) y(x) = 0$. % czyżby Braun?
Rozwiązań szukamy szeregami $y(x) = \sum_{k=0}^\infty a_k(x-x_0)^k$.
Jeśli $p(x_0)\neq 0$, to $x_0$ jest \wcht{ordynarny}, jeśli nie, to \wcht{singularny}.
Singularny jest \wcht{regularny}, jeśli $\lim_{x \to x_0} (x-x_0)q(x) / p(x)$ oraz $\lim_{x \to x_0} (x-x_0)^2 r(x) / p(x)$ są skończone.
\wcht{Metoda Frobeniusa}: jeśli $x = 0$ jest regularny singularny, to przynajmniej jedno rozwiązanie równania jest postaci $x^r \sum_{k=0}^\infty a_k x^k$.
R--e indeksowe ma pierwiastki $r_1, r_2$.
$y_1 = x^{r_1} \sum_{k=0}^\infty a_k x^k$.
Jeśli $r_1 = r_2$, to $y_2 = x^r \sum_{k=0}^\infty b_kx^k y_1 \ln x$.
Jeśli $r_1 - r_2 \in \Z$, to rozwiązaniami są $y_1 = x^{r_1} \sum_{k=0}^\infty a_kx^k$ oraz $y_2 = x^{r_2} \sum_{k=0}^\infty b_k x^k + C y_1 \ln x$.
Wreszcie dla zespolonych: wziąć $\Re$, $\Im$ z $x^{r_1} \sum_{k=0}^\infty a_k x^k$.

\emph{Uzmiennianie} \prawo{m} $a_n y^{(n)} + \ldots + a_1 y' + a_0 y = f(x)$.
Jeśli $C_1 y_1 + \ldots + C_n y_n$ rozwiązuje jednorodne, to traktujemy $C_i$ jako funkcji $C_i(x)$, które wyznacza się z gargantuicznej macierzy (Wrońskiego).
\[
\begin{pmatrix} y_1 & \ldots & y_n \\ y_1^\prime & \ldots & y_n^\prime \\ \vdots & \ddots & \vdots \\ y_1^{(n-1)} & \ldots & y_n^{(n-1)} \end{pmatrix} \begin{pmatrix} C_1^\prime \\ C_2^\prime \\ \vdots \\ C_n^\prime \end{pmatrix} = \begin{pmatrix} 0 \\ 0 \\ \vdots \\ \frac{f(x)}{a_n} \end{pmatrix}
\]

Jeśli jedno rozwiązanie dla $\ddot y + P(x) \dot y + Q(x) y = 0$ jest znane, $y_1$, to drugie można znaleźć przez redukcję rzędu.
Mamy bowiem $(dW)/W = -P(x)dx$, gdzie $W = y_1y'_2 - y_1'y_2$ jest wrońskianem.
Scałkowanie daje $\int_a^x dW/W = -\int_a^x P(x') dx'$, czyli $$\ln |W(x) / W(a)| = - \int_a^x P(x') dx'$$.
Wynika stąd, że $W(x) = W(a) \exp |-\int_a^x P(x') dx'$.
Ale $W = y_1^2 d/dx \, (y_2 / y_1)$, więc $$y_2 = y_1(x) W(a) \int_b^x \exp[-\int_a^{x'} P(x'')dx''] / [y_1(x')]^2 \,dx'$$.

Jak rozwiązać $\dot x = Ax + p$ dla wektora $x$?
Jeśli $p=0$, to rozwiązaniem jest $x(t) = \exp(At)$.
Znajdź wartości własne $\lambda_i$ oraz wektory własne $u_i$.
Policz $x_i = \exp(\lambda_i t) u_i$.
Rzeczywiste wektory są rozwiązaniem jednorodnego równania.
Jeśli $A$ jest $2 \times 2$, to rozwiązaniem jest $\Re [x_i]$ oraz $\Im [x_i]$.
Jeśli równanie nie jest jednorodne, to znajdź rozwiązanie szczególne przez $x^*(t) = X(t) \int X^{-1}(t)p(t) dt$, gdzie kolumny $X$ to $x_i$.
Jeśli jest jednorodne, popatrz na rozwiązania postaci $x = \xi \exp(\lambda t)$.
Ogólnie: $x(t) =x^*(t) + \sum_{i=1}^n c_i x_i$

\emph{Uzmiennianie stałej}. \prawo{Br}
Dla operatora $L = D^2 + p(x)D + q(x)$ chcemy rozwiązać $Lu(x) = f(x)$, dla danej $f$.
Niech $u_1$, $u_2$ będą rozwiązaniami jednorodnego.
Podejrzewamy, że $u_g(x)$ załatwi sprawę.
Umówmy się, że $A' u_1 + B' u_2 = 0$, wtedy można odtworzyć $A$, $B$.
Tutaj $W$ jest wrońskianem, $u_1u_2' - u_2u_1'$ (całość nazywa się \wcht{uzmiennianiem stałej}).
\[
	u_g(x)  = A(x) u_1(x) + B(x) u_2(x) \spk
	A(x) = - \int \frac{u_2(x) f(x) }{W}\,\textrm{d}x \spk
	B(x) = \int \frac{u_1(x) f(x) }{W}\,\textrm{d}x \spk
\]
\end{document}

http://eqworld.ipmnet.ru/en/solutions/ode.htm
Prüfer transformation
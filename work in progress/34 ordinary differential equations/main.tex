\documentclass[a4paper, fleqn, 9pt]{extarticle}
\usepackage{etoolbox}
\usepackage{Alegreya}
\usepackage[euler-digits,euler-hat-accent]{eulervm}

\usepackage{amssymb, mathrsfs}
\usepackage[left=1.5cm,top=2.5cm,right=1.5cm,bottom=1.5cm]{geometry}
\usepackage{multicol}
\usepackage{MnSymbol}

\usepackage{fancyhdr}
\makeatletter
\fancypagestyle{mypagestyle}
{\newpage \fancyfoot[C]{} \renewcommand{\footrulewidth}{0pt}}
\makeatother
\pagestyle{mypagestyle}
\headsep 5pt                 %% put this outside
\usepackage{lastpage}

\usepackage[usenames,dvipsnames]{color}
\usepackage{xcolor}

\newcommand{\prawo}[1]{\marginpar{\textbf{\underline{#1}}}}
\newcommand{\wcht}[1]{{\bf {#1}}}
\newcommand{\spk}{\, \bullet \,}
\newcommand{\datum}[1]{{\color{Purple} \bf {#1}}}

% ? - pożyteczne zbiory
\newcommand{\K}{\mathbb{K}}
\newcommand{\C}{\mathbb{C}}
\newcommand{\N}{\mathbb{N}}
\newcommand{\R}{\mathbb{R}}
\newcommand{\Q}{\mathbb{Q}}
\newcommand{\Z}{\mathbb{Z}}
\newcommand{\Qp}{\mathbb{Q}_p}
\newcommand{\Zp}{\mathbb{Z}_p}

% Logika
\newcommand{\Lra}{\Leftrightarrow}
\newcommand{\Ra}{\Rightarrow}

\usepackage{graphicx}
\relpenalty=10000
\binoppenalty=10000
\interlinepenalty=10000

\newenvironment{enumx}{\begin{enumerate} \setlength{\itemsep}{0pt} \setlength{\parskip}{0pt} \setlength{\parsep}{0pt}}{\end{enumerate}}
\newenvironment{itemx}{\begin{itemize} \setlength{\itemsep}{0pt} \setlength{\parskip}{0pt} \setlength{\parsep}{0pt}}{\end{itemize}}


\usepackage[polish]{babel}
\usepackage[utf8]{inputenc}
\usepackage[T1]{fontenc}
\selectlanguage{polish}

\begin{document}
\setlength{\belowdisplayskip}{2pt}
\setlength{\belowdisplayshortskip}{2pt}
\setlength{\abovedisplayskip}{2pt}
\setlength{\abovedisplayshortskip}{2pt}

\renewcommand{\footrulewidth}{0.4pt}
\fancyhead[LE,LO]{Zwyczajne równania różniczkowe (MSC 34)}
\fancyhead[RO,RE]{leon.aragones@gmail.com, listopad 2015}
\fancyfoot[RF]{Aulbach, czerwiec 1997}

% \wcht{Podzielność}, \prawo{1.1} \wcht{kgV} \prawo{1.2} oraz \wcht{ggT}. 
% \wcht{Algorytm Euklid\wcht{Podzielność}, \prawo{1.1} \wcht{kgV} \prawo{1.2} oraz \wcht{ggT}. 
% \wcht{Algorytm Euklidesa}.
% Dzielenie z resztą: $a = bq+r$ (quotient, rest) dla $0 \le r < |b|$.
% Binarne znajdowanie ggT (Silver, Terzian \datum{1962}): jeśli obie liczby są parzyste, to połowimy je i zapamiętujemy czynnik $2$, gdy tylko jedna jest parzysta, to połowimy ją, jeśli żadna: odejmujemy mniejszą od większej. 
% Jeśli $n_1, n_2, \dots$ dzielą $n$, to $(n_1, \dots, n_k)[n/n_1, \dots, n / n_k] = n$.
% Losowe liczby są kopierwsze z p-stwem $6/\pi^2$. 
% \wcht{Równanie diofantyczne}: \prawo{1.3} rozwiązania w $\Z$.
% Jeśli $a_i \in \Z \setminus \{0\}$, to $\sum_{j=1}^n a_j X_j = b$ ma rozwiązanie w $\Z$ $\Lra$ $(a_1, \dots, a_n) \mid b$; wtedy można wybrać takie, że $|X_i| \le |b/H| + (n-1)H/2$ (i $H = \max_i |a_i|$).
% \wcht{Problem Frobeniusa}: jeśli $(a,b)=1$, to wszystkie większe od $ab-a-b$ zapisują się jako $ax+by$ dla $x,y \ge 0$.esa}.
% Dzielenie z resztą: $a = bq+r$ (quotient, rest) dla $0 \le r < |b|$.
% Binarne znajdowanie ggT (Silver, Terzian \datum{1962}): jeśli obie liczby są parzyste, to połowimy je i zapamiętujemy czynnik $2$, gdy tylko jedna jest parzysta, to połowimy ją, jeśli żadna: odejmujemy mniejszą od większej. 
% Jeśli $n_1, n_2, \dots$ dzielą $n$, to $(n_1, \dots, n_k)[n/n_1, \dots, n / n_k] = n$.
% Losowe liczby są kopierwsze z p-stwem $6/\pi^2$. 
% \wcht{Równanie diofantyczne}: \prawo{1.3} rozwiązania w $\Z$.
% Jeśli $a_i \in \Z \setminus \{0\}$, to $\sum_{j=1}^n a_j X_j = b$ ma rozwiązanie w $\Z$ $\Lra$ $(a_1, \dots, a_n) \mid b$; wtedy można wybrać takie, że $|X_i| \le |b/H| + (n-1)H/2$ (i $H = \max_i |a_i|$).
% \wcht{Problem Frobeniusa}: jeśli $(a,b)=1$, to wszystkie większe od $ab-a-b$ zapisują się jako $ax+by$ dla $x,y \ge 0$.

\fancyfoot[LF]{strona \thepage { }z \pageref{LastPage} [od 1 do 212]}

\wcht{Podgrupa}: \prawo{2.1} niepusty podzbiór zamknięty na $(x, y) \mapsto xy^{-1}$ (przykłady: jądra oraz obrazy morfizmów).
\wcht{Cykliczna}: $\langle g \rangle := \{g^n : n \in \Z\}$, jej rząd to \wcht{rząd} $a$.
Uwaga: $H \vee K := \langle H \cup K \rangle$.
Grupę $S_n$ można włożyć w $A_{n+2} \le S_{n+2}$ (permutacji parzystych), ale nie w $A_{n+1}$, jeśli $n \ge 2$.
Dla $n > 2$, grupa $A_n$ jest generowana przez 3-cykle, zaś $S_n$ przez 2-cykle $(1, i)$ (dla $2 \le i \le n$) albo 2-cykl $(1, 2)$ i pełny cykl $(1, 2, \ldots, n)$.

Jeżeli \prawo{2.2} $H \le G$, to \prawo{2.3} relacja $a \simeq b \Lra ab^{-1} \in H$ rozbija $G$ na warstwy, $Hg = \{hg : h \in H\}$, których jest $[G:H]$ (\wcht{indeks}).
\wcht{Hall} (\datum{1935}): podgrupa skończonej ma wspólny układ reprezentantów (dla lewych i prawych warstw).
\wcht{Lagrange} (\datum{1770}?): $[G:H] = |G|/|H|$ dla skończonych $G$; jeśli $K \le H \le G$, to $[G:K] = [G:H][H:K]$.
Grupa $G$ rzędu $n$ jest cykliczna $\Lra$ w $G$ nie ma dwóch cyklicznych podgrup tego samego rzędu $d \mid n$.
Skończona podgrupa $F^*$ ($F$: ciało) jest cykliczna.
Grupa rzędu $p^n$ cykliczna $\Lra$ abelowa, tylko jedna podgrupa rzędu $p$.

Jeśli $\varnothing \neq S,T \subseteq G$, \prawo{2.4} to $ST = \{st : s \in S, t \in T\}$; dla $S, T \le G$ ($G$ skończona) mamy $|ST| \cdot |S \cap T| = |S| \cap |T|$.
\wcht{Podgrupa normalna}: $K \trk G$, gdy $g K = Kg$ dla $g \in G$; wtedy $G/K$ (zbiór warstw) jest grupą.
\wcht{Mann}: jeśli $G$ jest skończona, zaś $S, T \subseteq G$ niepuste, to $G=ST$ lub $|G| \ge |S| + |T|$.
Każdy element skończonego ciała jest sumą dwóch kwadratów.
Wszystkie podgrupy $G$ są normalne, gdy $G$ jest abelowa (ale nie odwrotnie).
Podgrupy indeksu dwa są normalne.
Jeśli $K \le H \le G$ i $K \trk G$, to $K \trk H$.
Jeśli $H \le G$, to $H \trk G \Lra$ dla $x, y \in G$, $xy \in H \Lra yx \in H$.

\wcht{Komutant}: \prawo{2.5} $G' \trk G$, normalna podgrupa generowana przez \wcht{komutatory}, $[a,b] := aba^{-1}b^{-1}$.
Jeśli $H \trk G$, to: $G/H$ abelowa $\Lra$ $G' \le H$.
Komutant składa się z długich komutatorów, elementów postaci $a_1a_2\ldots a_na_1^{-1}a_2^{-1}\ldots  a_n^{-1}$.
Istnieją dwie grupy rzędu 96, w których $G'$ nie składa się z samych komutatorów (w niższych rzędach takie zjawisko nie występuje).

Trzy \prawo{2.6} twierdzenia o izomorfizmach: $G / \ker f \cong \im f$ dla homo- $f \colon G \to H$.
Jeśli $N \trk G$, $T \le G$, to $T/(N \cap T) \cong NT/N$.
Jeśli $K \le K, H \trk G$, to $(G/K) / (H/K) \cong G/H$.
Gdy $A,B,C \le G$, $A \le B$, to $AC \cap B = A(C \cap B)$ (\wcht{prawo Dedekinda}).
Jeśli $A \cap C = B \cap C$ i $AC = BC$, to $A = B$ (\wcht{prawo modularne}).
\wcht{Zassenhaus}: \prawo{2.7} jeśli dla skończonej $G$ istnieje $n > 1$, że $(xy)^n = x^ny^n$, to $G[n] = \{z \in G : z^n = 1\}$ i $G^n = \{x^n : x\in G\}$ są dzielnikami normalnymi $G$ i $|G^n| \cdot |G[n]| = |G|$.
Grupa \wcht{prosta} ma tylko dwa dzielniki normalne; jedyne abelowe to $\Z/p\Z$.
Grupa $H \le G$ jest maksymalnie normalna $\Lra$ $G/H$ jest prosta.
\wcht{Schur}: jeśli homo- $f \colon G \to H$ nie posyła prostej $G$ w $1$, to jest mono-.

%,,\wcht{Czwarte}'': $K \trk G$, $v \colon G \to G/K$ naturalne.
%Funkcja $S \mapsto S/K$ jest bijekcją z rodziny podgrup $S \le G$ zawierających $K$ w rodzinę podgrup $G/K$.
%Gdy $S^* := S/K$, to $T \le S \Lra T^* \le S^* \Ra [S:T] = [S^*: T^*]$.
%Inny fakt:
%$T \trk S \Lra T^* \trk S^* \Ra S/T \cong S^*/T^*$.

\wcht{Produkt prosty} $H \times K$: produkt \prawo{2.8} kartezjański z $(h, k) \cdot (h', k') = (hh', kk')$; abelowy $\Lra$ $H, K$ abelowe.
Jeżeli $H, K \trk G$, $HK = G$ oraz $H \cap K = \{1\}$, to $G \cong H \times K$.
Jeśli $A \trk H$ i $B \trk K$, to $A \times B \trk H \times K$ oraz $(H \times K) / (A\times B) \cong (H / A) \times (K/B)$.
Każdy $N \trk X \times Y$ jest przemienny lub kroi jeden z czynników nietrywialnie.
\wcht{Elementarna abelowa $p$-grupa}: $(\Z/p\Z)^n$.
Abelowa $G$ o pierwszym wykładniku $p$ jest taka; każdy endomorfizm $G$ jest liniowy (bo $G$ jest przestrzenią wektorową nad $\Z_p$).
Wykładnik grupy: nww rzędów jej elementów.


\wcht{Schemat Polya}: \prawo{3.X} p-stwo wyciągnięcia $k$ czarnych w $n$ losach z urny (początkowo: $b$ białych, $c$ czarnych, za każdym razem zwracamy nie $1$, tylko $d + 1$) to $(-c/d \mbox{ nad } k)(-b/d \mbox{ nad } n-k)/(-(c + b)/d \mbox{ nad } n)$.
\wcht{P-stwo} zajścia $A$ \wcht{pod warunkiem} $B$: $\pstwo (A \mid B)$.
\wcht{P-stwo całkowite}: jeśli $\{B_i\}_1^n$ jest \wcht{rozbiciem} $\Omega$ (rozłączna unia $B_i$ to $\Omega$) i $\pstwo (B_i) > 0$, to $\pstwo (A) = \sum_{k=1}^n \pstwo (A \mid B_k) \pstwo (B_k)$; po wyrafinowaniu: khm-2 ($H_i$ to rozbicie $\Omega$, $\pstwo(H_i) > 0$).
\wcht{Wzór Bayesa}: $\{H_i\}_1^\infty$ to rozbicie $\Omega$, $\pstwo (H_i), \pstwo (A) > 0$ pociąga khm-3.
\hfill $* = \{k : \pstwo (B\cap H_i) > 0\}$
\[
	\pstwo(A \mid B) = \frac{\pstwo (A \cap B)}{\pstwo (B)} \spk
	\pstwo(A \mid B) = \sum_* \pstwo (A \mid B \cap H_k) \pstwo (H_k \mid B) \spk
	\pstwo(H_j \mid A) = \frac{\pstwo (A \mid H_j) \pstwo (H_j)}{\sum_{i=1}^\infty \pstwo (A\mid H_i) \pstwo(H_i)} 
\]

$\mathcal B(X, Y)$: \prawo{4.1} przestrzeń ograniczonych, liniowych $X \to Y$ (wektopy) z normą $\|\Lambda\| = \sup\{\|\Lambda x\| : x \in X, \|x\| \le 1\}$.
Jest $\mathscr B$, gdy $Y$ jest $\mathscr B$.
Ważny skrót: $\langle x, x^* \rangle := x^*(x)$.
Dla kuli $B_{\le 1}$ w $X$ z normą i $x^* \in X^*$, $\|x^*\| = \sup \{|\langle x, x^* \rangle| : x \in B\}$ zamienia $X^*$ w $\mathscr B$.
Jeśli $B^*_{\le 1}$ to kula w $X^*$, zaś $x \in X$, to $\|x\| = \sup\{|\langle x, x^*\rangle| : x^* \in B^*\}$, więc $x^* \mapsto \langle x, x^*\rangle$ jest ograniczonym funkcjonałem na $X^*$ o normie $\|x\|$.
Sama kula $B^*_{\le 1}$ jest słabo* zwarta.
Ustalmy podprzestrzenie: $M$ w $X$ ($\mathscr B$), $N$ w $X^*$, wtedy $M^\perp = \{x^* \in X^* : (\forall x \in M) (\langle x, x^* \rangle = 0)\}$, ${}^\perp N = \{x \in X : (\forall x^* \in N) (\langle x, x^* \rangle = 0)\}$ to \wcht{anihilatory}.
${}^\perp(M^\perp)$ to normowe domknięcie $M$ w $X$, $({}^\perp N)^\perp$: słabe* domknięcie $N$ w $X^*$.
Jeśli $M$ to domknięta podprzestrzeń $X$ ($\mathscr B$), to każdy $m^* \in M^*$ rozszerza się do $x^* \in X^*$.
$m^* \mapsto x^* + M^\perp$ to izometryczny izomorfizm $M^* \to X^*/M^\perp$; zaś $y^* \mapsto y^* \pi$ to izometryczny izomorfizm $(X/M)^* \to M^\perp$, przy czym $\pi \colon X \to X/M$ jest ilorazowym.

Dla \prawo{4.2} unormowanych $X, Y$, każdemu $T \in \mathcal B(X,Y)$ odpowiada dokładnie jeden $T^* \in \mathcal B(Y^*, X^*)$, że $\langle Tx , y^*\rangle = \langle x, T^*y^*\rangle$.
Spełnia  $\|T^*\| = \|T\|$.
Jeśli $X, Y$ są $\mathscr B$, to $\ker T^* = (\im T)^\perp$ i $\ker T = {}^\perp (\im T^*)$, zatem $\ker T^*$ jest słabo* domknięte; $\im T \subseteq Y$ jest gęsta $\Lra$ $T^*$ wzajemnie jednoznaczne, $T$ wzajemnie jednoznaczne $\Lra$ $\im T^*$ słabo gęsta.
\wcht{Trzy tw. o domkniętym obrazie}: jeśli $U, V \subseteq X, Y$ ($\mathscr B$) to $B_{< 1}$ kule i $\delta > 0$, to $\|T^*y^*\| \ge \delta \|y^*\|$ $\Ra$ $\operatorname{cl}T(U) \supset \delta V$ $\Ra$ $T(U) \supset \delta V$ $\Ra$ $T(X) = Y$.
Ostatni warunek pociąga pierwszy dla pewnego $\delta$.
NWSR: $T$ ,,na''; $T^*$ to bijekcja; $\im T^*$ jest normowo domknięty.
Inne NWSR: $\im T$ domknięty w $Y$, $\im T^*$ słabo* w $X^*$; normowo domknięty w $X^*$.

Liniowe \prawo{4.3} $X \to Y$ (obie są $\mathscr B$) jest \wcht{zwarte}, jeśli obraz kuli $B_{< 1}$ jest prezwarty, pociąga ograniczoność.
Uwaga: $\mathcal B(X)$ jest nie tylko p. Banacha, ale także algebrą.
\wcht{Widmo} $\sigma(T)$ dla $T \in \mathcal B(X)$: zbiór skalarów $\lambda$, że $T - \lambda I$ nie jest \wcht{odwracalny} (,,$ST = I = TS$''), równoważnie: $\im T - \lambda I \neq X$ $\Lra$ $T - \lambda I$ nie jest bijekcją ($\lambda$ to \wcht{wartość własna}).
Operator $T \in \mathcal B(X, Y)$ o obrazie skończonego wymiaru jest zwarty.
Obraz zwartego operatora, jest skończonego wymiaru, gdy jest też domknięty.
Operatory zwarte tworzą normowo domkniętą podprzestrzeń $\mathcal B(X, Y)$.
Jeśli $T \in \mathcal B(X)$ jest zwarty i $\lambda \neq 0$, to $\dim \mathcal N(T - \lambda I) < \infty$.
Jeśli $\dim X = \infty$, a $T \in \mathcal B(X)$ jest zwarty, to $0 \in \sigma(T)$.
Jeśli $S,T \in \mathcal B(X)$ i $T$ jest zwarty, to $ST$, $TS$ też.
Ograniczony $T \colon X \to Y$ jest zwarty $\Lra$ $T^*$ jest zwarty.
Jeśli $\lambda \neq 0$, zaś $T \in \mathcal B(X)$ jest zwarty, to $T - \lambda I$ ma domknięty obraz.
Każdy $\lambda \in \sigma(T)$ jest wartością własną $T$ i $T^*$.
Zbiór $\sigma(T)$ jest zwarty, co najwyżej przeliczalny i tylko $0$ może być jego punktem skupienia.
Co więcej, liczby: $\dim \ker (T - \lambda I)$, $\dim X / \im (T - \lambda I)$, $\dim \ker (T^* - \lambda I)$, $\dim X^* / \im (T - \lambda I)$ są równe i skończone.
% Gdy $E$ jest zbiorem w-wartości $\lambda$ dla $T \in \mathcal B(X)$, że $|\lambda| >r > 0$, to dla $\lambda \in E$ jest $\mathcal R(T - \lambda I) \neq X$.
% Dodatkowo $E$ jest skończony.


%(\wcht{Trajektorie to krzywe rozwiązań}) 
Dla \prawo{5.1} dwuwymiarowego układu autonomicznego $\dot x = f(x,y)$, $\dot y = g(x,y)$ z prawą stroną Lipschitza na $D \subseteq_o \R^2$ i $(\xi, \eta) \in D$ mamy:
jeśli $f(x,y) \neq 0$ na $D$, to trajektoria $O(\xi, \eta)$ pokrywa się z maksymalną krzywą rozwiązania $L(\xi, \eta)$ dla AWP $y'(x) = (g/f)(x,y)$, $y(\xi) = \eta$ [dla $g(x,y)$ analogicznie, wziąć odwrotności].
Układ jest \wcht{hamiltonowski}, gdy istnieje $\mathscr C^1$-funkcja Hamiltona, $H \colon D \to \R$, że $\partial_x H(x,y) = -g(x,y)$ i $\partial_y H(x,y) = f(x,y)$ na $D$; stała wzdłuż trajektorii.
\emph{Polowanie na trajektorie} (autonomicznego układu):
sprawdź, czy jest hamiltonowski (na każdym prostokącie $\partial_xf(x,y) + \partial_yg(x,y) \equiv 0$); $H_0$ jest funkcją Hamiltona; trajektorie leżą na poziomicy $N(\xi, \eta)$ dla $(\xi, \eta) \in D$.
\wcht{Pierwsza całka} to $\mathscr C^1$-funkcja $F \colon D \to \R$, że $[\partial_xF \cdot f + \partial_yF \cdot g](x,y) \equiv 0$ dla $(x,y) \in D$.
Pierwsza całka też jest stała wzdłuż trajektorii.
Jeśli $m \colon D \to \R$ jest czynnikiem całkującym dla $f(x,y) [\textrm{d}y/\textrm{d}x] - g(x,y) = 0$, zaś $S \colon D \to \R$ pierwotną dla $m(x,y) [f(x,y)  (\textrm{d}y/\textrm{d}x) - g(x,y)] = 0$, to $S(x,y)$ jest pierwszą całką dla układu.
\[
	H_0(x,y) = \int_{y_0}^y f(x,v) \,\textrm{d}v - \int_{x_0}^x g(w,y_0) \,\textrm{d}w \spk
	N(\xi, \eta) = \left\{(x,y) \in \R^2 : H(x,y) = H(\xi, \eta)\right\}
\]

Dany jest \prawo{5.2} autonomiczny, płaski układ $\dot x= f(x,y)$, $\dot y = g(x,y)$ z $f, g \colon \R^2 \setminus\{(0,0)\} \to\R$ Lipschitza.
Do tego mamy, dla $(r, \theta) \in (0,\infty) \times \R$, określony nowy układ $\dot r = p(r, \theta)$, $\dot \theta = q(r, \theta)$.
Jeśli $(\mu_1, \mu_2) \colon I \to \R^2$ jest rozwiązaniem nowego, to starego są funkcje $\mu_1 \cos \mu_2$ oraz $\mu_1 \sin \mu_2$.
\emph{Poczwary przenoszą się między układami w przyjemny i przewidywalny sposób.}
Klasyfikacja \prawo{5.3} równań $\dot z = Az$, $A \in M_{2 \times 2} (\R)$: dwie wartości własne, jedna; jedna klatka Jordana, zespolone. Wuchta obrazków.
\[
	p(r, \theta) = f(r \cos \theta, r \sin \theta) \cdot \cos \theta + g(r \cos \theta, r \sin \theta) \cdot \sin \theta \spk
	r q(r,\theta) = g(r \cos \theta, r \sin \theta) \cdot \cos \theta - f(r \cos \theta, r \sin \theta) \cdot \sin \theta
\]

W \prawo{6.2} tym rozdziale żyjemy w $(\Omega, \mathcal M, \pstwo)$, wszystkie $\sigma$-ciaua zawierają się w $M$.
Mamy \wcht{warunkowa wartość oczekiwaną} (khm-1) dla $\pstwo_A(B)$ równego $\pstwo (B \mid A)$ i $X$ o skończonej nadziei.
Jeżeli $\pstwo(A) > 0$, to khm-2.
Jeżeli $\{A_i\}$ to przeliczalne rozbicie $\Omega$ i $\pstwo(A_i) > 0$, zaś z-losowa $X$ jest caukowalna, to khm-3.
Jeśli $\Omega = \bigcup_i B_i$, $\pstwo(B_i) > 0$ i $\mathcal G = \sigma(B_i : i \in I)$, to $\expected (X \mid \mathcal G)(\omega) = \sum_{i \in I} \expected (X \mid B_i) [\omega \in B_i]$. 
Tak zefiniowana z-losowa jest $\mathcal G$-mierzalna; dla $B \in \mathcal G$ mamy khm-4.
\[
	\expected (X \mid A) := \int_\Omega X \,\D \pstwo_A = \int_A \frac{X}{\pstwo(A)} \,\D \pstwo \spk
	\expected X = \sum_{i=0}^\infty \expected (X \mid A_i) \cdot \pstwo (A_i) \spk
	\int_B X \,\D\pstwo = \int_B \expected (X \mid \mathcal G) \,\D\pstwo
\]

\wcht{Warunkowa wartość oczekiwana} \prawo{6.3} całkowalnej z-losowej $X$ pod warunkiem $\sigma$-ciaua $\mathcal F \subseteq \mathcal M$ to $\mathcal F$-mierzalna z-losowa $\expected (X \mid \mathcal F)$, że dla $A \in \mathcal F$ całki z ,,$X \, \D \pstwo$'', ,,$\expected (X \mid \mathcal F) \, \D \pstwo$'' nad $A$ są równe.
Zawsze istnieje, jednoznacznie z dokładnością do zdarzeń o p-stwie zero.
\wcht{Wielkie twierdzenie}: z-losowe $X, X_i$ mają skończoną nadzieję, $\mathcal G \subseteq \mathcal F \subseteq \mathcal M$ to $\sigma$-ciaua.
\wcht{Nierówność Jensena}: dla wypukłej $\varphi \colon \R \to \R$, z-losowych $X$, $\varphi(X)$ z $L^1(\Omega, \mathcal M, \pstwo)$ i $\sigma$-ciaua $\mathcal F \subseteq \mathcal M$ mamy $\varphi (\expected(X \mid \mathcal F)) \le \expected (\varphi(X) \mid \mathcal F)$ p.n.
Jeżeli z-losowa $X$ spełnia $\expected |X| < \infty$, zaś $Y$ ma wartości w $\R^n$, to istnieje borelowska $h \colon \R^n \to \R$, że $\expected (X \mid Y) = h(Y)$.
Warwaroczem z-losowej $X$ pod warunkiem $\{Y = y\}$ nazywamy $h(y)$.

\wcht{4.A}: dla $\mathcal F$-mierzalnej $X$: $\expected (X\mid \mathcal F) = X$ p.n.
\wcht{4.B}: dla $X \ge 0$: $\expected (X \mid \mathcal F) \ge 0$ p.n.
\wcht{4.C}: $|\expected(X \mid \mathcal F)| \le \expected (|X| \mid \mathcal F)$ p.n.
\wcht{4.D}: $\expected (\alpha X_1 + \beta X_2 \mid \mathcal F)$ jest równe $\alpha \cdot \expected (X_1 \mid \mathcal F) + \beta \cdot \expected (X_2 \mid \mathcal F)$ p.n.
\wcht{4.E}: $X_n \uparrow X$ implikuje $\expected (X_n \mid \mathcal F) \uparrow \expected(X \mid \mathcal F)$ p.n. 
\wcht{4.G}: $\expected X = \expected( \expected (X \mid \mathcal F))$ p.n.
\wcht{4.F}: $\expected (X \mid \mathcal G)$, $\expected (\expected (X \mid  \mathcal F) \mid  \mathcal G)$ oraz $\expected (\expected (X \mid  \mathcal G) \mid \mathcal F)$ są równe sobie p.n.
\wcht{4.H}: dla niezależnych $\mathcal F$ i $\sigma(X)$: $\expected(X \mid \mathcal F) = \expected X$ p.n.
\wcht{4.I}: dla ograniczonej oraz $\mathcal F$-mierzalnej z-losowej $Y$, $\expected (XY \mid \mathcal F) = Y \expected (X \mid \mathcal F)$.

\wcht{Warunkowy Fatou}: dla $X_n \ge 0$ mamy $\expected (\liminf X_n \mid \mathcal F) \le \liminf \expected (X_n \mid \mathcal F)$.
\wcht{Levi}: gdy $|X_n (\omega)| \le Y (\omega)$, $\expected Y < \infty$ oraz $X_n \to X$ p.n., to $\lim_n \expected (X_n \mid \mathcal F) = \expected (X \mid \mathcal F)$ p.n.
\wcht{Wariancja}: $\variance (X \mid \mathcal F) := \expected ((X - \expected (X \mid \mathcal F))^2 \mid \mathcal F)$, gdy $\expected X^2 < \infty$, wtedy $\variance X = \expected \variance (X \mid \mathcal F) + \variance \expected (X \mid \mathcal F)$.
\wcht{Fubini}: $\sigma$-ciauo $\mathcal F \subseteq \mathcal M$, p. mierzalna $(E, \Sigma, \mu)$, $X \in L^1 (E \times \Omega, \Sigma \times \mathcal F, \mu \times \pstwo)$.
Wtedy khm-1, khm-2.
\wcht{Niezależność} $\sigma$-ciau $\mathcal F_1, \ldots, \mathcal F_n; \mathcal G \subseteq M$: $\pstwo (\bigcap_{i=1}^n A_i \mid \mathcal G) = \prod_{i=1}^n \pstwo (A_i \mid \mathcal G)$ dla każdego $A_i \in \mathcal F_i$.
$\mathcal F, \mathcal H$ są wnz względem $\mathcal G$ $\Lra$ dla każdego $H \in \mathcal H$, $\pstwo (H \mid \mathcal F \vee \mathcal G) = \pstwo (H \mid \mathcal G)$ p.n.
\[
	\expected \left | \int_E \expected (X_s \mid \mathcal F) \mu(\D s) \right| < \infty \spk
	\expected \left[\left. \int_E X_s \mu(\D s) \right\mid \mathcal F \right] = \int_E \expected(X_S \mid \mathcal F) \mu (\D s)
\]

\wcht{P-stwo warunkowe} \prawo{6.4} $A \in \mathcal M$ pod warunkiem $Y = y$: $\pstwo(A \mid Y = y) := \expected (1_A \mid Y= y)$. 
Gdy $(X,Y)$ ma ciągły rozkład o gęstości $g$, to khm-1 i khm-2 dla tych borelowskich $\varphi$, że $\expected |\varphi(x)| < \infty$ (gdy mianownik się zeruje, kładziemy $0$ po prawej).
\wcht{Uogólniony Bayes} $\mathcal G \subseteq \mathcal F$: $\sigma$-ciało, $B \in \mathcal G$, $A \in \mathcal F$, $\pstwo(A) > 0$ i ,,$\pstwo(A\mid \mathcal G) = \expected(\mathbb I_A \mid \mathcal G)$ dają khm-3.
\wcht{Abstrakcyjny}: $P, Q$ miarami probabilistycznymi na $(\Omega, \mathcal F)$, że gęstość $\D Q / \D P = Z >0$ istnieje, $\mathcal G \subseteq \mathcal F$, $X$: z-losowa $Q$-caukowalna; wtedy: $\expected_Q X = \expected_P XZ$ i khm-4 jest równe $\expected_Q (X \mid \mathcal G)$.
\[
	\pstwo (X\in B \mid Y) = \frac{\int_B g(x, Y) \, \D x}{\int_\R g(x,Y) \, \D x} \spk
	\expected (\varphi(x) \mid Y) = \frac{\int_\R \varphi(x) g(x, Y) \, \D x}{\int_\R g(x,Y) \, \D x} \spk
	\pstwo (B \mid A) = \frac{\int_B \pstwo (A \mid \mathcal G) \, \D \pstwo}{\int_\Omega \pstwo(A \mid \mathcal G) \, \D \pstwo} \spk
	\frac{\expected_P(XZ \mid \mathcal G)}{\expected_P (Z \mathcal G)}
 \]

\wcht{P-stwo} $B$ \prawo{6.5} pod warunkiem $\sigma$-ciaua $\mathcal F$: $\mathcal F$-mierzalna z-losowa $\pstwo (B \mid \mathcal F) := \expected (1_B \mid \mathcal F)$ o wartościach w $[0,1]$.
Khm-1 ($A \in \mathcal F$); jeśli $B_n$ są rozłączne parami, to khm-2.
\wcht{Regularny rozkład warunkowy} względem $\mathcal F$: funkcja $\pstwo_{\mathcal F} \colon \mathcal M \times \Omega \to [0,1]$, że $\pstwo_{\mathcal F} (B, \cdot)$ jest wersją $\expected(1_B, \mathcal F)$, zaś $\pstwo_{\mathcal F}(\cdot, \omega)$ to rozkłady p-stwa na $\mathcal M$.
Dla całkowalnej z-losowej $X$: khm-3. 
,,Regrowar'' nie musi istnieć, lecz istnieje on dla z-losowej $X$ pod warunkiem $\sigma$-ciaua $\mathcal F$ (funkcja $\pstwo_{X \mid \mathcal F}$ na $\mathfrak B(\R) \times \Omega$, że $\pstwo_{X \mid \mathcal F}(B, \omega) = \pstwo (X \in B \mid \mathcal F)(\omega)$, zaś $\pstwo_{X \mid \mathcal F}(\cdot, \omega)$ to rozkłady p-stwa na $\mathfrak B(\R)$).
\[
	\pstwo (A \cap B) = \int_A \pstwo (B \mid \mathcal F) \, \D \pstwo \spk
	\pstwo \left(\left. \bigcup_{n=1}^\infty B_n \right \mid \mathcal F \right) = \sum_{n=1}^\infty \pstwo (B_n \mid \mathcal F) \textrm{ p.n.} \spk
	\expected (X \mid \mathcal F)(\omega) = \int_{\Omega} X (\tilde{\omega}) \pstwo_{\mathcal F} (\D{} \tilde{\omega}, \omega) \textrm{ p.n.}
\]

\fancyfoot[LF]{strona \thepage { }z \pageref{LastPage} [od 213 do 355]}

Z-losowa $X_n$ generuje \prawo{7.2} $\sigma$-ciauo $\mathcal F_n \subseteq \mathcal F$, zasób wiedzy o $(\Omega, \mathcal F, \pstwo )$ w chwili $n$; $\sigma$-ciało \wcht{ogonowe/resztowe} to $\mathcal F_\infty = \bigcap_{\downarrow} \mathcal F_{n, \infty}$, gdzie $\mathcal F_{n, \infty}$ to wiedza o przyszłości ($\sigma(\mathcal F_n, \mathcal F_{n+1}, \dots)$).
\wcht{Prawo 0-1 Kołmogorowa}: jeśli $\sigma$-ciała $\mathcal F_n$ są nz, to dla $A \in \mathcal F_{\infty}$ jest $\pstwo (A) \in \{0,1\}$; zatem: z-losowe $X_n$ są nz $\Ra$ $\sum_{n=1}^\infty X_n$ zbiega z p-stwem $0$ lub $1$.
\wcht{Tw. Hewitta-Savage'a}: [$X_1, \ldots$ są iid, $B \in \mathfrak B(\R^\infty)$, $A = \{(X_1, \ldots) \in B\}$ jest \wcht{permutowalne}: jeśli bijekcja $\pi \colon \N \to \N$ permutuje skończenie wiele wyrazów, to $\pi(A) := \{(X_{\pi(1)}, X_{\pi(2)}, \ldots) \in B\} = A$] pociąga $\pstwo (A) \in \{0, 1\}$.

\wcht{Nierówność Levy'ego-Ottavianiego}: \prawo{7.3} gdy z-losowe $X_i$ są nz i $\varepsilon > 0$, to khm-1; $X_i$ symetryczne: khm-2.
\wcht{Tw. o dwóch szeregach}: $X_i$ są nz, $\sum_{i=1}^\infty \expected X_i$ i $\sum_{i=1}^\infty \variance X_i$ są zbieżne $\Ra$ $\sum_{i=1}^\infty X_i$ też, ale p.n.
\wcht{Tw. Kołmogorowa o trzech}: dla pewnego $c > 0$, $(*)$ są zbieżne $\Ra$ $\sum_i X_i$ zbiega p.n. $\Lra$ $\sum_i X_i$ zbiega wg p-stwa (\wcht{Levy}) $\Ra$ dla każdego $c > 0$\ldots{} 
\wcht{Nierówność Kołmogorowa}: jeśli $X_1, \dots, X_n$ są nz, $\expected X_i = 0$, $\expected X_i^2 < \infty$ i $\varepsilon > 0$, to $p_n = \pstwo (\max_{k \le n} |S_k| \ge \varepsilon) \le \expected (S_n/\varepsilon)^2$; $P(|X_i| \le C) = 1$ i $\expected S_n^2 > 0$ dają $p_n \ge 1 - (C + \varepsilon)^2 / \expected S_n^2$.
\hfill $X^{(c)} = X [|X| \le c]$. $S_k = X_1 + \dots + X_k$.
\[
	\pstwo (\max_{i \le n} |S_i| > \varepsilon) \le 3 \max_{i \le n} \pstwo (|S_i| > \varepsilon/3) \spk
	\pstwo (\max_{i \le n} |S_i| > \varepsilon) \le 2 \pstwo (|S_n| > \varepsilon)
	\hfill (*)
	\sum_{i=1}^\infty \expected X_i^{(c)} \spk
	\sum_{i=1}^\infty \variance X_i^{(c)} \spk
	\sum_{i=1}^\infty \pstwo (|S_i|> c)
\]

Tu $S_n$ \prawo{7.4} zlicza sukcesy w schemacie Bernoulliego ($n$ prób, p-stwo sukcesu $p$).
\wcht{PWL, Bernoulli}: $\lim_{n \to \infty} \pstwo (|S_n/n - p| \le \varepsilon) = 1$. 
\wcht{Nierówność Bernsteina}: $\pstwo (|S_n/n - p| > \varepsilon) \le 2 \exp (-2n \varepsilon^2)$. 
\wcht{MPWL Bernoulliego}: $S_n/n \to p$ p.n., ogólnie: \wcht{MPWL} dla $X_n$ oznacza, że $(T_n - \expected T_n)/n \to 0$ p.n.; zaś \wcht{SPWL}, że wg p-stwa (np. gdy $(\variance T_n)/n^2 \to 0$, $X_n$ są parami nieskorelowane i mają ograniczony drugi moment).
\wcht{Tw. Kołmogorowa}: $X_i$ to nz z-losowe, $\variance X_n < \infty$, dodatni ciąg $b_n$ rozbiega monotonicznie do $\infty$, że $\sum_{n=1}^\infty \variance (X_n/b_n) < \infty$: MPWL.
\wcht{MPWL Kołmogorowa}: $X_i$ są iid, $\expected |X_1| <\infty$.
\emph{Dla zwiększenia czytelności $T_n = X_1 + \dots + X_n$.}

\wcht{Tw. Bernsteina}: jeśli $\variance X_n \le C < \infty$ i $\rho(X_i, X_j) \to 0$ (współczynnik korelacji) dla $|i - j| \to \infty$, to $X_n$ spełnia SPWL.
\wcht{Tw. Chinczyna}: jeśli parami nz z-losowe $X_n$ mają jeden rozkład i $|\expected X_1| < \infty$, to dla $X_n$ zachodzi SPWL.
\wcht{Tw. Marcinkiewicza}: jeśli $X_n$ są iid i $\expected |X_1|^p <\infty$ dla pewnego $p \in (0,2)$, to $\pstwo (\lim_n (T_n-nu) / n^{1/p} = 0) = 1$, gdzie $u = \expected X_1 \cdot [p \ge 1]$.
\wcht{Tw. Etemadiego}: jeżeli parami nz z-losowe $X_n$ mają jeden rozkład i $\expected |X_1| < \infty$, to $T_n/n \to \expected X_1$ p.n.

\wcht{Tw. Poissona}: \prawo{7.5} jeżeli $n \to \infty$, $p_n \to 0$, i $n p_n \to \lambda > 0$, to khm-1.
Jeżeli iid z-losowe $X_1, \ldots$ mają rozkład $\pstwo (X_i = 1) = p$ i $1 - p = \pstwo (X_i = 0)$, $\lambda = np$ i $T_n = X_1 + \dots + X_n$, zaś $B \in \mathfrak B(\R)$, to mamy mocne uogólnienie, khm-2.
\[
	{n \choose k} \cdot p_n^k (1-p_n)^{n-k} \stackrel{n \to \infty}{\to} \frac{\lambda^k}{k!} e^{-\lambda} \spk
	\left|\pstwo (T_n \in B) - \sum_{k = 0}^\infty \frac{\lambda^k}{k!} \cdot \frac{[k \in B]}{\exp \lambda} \right| \le \frac{\lambda^2}{n} 
\]

Przyjmijmy $B(k,n,p) = C_k^n p^k q^{n-k}$, $h = (npq)^{-1/2}$, $\delta_k = k - np$ oraz $x_k= \delta_k h$.
\wcht{Tw. de Moivre'a-Laplace'a}: \prawo{7.6} gdy $h |x_k| \max(p,q) \le .5$, to khm-1, przy czym $|R(n,k)| \le {3|x_k|h}/{4} + |x_k|^3h/3 + 1/3n$ (\wcht{lokalne}).
Kiedy $h \max(|x_a|, |x_b|) \max(p,q) \le .5$ i khm-2, to mamy \wcht{integralne}: $\pstwo (a \le S_n \le b) = [\Phi(x_{b + 1/2}) - \Phi(X_{a - 1/2})] \exp D(n,a,b)$ .
\wcht{Nierówność Bernsteina}: jeśli $X_1, \dots, X_n$ są iid, $|X_i| \le K$, $\expected X_i = 0$, $\expected X_i^2 = \sigma^2$ oraz $S_n = X_1 + \dots + X_n$, to khm-3.
\wcht{Trzy sigmy dla schematu Bernoulliego}: $\pstwo (np - 3/h < S_n < np + 3/h) \ge 0.997$ gdy $n > 9 \max (q/p, p/q)$.
\[
	B(k,n,p) = \frac{\exp(R(n,k) - x^2_k/2)}{\sqrt{2\pi n p q}} \spk
	|D(n,a,b)| \le \max_{k\in\{a,b\}} \left[\frac{5 |x_k| h}{4} + \frac{|x_k|^3 h}{3} \right] + \frac{1}{3n} + \frac{h^2}{8} \spk
	\frac{\pstwo (|S_n| > t \sigma \sqrt{n})}{2} \le \exp \frac{-t^2/2}{1 + Kt / [3\sigma \sqrt{n}]}
\]

\fancyfoot[LF]{strona \thepage { }z \pageref{LastPage} [od 356 do 417]}

\newpage

\wcht{Równanie/problem Sturma-Liouville'a}: \prawo{SL} znamy $p, q$ oraz $w$, dla jakich $\lambda$ istnieją nietrywialne rozwiązania $y$?
Takiej postaci są np. Bessela, Legendre'a.
\[
	\frac{d}{dx} \left[ p(x) \frac{dy}{dx}\right] + [\lambda w(x) - q(x)] y = 0
\]

% https://en.wikipedia.org/wiki/Sturm%E2%80%93Liouville_theory

Mamy \prawo{GF} $\dot x = A(t)x$, macierz $n \times n$ z okresem $T$.
Jeśli $\Phi(t)$ jest macierzą fundamentalną, to $\Phi(t+T) = \Phi(t) \Phi^{-1}(0)\Phi(T)$.
Dalej, jeśli $\exp(TB) = \Phi^{-1}(0)\Phi(T)$ (to \wcht{macierz monodromii}), to istnieje $T$-okresowa macierz $P(t)$ ($n \times n$), że $\Phi(t) = P(t) \exp(tB)$: tutaj $B$ może być zespolona.
Do tego jest rzeczywista macierz $R$ i rzeczywista $2T$-okresowa $Q(t)$, że $\Phi(t) = Q(t)\exp(tR)$ (to było \wcht{tw. Floqueta}, \datum{1883}).

% http://mathworld.wolfram.com/FloquetsTheorem.html

Jeśli \prawo{PSL} $f(t) \colon [0, \infty)\to \R$ jest przedziałami ciągła i spełnia $|f(t)| \le Me^{ct}$, to dla $s > c$ ma \wcht{transformatę Laplace'a}.
Różne funkcje mają różne transformaty (\wcht{tw. Lercha}).
%AWP $a y'' + by' + cy = f(t)$, $y(0) = y_0$ i $y'(0) = y_0'$ upraszcza się przez nałożenie transformaty na obie strony, uporządkowanie i użycie odwrotnej.
Fakt: $\mathcal L\{(f*g)(t)\} = \dots = \mathcal L\{f(t)\} \mathcal L\{g(t)\}$. 
\wcht{Całka Bromwicha}: khm-4, przy czym pionowy kontur $\gamma$ nie dotyka osobliwości $F(s)$.
Dalej, $\mathcal L\{e^{at}f(t)\} (s) = F(s-a)$ oraz $\mathcal L\{-t f(t)\}(s) = F'(s)$.
\[
	\mathcal L\{f(t)\} = \int_0^\infty \frac{f(t)}{e^{st}} \,\textrm{d}t \spk
	\mathcal L\left\{\int_0^t f(\tau) g(t-\tau) \textrm{d}\tau\right\} \spk
	\mathcal L[f^{(n)}(t)] = s^n \mathcal L[f(t)] - \sum_{k=0}^{n-1} s^kf^{(n-k-1)}(0) \spk
	f(t) = \int_{\gamma-i \infty}^{\gamma + i \infty} \frac{e^{st} F(s)}{2\pi i}\,\textrm{d}s
\]

Tablice transformat Laplace'a:
\begin{align*}
\exp(at) & \mapsto 1/(s-a) &
t^{n} & \mapsto {n!}/{s^{n+1}} \\
%\sqrt{t} & \mapsto \sqrt{\pi/4s^3} \\
\sin at & \mapsto {a} / (s^2+a^2) &
\cos at & \mapsto {s} / (s^2+a^2) \\
\sinh at & \mapsto a/ (s^2-a^2) &
\cosh at & \mapsto s / (s^2-a^2) \\ 
t \sin at & \mapsto {2as}/(s^2+a^2)^2 &
t \cos at & \mapsto (s^2-a^2)/(s^2+a^2)^2 \\
e^{at} [\sin bt] & \mapsto {b}/{[(s-a)^2 + b^2]} &
e^{at} [\cos bt] & \mapsto {(s-a)}/{[(s-a)^2 + b^2]} \\
e^{at} [\sinh  bt] & \mapsto {b}/[(s-a)^2 - b^2] &
e^{at} [\cosh bt] & \mapsto {(s-a)}/[(s-a)^2 - b^2]
\end{align*}

Dane \prawo{??} jest zwyczajne $p(x) y''(x) + q(x) y'(x) + r(x) y(x) = 0$. % czyżby Braun?
Rozwiązań szukamy szeregami $y(x) = \sum_{k=0}^\infty a_k(x-x_0)^k$.
Jeśli $p(x_0)\neq 0$, to $x_0$ jest \wcht{ordynarny}, jeśli nie, to \wcht{singularny}.
Singularny jest \wcht{regularny}, jeśli $\lim_{x \to x_0} (x-x_0)q(x) / p(x)$ oraz $\lim_{x \to x_0} (x-x_0)^2 r(x) / p(x)$ są skończone.
\wcht{Metoda Frobeniusa}: jeśli $x = 0$ jest regularny singularny, to przynajmniej jedno rozwiązanie równania jest postaci $x^r \sum_{k=0}^\infty a_k x^k$.
R--e indeksowe ma pierwiastki $r_1, r_2$.
$y_1 = x^{r_1} \sum_{k=0}^\infty a_k x^k$.
Jeśli $r_1 = r_2$, to $y_2 = x^r \sum_{k=0}^\infty b_kx^k y_1 \ln x$.
Jeśli $r_1 - r_2 \in \Z$, to rozwiązaniami są $y_1 = x^{r_1} \sum_{k=0}^\infty a_kx^k$ oraz $y_2 = x^{r_2} \sum_{k=0}^\infty b_k x^k + C y_1 \ln x$.
Wreszcie dla zespolonych: wziąć $\Re$, $\Im$ z $x^{r_1} \sum_{k=0}^\infty a_k x^k$.

\emph{Uzmiennianie} \prawo{m} $a_n y^{(n)} + \ldots + a_1 y' + a_0 y = f(x)$.
Jeśli $C_1 y_1 + \ldots + C_n y_n$ rozwiązuje jednorodne, to traktujemy $C_i$ jako funkcji $C_i(x)$, które wyznacza się z gargantuicznej macierzy (Wrońskiego).
\[
\begin{pmatrix} y_1 & \ldots & y_n \\ y_1^\prime & \ldots & y_n^\prime \\ \vdots & \ddots & \vdots \\ y_1^{(n-1)} & \ldots & y_n^{(n-1)} \end{pmatrix} \begin{pmatrix} C_1^\prime \\ C_2^\prime \\ \vdots \\ C_n^\prime \end{pmatrix} = \begin{pmatrix} 0 \\ 0 \\ \vdots \\ \frac{f(x)}{a_n} \end{pmatrix}
\]

Jeśli jedno rozwiązanie dla $\ddot y + P(x) \dot y + Q(x) y = 0$ jest znane, $y_1$, to drugie można znaleźć przez redukcję rzędu.
Mamy bowiem $(dW)/W = -P(x)dx$, gdzie $W = y_1y'_2 - y_1'y_2$ jest wrońskianem.
Scałkowanie daje $\int_a^x dW/W = -\int_a^x P(x') dx'$, czyli $$\ln |W(x) / W(a)| = - \int_a^x P(x') dx'$$.
Wynika stąd, że $W(x) = W(a) \exp |-\int_a^x P(x') dx'$.
Ale $W = y_1^2 d/dx \, (y_2 / y_1)$, więc $$y_2 = y_1(x) W(a) \int_b^x \exp[-\int_a^{x'} P(x'')dx''] / [y_1(x')]^2 \,dx'$$.

Jak rozwiązać $\dot x = Ax + p$ dla wektora $x$?
Jeśli $p=0$, to rozwiązaniem jest $x(t) = \exp(At)$.
Znajdź wartości własne $\lambda_i$ oraz wektory własne $u_i$.
Policz $x_i = \exp(\lambda_i t) u_i$.
Rzeczywiste wektory są rozwiązaniem jednorodnego równania.
Jeśli $A$ jest $2 \times 2$, to rozwiązaniem jest $\Re [x_i]$ oraz $\Im [x_i]$.
Jeśli równanie nie jest jednorodne, to znajdź rozwiązanie szczególne przez $x^*(t) = X(t) \int X^{-1}(t)p(t) dt$, gdzie kolumny $X$ to $x_i$.
Jeśli jest jednorodne, popatrz na rozwiązania postaci $x = \xi \exp(\lambda t)$.
Ogólnie: $x(t) =x^*(t) + \sum_{i=1}^n c_i x_i$

\emph{Uzmiennianie stałej}. \prawo{Br}
Dla operatora $L = D^2 + p(x)D + q(x)$ chcemy rozwiązać $Lu(x) = f(x)$, dla danej $f$.
Niech $u_1$, $u_2$ będą rozwiązaniami jednorodnego.
Podejrzewamy, że $u_g(x)$ załatwi sprawę.
Umówmy się, że $A' u_1 + B' u_2 = 0$, wtedy można odtworzyć $A$, $B$.
Tutaj $W$ jest wrońskianem, $u_1u_2' - u_2u_1'$ (całość nazywa się \wcht{uzmiennianiem stałej}).
\[
	u_g(x)  = A(x) u_1(x) + B(x) u_2(x) \spk
	A(x) = - \int \frac{u_2(x) f(x) }{W}\,\textrm{d}x \spk
	B(x) = \int \frac{u_1(x) f(x) }{W}\,\textrm{d}x \spk
\]
\end{document}

http://eqworld.ipmnet.ru/en/solutions/ode.htm
Prüfer transformation
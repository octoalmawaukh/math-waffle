\subsection{Węzły pierwsze}

Okazuje się, że liczby pierwsze znane z teorii liczb mają swój odpowiednik wśród węzłów.
Odpowiednik ten jest ściśle związany z operacją sumy spójnej, którą {\bf \color{red} teraz lub wcześniej} zdefiniujemy.

\begin{definicja}
Kiedy $A$ i $B$, to ich sumą spójną $A \# B$ jest węzeł, który powstaje z poprzednich przez zszycie (?).
\end{definicja}

Okazuje się, że tak określone działanie jest dobrze określone, to znaczy nie zależy od wyboru łuków zszywających.
Możemy przejść do definicji węzłów pierwszych.

\begin{definicja}
Nietrywialny węzeł nazywamy pierwszym, kiedy nie można przedstawić go jako sumy spójnej $K_1 \# K_2$ dwóch nietrywialnych węzłów $K_1, K_2$ (nie jest złożony).
\end{definicja}

W tym miejscu pojawia się pytanie, czy niewęzeł nie jest być może złożony.
Gdyby tak rzeczywiście było, każdy węzeł okazałby się złożony, jako suma spójna siebie z niewęzłem.
Na szczęście przy pomocy powierzchni (Seiferta?) można pokazać, że niewęzeł nie powstaje z dwóch nietrywialnych węzłów przez wzięcie sumy spójnej.

Ze względu na niedostatecznie rozwinięty aparat matematyczny nie możemy podać dowodu następującego faktu, analogonu zasadniczego twierdzenia arytmetyki.

\begin{twierdzenie}[Schubert, 1949]
Każdy węzeł rozkłada się jednoznacznie na węzły pierwsze (z dokładnością do kolejności składników).
\end{twierdzenie}

Czy węzłów pierwszych jest nieskończenie wiele?
Tak, potrafimy nawet oszacować ich liczbę.
W roku 1987 C. Ernst wraz z D. Sumnersem w oparciu na wynikach Kauffmana, Murasugiego oraz Thistlethwaite'a dotyczących węzłów alternujących pokazali, że różnych węzłów pierwszych o $n$ skrzyżowaniach jest co najmniej $\frac 1 3 (2^{n- 2} - 1)$ dla $n \ge 4$, przy czym węzły lustrzane traktowane są jako różne.
Niedawno D. Welsh pokazał, że liczba takich węzłów jest ograniczona z góry przez funkcję wykładniczą od $n$.

Przykładem nieskończonej rodziny węzłów, które są parami różne, są węzły torusowe.

W roku 1985 M. Scharleman rozwiązał otwarty od ponad stu lat problem wiążący pierwszość z liczbą rozsupłującą\footnote{unknotting number}: jeżeli odwrócenie dokładnie jednego skrzyżowania wystarcza do zmiany węzła w niewęzeł, to jest on pierwszy.
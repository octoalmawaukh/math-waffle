\subsection{Wielomian Jonesa}

\begin{definicja}
\emph{Wielomian Jonesa} zorientowanego splotu to wielomian Laurenta $V(L)\in\Z[t^{1/2},t^{-1/2}]$ określony przez
\[V(L)=\left[ (-A)^{-3w(D)}\langle D\rangle\right]_{t^{1/2}=A^{-2}},\]
gdzie $D$ to dowolny diagram dla $L$.
\end{definicja}

\begin{twierdzenie}
Wielomian Jonesa jest niezmiennikiem zorientowanych splotów.
\end{twierdzenie}

\begin{proof}
%Skorzystamy z tego, że indeks zaczepienia jest niezmiennikiem.
Wystarczy pokazać niezmienniczość $(-A)^{-3w(D)}\langle D\rangle$ na ruchy Reidemeistera.
Ale
\[
(-A)^{-3
w\left(\begin{tikzpicture}[scale=0.025, baseline=-3]
	\clip (-12,-12) rectangle (1,7);
	\path[TEXTARC] (-10,7) .. controls (-10,3) and (-10,0) .. (-6,-4);
	\path[TEXTARC] (-6,0) .. controls (2,8) and (2,-10) .. (-6,-4);
	\path[TEXTARC] (-10,-11) .. controls (-10,-8) and (-10,-5) .. (-9,-4);
\end{tikzpicture}\right)}
\left\langle\begin{tikzpicture}[scale=0.025, baseline=-3]
	\clip (-12,-12) rectangle (1,7);
	\path[TEXTARC] (-10,7) .. controls (-10,3) and (-10,0) .. (-6,-4);
	\path[TEXTARC] (-6,0) .. controls (2,8) and (2,-10) .. (-6,-4);
	\path[TEXTARC] (-10,-11) .. controls (-10,-8) and (-10,-5) .. (-9,-4);
\end{tikzpicture}\right\rangle
=
(-A)^{-3
w\left(\ \begin{tikzpicture}[scale=0.025,baseline=-4]
	\path[TEXTARC] (-10,7) -- (-10,-11);
\end{tikzpicture}\ \right)+3}
(-A)^{-3}
\left\langle\ \begin{tikzpicture}[scale=0.025,baseline=-4]
	\path[TEXTARC] (-10,7) -- (-10,-11);
\end{tikzpicture}\ \right\rangle =
(-A)^{-3
w\left(\ \begin{tikzpicture}[scale=0.025,baseline=-4]
	\path[TEXTARC] (-10,7) -- (-10,-11);
\end{tikzpicture}\ \right)}
\left\langle\ \begin{tikzpicture}[scale=0.025,baseline=-4]
	\path[TEXTARC] (-10,7) -- (-10,-11);
\end{tikzpicture}\ \right\rangle.\qedhere
\]
\end{proof}

Wielomian Jonesa jest naprawdę potężnym narzędziem.
Pozwala bowiem odróżnić dowolne dwa węzły pierwsze o co najwyżej dziewięciu skrzyżowaniach.

\begin{hipoteza}
Nie istnieje nietrywialny węzeł, którego wielomian Jonesa nie odróżnia od niewęzła.
\end{hipoteza}

\begin{twierdzenie}
Wielomianem węzła $(m, n)$-torusowego jest
\[
	\frac {t^{(m-1)(n-1):2}}{1-t^2} \cdot (1 - t^{m+1} - t^{n+1} + t^{m+n}).
\]
\end{twierdzenie}
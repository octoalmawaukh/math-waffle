\subsection{Wielomian HOMFLY}
Po tym, jak Jones przedstawił światu swój wielomian w 1984 roku, matematycy zaczęli szukać jego uogólnienia zależnego nie od jednej, lecz dwóch zmiennych.
Pierwszym takim niezmiennikiem węzłów okazał się wielomian (Laurenta) HOMFLY.

\begin{definicja}
\emph{Wielomian HOMFLY} zorientowanego splotu to wielomian Laurenta zależny od $m$ i $l$, który spełnia dwa  aksjomaty: $P(\NieWezel) = 1$ oraz $l P(L_+) + l^{-1} P(L_-) + mP(L_0) = 0$ (przy oznaczeniach z twierdzenia \ref{tracheotomia}).
\end{definicja}

\begin{twierdzenie} Wielomian HOMFLY dla sum:
\begin{enumerate}
	\item $P(L_1 \sqcup L_2) = - (l + l^{-1}) m^{-1} P(L_1) P(L_2)$.
	\item $P(L_1 \# L_2) = P(L_1) P(L_2)$.
\end{enumerate}
\end{twierdzenie}

Okazuje się, że wielomian HOMFLY uogólnia jednocześnie wielomiany Jonesa i Alexandera:

\begin{twierdzenie} Dla dowolnego (splotu / węzła, nie/zorientowanego?) zachodzą równości:
\begin{itemize}
\item $V(t) = P(l = it^{-1}, m = i(t^{-1/2} - t^{1/2}))$,
\item $\Delta(t) = P(l = i, m = i(t^{1/2} - t^{-1/2}))$.
\end{itemize}
\end{twierdzenie}

Zaletą wielomianu HOMFLY jest to, że zazwyczaj wykrywa chiralność\footnote{węzeł chiralny: posiadający formy różnej handedness, które nie są lustrzanie symetryczne}, ale nie odróżnia enancjomerów węzłów 09-042, 10-048, 10-071, 10-091, 10-104, oraz 10-125.

Mutanty mają ten sam wielomian HOMFLY.
Okazuje się, że istnieje nieskończenie wiele parami różnych węzłów o tym samym wielomianie (Kanenobu 1986).
Przykłady:  (05-001, 10-132), (08-008, 10-129) (08-016, 10-156), oraz (10-025, 10-056) (Jones 1987).

M. B. Thistlethwaite wyznaczył HOMFLY dla tych, które mają mniej niż 14 skrzyżowań.
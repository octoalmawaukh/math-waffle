\subsection{Spin}
Przypomnijmy, że znak skrzyżowania na diagramie to liczba $1$ lub $-1$:
$\operatorname{sign}
	\begin{tikzpicture}[scale=0.03, baseline=-3]
	\path[TEXTARC,->] (-5,-5) -- (5,5);
	\path[TEXTARC] (5,-5) -- (1.5,-1.5);
	\path[TEXTARC,<-] (-5,5) -- (-1.5,1.5);
	\end{tikzpicture}
 = +1$,
$\operatorname{sign} \begin{tikzpicture}[scale=0.03, baseline=-3]
\path[TEXTARC,->] (5,-5) -- (-5,5);
\path[TEXTARC] (-5,-5) -- (-1.5,-1.5);
\path[TEXTARC,<-] (5,5) -- (1.5,1.5);
\end{tikzpicture} = -1$.

\begin{definicja}
	Niech $D$ będzie diagramem zorientowanego splotu lub węzła.
	\textbf{Spinem} $D$ jest $w(D) = \sum_c \operatorname{sign} c$, gdzie sumowanie przebiega po wszystkich skrzyżowaniach.
\end{definicja}

\begin{przyklad}
Spinem trójlistnika w takiej wersji jest $+3$:
\[
	\begin{tikzpicture}[scale=0.035]
		\clip (-16,-16) rectangle (16,17); %nie etykietuj
		\foreach \x in {270,30, 150}
		\path[TEXTARC,->-] (15+\x:6) .. controls (130+\x:25) and (200+\x:25) .. (225+\x:10);
		\node[red] (C1) at (30:14) {\small $+1$};
		\node[red] (C2) at (150:14) {\small $+1$};
		\node[red] (C3) at (270:14) {\small $+1$};
	\end{tikzpicture}
\]
\end{przyklad}

\begin{lemat}
Tylko I ruch Reidemeistera zmienia spin:
$
w(\begin{tikzpicture}[scale=0.025, baseline=-3]
	\clip (-12,-12) rectangle (1,7);
	\path[TEXTARC] (-10,7) .. controls (-10,3) and (-10,0) .. (-6,-4);
	\path[TEXTARC] (-6,0) .. controls (2,8) and (2,-10) .. (-6,-4);
	\path[TEXTARC] (-10,-11) .. controls (-10,-8) and (-10,-5) .. (-9,-4);
\end{tikzpicture})
=
w(\ \begin{tikzpicture}[scale=0.025,baseline=-4]
	\path[TEXTARC] (-10,7) -- (-10,-11);
\end{tikzpicture}\ )-1
$, pozostałe nie mają wpływu.
Spin nie zależy od orientacji.
\end{lemat}

\begin{proof}
Proste ćwiczenie.
\end{proof}
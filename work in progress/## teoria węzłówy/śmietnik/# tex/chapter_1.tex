% Maciek
\section{Wstęp, czyli trochę zakłamanej historii i intuicji}
Pojęcie węzła wywodzi się z teorii atomu Lorda Kelvina, która głosi, że atomy powstają z wirów energii (cokolwiek miałoby to znaczyć). Takie ``wiry`` Kelvin 
wyobrażał sobie jako okręgi w przestrzeni trójwymiarowej, lub, bardziej ogólnie, jako krzywe zamknięte w $\mathbb{R}^3$ bez samoprzecięc. Po tym, jak owa teoria się przyjęła 
(co prawda nie na długo), zaczęto szukać języka, w którym możnaby w miarę precyzyjnie (jak na tamte matematycznie prehistoryczne czasy) ją wyrazić. Jak mniemam (autor świadomie
używa nieco wycofanej formy ''jak mniemam'', zamiast np. ''z pewnością'', co pozwala mu nie wdawać się w dyskusję na temat historii matematyki, o której nie ma zielonego pojęcia)
te rozważania dały początek teorii węzłów. 

Praca Petera Taita, problemy, które naptokał, okazało się, że wiele z nich jest równoważnych temu, czy węzeł nie jest równoważny niewęzłowi, próba sklasyfikowania wszystkich 
węzłów o niewielkiej liczbie przecięć.

Trochę historii (najpierw Alexander, potem Reidemeister, potem jeszcze ktoś... dobra, to miał być szkielet)

Intuicja, kilka wstępnych przykładów... (muszę się nauczyć tej magicznej techologii służącej do rysowania tych... węzłów)



\section{Moja część...}

Jest dość dużo (więcej niż trzy) definicji węzła. Każda z nich w pewnym sensie oddaję intuicję, która kryje się za potocznym rozumieniem nazwy tego pojęcia. W tym rozdziale podamy 
jedną z nich. Powiemy też, co to znaczy, że dwa węzły są równoważne, tj. zastanowimy się, kiedy mając dane dwa węzły jesteśmy wstanie zdeformować jeden z nich (powiemy też, co to znaczy zdeformować), 
aby otrzymać drugi. Na koniec sformułujemy i udowodnimy twierdzenie Reidemeister'a, które dostarczy nam potężnych narzędzi do rozstrzygania, kiedy dane dwa węzły są równoważne.

\subsection{Sformuowanie definicji węzła}

Tutaj powiem, czemu jakieśtam podejście jest złe...

Trararam! Teraz podaję lepsze, od tego złego i mówię czemu jest lepsze...

\begin{definicja}
 Węzeł to łamana w $\mathbb{R}^3$ bez samoprzecięć.
\end{definicja}

\begin{definicja}
 Wierzchołek węzła...
\end{definicja}

\begin{definicja}
 Splot - rozłączna suma węzłów...
\end{definicja}

\subsection{Równoważność węzłów}
% Skoro już w sposób ścisły zdefiniowaliśmy pojęcie węzła, możemy zastanowić się, jak sformalizować operację przekształcenia węzła w inny węzeł. Oczywiście możemy potraktować węzeł, 
% jako pewien podzbiór w $\mathbb{R}^3$ i powiedzieć, że możemy określić na owym podzbiorze dowolną funkcję, ale to raczej nudne podejście. Chcielibyśmy wiedzieć, kiedy obarzem takiej
% funkcji będzie węzeł, i ''jak bardzo'' będzie się on różnił od węzła wyjściowego. 
Trochę gadki wstępnej dotyczącej potrzeby wprowadzenia relacji równoważności, tzn. kiedy węzły, które wyglądają trochę inaczej w istocie są tym samym węzłem.


\begin{definicja}
 Elementarnym przekształceniem węzła nazywamy...
\end{definicja}

\begin{definicja}
 Mówimy, że węzły $J$ i $K$ są równoważne, gdy...
\end{definicja}

Sprawdzenie, że podana relacja jest w istocie relacją równoważności pozostawiamy czytelnikowi jako ćwiczenie (dobrze?...)

Gadka-szmatka: utożsamiamy z sobą węzły równoważne, mówimy, że dwa węzły, to ten sam węzeł, gdy te dwa węzły są równoważne.

\subsection{Diagramy i rzuty węzła na płaszczyznę}
Od początku tej pracy, artykułu (? trzeba uzgodnić terminologię) z konieczności rysowaliśmy węzły (obiekty żyjące w przestrzeni trójwymiarowej) na płaszczyźnie. 
Niestety, masowa technologia nie pozwala jeszcze na  pisanie prac w formie hologramów, nad czym wszyscy czterej autorzy mniej lub bardziej ubolewają. Głównym celem
tego podrozdziału jest pokazanie, że jeśli dwa różne węzły są reprezentowane przez ten sam dwuwymiarowy rysunek, to są równoważne.

\begin{definicja}
 Rzutem na płaszczyznę nazywamy...
\end{definicja}

Rzuty na płaszczyznę różnych (nierównoważnych) węzłów mogą być równe. Chcemy jednak mówić o takich ``ładnych'' rzutach...

\begin{definicja}
 Rzutem regularnym nazywamy takie rzutowanie węzła, że...
\end{definicja}

Diagram to rzut + info o góra-dół.

\begin{definicja}
 Diagramem węzła nazywamy...
\end{definicja}

Trzeba wyraźnie powiedzieć, że różne węzły mogą mieć ten sam diagram (np. w diagramie nie ma info o tym, jak wysoko jeden łuczek przechodzi pod drugim). 
Okazuje się jednak, że jeśli dwa różne węzły mają ten sam diagram, to są one równoważne. Aby to udowodnić potrzebujemy zdefiniować kilka pojęć pomocniczych...


\begin{definicja}
 Mówimy, że dwa węzły $(p_i)$, $(q_j)$ są od siebie odległe o mniej, niż $t$, gdy mają tyle samo wierzchołków i gdy dla każdej pary wierzchołków zachodzi $d(p_k,q_k) < t$.
\end{definicja}

\begin{twierdzenie}
 Niech $K$ będzie węzłem o uporządkowanym zbiorze wierzchołków $(p_1, p_2, \ldots, p_n)$. Dla każdego $\epsilon > 0$ istnieje węzeł $K'$ o zbiorze wierzchołków $(q_1, q_2, \ldots, q_n)$, 
który jest odległy od węzła $K$ o nie więcej, niż $\epsilon$, oraz jego rzut na płaszczyznę $OXY$ jest regularny. 
\end{twierdzenie}

\begin{twierdzenie}
 Jeśli węzeł $K$ ma regularny rzut na płaszczyznę $OXY$, to istnieje taka $\delta>0$, że dla każdego węzła $K'$ odległego od $K$ o mniej, niż $\delta$ węzły $K, K'$ są równoważne,
 oraz $K'$ również ma rzut regularny. 
\end{twierdzenie}

\begin{twierdzenie}
 Jeśli dwa węzły $K$ oraz $J$ mają ten sam diagram, to są równoważne.
\end{twierdzenie}

Terminologia: łuki, skrzyżowania, overpass, underpass - chciałbym to sensownie przetłumaczyć, bo wygodnie by było mieć jednosłowne nazwy na te pojęcia.

ćwiczenie: liczba skrzyżowań $=$ liczba łuków.

\subsection{Orientacja węzła}
ble, ble, ble... Węzeł to taki uporządkowany zbiór skończony. Jak go cyklicznie spermutujemy, albo odwrócimy kolejność jego elementów, to otrzymamy ten sam węzeł. O ile rąbnięcie
naszego porządku cyklem wydaje się nie zaburzać porządku krawędzi naszego węzła, o tyle odwrócenie kolejności elementów, już tak...

\begin{definicja}
 Węzłem zorientowanym nazywamy...
\end{definicja}

Na zbiorze węzłów zorientowanych można położyć relację równoważności podobną do tej, którą zdefiniowaliśmy wcześniej. Musimy jeszcze żądać, żeby ciąg elementarnych operacji nie 
zmieniał oriantacji naszego wyjściowego węzła...

\begin{definicja}
 Zorientowane węzły nazywamy równoważnymi, gdy...
\end{definicja}

Gdyby równoważność w szerszym sensie implikowała równoważność węzłów zorientowanych, powyższa definicja byłaby nudna. Tak na szczęście nie jest, istnieje przykład dwóch węzłów
równoważnych, ale nie równoważnych w sensie orientacji.

\begin{definicja}
 Węzłem odwrotnie zorientowanym nazywamy węzeł zorientowany $(p_n, p_{n-1}, \ldots, p_1)$ powstały z węzła... i oznaczamy $K^r$.
\end{definicja}

\subsection{Ruchy Reidemeister'a}
W tym podrozdziale przedstawimy pierwsze poważne narzędzie, które w wielu przypadkach pozwoli nam rozstrzygnąć, czy dwa węzły są równoważne, czy też nie. 
Metody komblinatoryczne, ble, ble, ble...

Wcześniej pokazaliśmy, że problem równoważności węzłów można próbować rozstrzygać posługując się diagramami tych węzłów. Sformułujemy i udowodnimy twierdzenie wiążące...

\begin{definicja}
 Dwa diagramy uważa się za równoważne, gdy wykonując skończoną liczbę przekształceń zwanych ruchami Reidemeister'a, można
 z jednego diagramu otrzymać drugi
\end{definicja}


\begin{definicja}
 Ruchy Reidemeister'a - rysunki.
\end{definicja}

Trzy ruchy wraz z ich odwrotnościami.

\begin{twierdzenie}{(Reidemeister'a)}
Dwa węzły są równoważne, wtedy i tylko wtedy, gdy ich diagramy są równoważne.
\end{twierdzenie}

Trochę pozachwycam się tym twierdzeniem i prostotą dowodu, uzasadnię, czemu przyjąłem taką definicję i wystarczy...

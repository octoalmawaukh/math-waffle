Nad $o, p, q, r, s$ są kropki nawet wtedy, kiedy ich nie ma!

\wcht{P. afiniczna} \prawo{241} $\mathbb A$ (z $V$ nad $\mathfrak K$): dane jest $\mathbb A \times V \to \mathbb A$, ,,$+$'', że $\dot p + 0 = \dot p$ i $(\dot p + v)+u = \dot p + (v+u)$; każde $\dot p, \dot q \in \mathbb A$ mają dokładnie jeden $v \in V$, że $\dot p + v = \dot q$ (wektor ,,od $\dot p$ do $\dot q$'', $\overrightarrow{pq}$).
Wymiar: $\dim_{\mathfrak K} V$.
\wcht{Przesunięcia równoległe} ($t_v \colon \dot p \mapsto \dot p+v$) tworzą grupę $A^{\#}$, izo- z $V$.
Odwzorowanie $f \colon \mathbb A \to \mathbb A'$ (obie stowarzyszone z $V, V'$ nad $\mathfrak K$) jest \wcht{afiniczne}: istnieje liniowe $\mathcal F \colon V \to V'$ (\wcht{różniczka}), że $f(\dot p + v) = f(\dot p) + \mathcal F(v)$.
Każde dwie p. afiniczne tego samego wymiaru skończonego są izo-.
\wcht{Układ współrzędnych} w $(\mathbb A^n, V)$: para $(\dot o; e_1, \dots, e_n)$ (punkt w $\mathbb A$ i baza dla $V$).
\wcht{Współrzędne} $x_1, \dots, x_n$ punktu $\dot p$: współrzędne $\overrightarrow{op} = x_1 e_1 + \dots + x_n e_n$.
Jeśli $U \le V$, to $\Pi = \dot p + U$ jest podprzestrzenią o kierunku $U$; $\Pi \subseteq \mathbb A$ to podp. afiniczna $\Lra$ razem z dowolnymi dwoma punktami zawiera prostą przechodzącą przez nie.
Punkty $\dot p_0, \dots, \dot p_m$ są \wcht{afinicznie nz}, jeśli nie leżą w żadnej płaszczyźnie wymiaru $m$.
\emph{Powłoki afiniczne}.

Jeśli $\dot p_0, \dots, \dot p_m$ są punktami w $\mathbb A$, zaś skalary $\alpha_0, \dots, \alpha_m \in \mathfrak K$ sumują się do jedynki, to $\sum_{i=0}^m \alpha_i \dot p_i$ jest \wcht{kombinacją barycentryczną}; $\dot p + \sum_i \alpha_i (\dot p_i -\dot p)$.
Jeśli każdy punkt jest ,,jednoznaczną taką kombinacją'', to $\dot p_i$ są \wcht{barycentrycznym układem współrzędnych}.
Afiniczne trzymają kombinacje barycentryczne i są wyznaczone jednoznacznie przez obrazy $\dot p_i$.
\emph{Układy równań liniowych.} % Kostrikin 2, Rozdział 4, Paragraf 1, Sekcja 6: funkcje afiniczne i układy równań liniowych.
Jeśli $U', U''\le V$, to $\Pi' = \dot p + U'$, $\Pi'' = \dot q + U''$ są \wcht{równoległe}, gdy $U' \subset U''$ lub odwrotnie.
Każda podp. afiniczna $\Pi \subseteq \mathbb A$ z punktem $\dot q \in \mathbb A$ istnieje dokładnie jedna podp. równoległa do $\Pi$, tego samgo wymiaru, przez $\dot q$.
\wcht{Podp. skośne}: afiniczne, rozłączne, nierównoległe.

\wcht{Euklidesowa}: \prawo{2 / 42} afiniczna $(E, V)$, gdy na $V$ jest dodatnio określony iloczyn skalarny.
\wcht{Kąt} między prostymi przez $\dot p, \dot q$ i $\dot r, \dot s$: liczba $0 \le \varphi \le \pi/2$, której kosinus wynosi $|(\overrightarrow{pq } \mid\overrightarrow{rs})| / \|\overrightarrow{pq}\|\|\overrightarrow{rs}\| $.
%Układ współrzędnych $(\dot o, e_1, \dots, e_n)$ jest \wcht{prostokątny}: $(e_i)$ w $V$ jest ortonormalny.
Afiniczne p. euklidesowego równego wymiaru $< \infty$ są izo- (zachowanie odległości).
\wcht{Odcinek}: $\dot p \dot q = \{\dot p + \lambda \overrightarrow{pq} : 0 \le \lambda \le 1\}$.
Jeśli dla $\dot r, \dot s \in \Pi$ jest $(\overrightarrow{p q} \mid \overrightarrow{r s})$, to $\Pi_{\dot p, \dot q} \perp \Pi$ (\wcht{prostopadłość} do podprzestrzeni).
Z każdego punktu spoza spoza afinicznej podprzestrzeni można opuścić prostopadłą.
,,Najkrótsza'' ze wspólnych prostopadłych: odległość podprzestrzeni.

Automorfizmy \prawo{2 / 43} afiniczne $\mathbb A \to \mathbb A$ tworzą grupę; te trzymające pewien $\dot o\in \mathbb A$ to podgrupy izo- z $\operatorname{GL} V$.
Odwzorowanie $\mathbb E \to \mathbb E$ jest izometrią $\Lra$ afiniczne z liniową częścią, która jest operatorem ortogonalnym na $V$.
Izometrie płaszczyzny to translacje, obroty i symetrie z poślizgiem; przestrzeni ($\R^3$): ruchy śrubowe, symetrie płaszczyznowe i obroty z prostopadłym odbiciem.
Afiniczne to dokładnie te $\mathbb A \to \mathbb A$, które trzymają współliniowość.
,,Geometria'': zbiór niezmienników pewnej grupy przekształceń.

\prawo{244}
\wcht{Grupa Lorentza}: $L$, $O(1,3)$.
Oznaczenia: $V= \langle e_0, e_1, e_2, e_3\rangle$, $x = t e_0 + x_1 e_1 + x_2 e_2 + x_3e_3$, $\|x\|^2 = q(x) = t^2-x_1^2 - x_2^2 - x_3^2$.
Każdy operator liniowy $\R^4 \to \R^4$, że [$\det \Gamma_A = 1$, $\Gamma_A$ trzyma górną powłokę stożka $q(x) = 0$ i jest auto- formy $q(x)$] to \wcht{właściwe przekształcenie Lorentza}, tworzą grupę $L^+$.
Fakt: odwzorowanie $\Gamma \colon A \to \Gamma_A$ grup $SL_2(\C)$ w $L^+$ jest epi-; każdy element $L^+$ jest obrazem dokładnie dwóch macierzy różniących się znakiem.
\hfill \emph{Dużo brakuje!} $\uparrow$
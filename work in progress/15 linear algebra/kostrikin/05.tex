\wcht{L. zespolone}: \prawo{1/5X} $\C = \langle 1, i \rangle_\R$ ma dwa ciągłe automorfizmy, $a+bi \mapsto a \pm bi$ i iloczyn skalarny: $\langle z_1 \mid z_2 \rangle = \Re z_1 \overline z_2$.
\wcht{Dwustosunek} jest odporny na ,,homomorfie''.
Przemienny i łączny pierścień z jedynką bez dzielników zera, a jednocześnie p. liniowa wymiaru dwa nad $\R$ jest izo- z $\C$.
Jeśli $z_1 \neq z_4$, $z_2 \neq z_3$, to współokręgowość $\Lra$ rzeczywisty dwustosunek.
Wzór de Moivre'a.
\[
	[z_1, z_2, z_3, z_4] = \frac{z_1 - z_2}{z_1-z_4} : \frac{z_3-z_2}{z_3-z_4} \spk
	[r(\cos \phi + i \sin \phi)]^n = r^n(\cos n\phi + i \sin n \phi)
\]

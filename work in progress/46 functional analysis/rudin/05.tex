\wcht{Tw. Hellingera-Toeplitza} (\datum{1910}): \prawo{5.X} liniowy i symetryczny ($\langle Tx, y \rangle = \langle x, Ty \rangle$) operator $T$ na p. Hilberta jest ciągły.
\wcht{Tw. Grothendiecka}: jeśli $0 < p < \infty$, $\mu$ to p-stwo na $\Omega$, zaś $S \subseteq L^\infty(\mu)$ jest domkniętą podprzestrzenią $L^p(\mu)$, to ma skończony wymiar.
\wcht{Tw. Kakutaniego}: gdy $K \subseteq^k X$ jest niepusty i wypukły, a $X$ lokalnie wypukła, zaś $G$ to jednakowo ciągła grupa odwzorowań afinicznych $K \to K$; $G$ ma punkt stały w $K$: taki $p \in K$, że $Tp = p$ dla $T \in G$.
Istnieją przestrzenie bez dopełnienia: domknięte $M \le X$, dla których nie ma domkniętej $N \le X$, że $X = M + N$, $M \cap N = \{0\}$: w $L^1$ (całkowalne funkcje na okręgu) jest $H^1$ (funkcje o zerowych współczynnikah Fouriera dla $n < 0$).
Każda ciągła funkcja z niepustego i wypukłego $K \subseteq^k X$ w lokalnie wypukłej $X$ w siebie ma punkt stały (\wcht{tw. Schaudera-Tichonowa}).
%,,Twierdzenie Bishopa''.
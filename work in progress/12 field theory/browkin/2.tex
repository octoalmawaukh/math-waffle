Każdy \prawo{2.1} wielomian $f \in K[x]$ rozkłada się na czynniki liniowe w pewnym rozszerzeniu $K$.
\wcht{Ciało rozkładu}: $K$ rozszerzone o pierwiastki $f$ z $M$.
Jeśli $f$ jest nierozkładalny, to ciała rozkładu są $K$-izo-: istnieje $\varphi \colon K(a) \to K(a')$, że $\varphi(a) = a'$.
Każdy izo- ciał $\varphi \colon K_1 \to K_2$ z izo- pierścieni $\overline \varphi \colon K_1[x] \to K_2[x]$ można rozszerzyć do $L_1 \to L_2$ między ciałami rozkładu (jakiegoś $f_1 \in K_1[x]$ i $\overline \varphi(f_1)$), więc każde dwa ciała rozkładu wielomianu z $K[x]$ są $K$-izo-.

\wcht{Ciało algebraicznie domknięte}: ,,nie daje'' się rozszerzyć (skończenie lub algebraicznie) $\Lra$ nierozkładalne wielomiany są liniowe $\Lra$ każdy wielomian z $K[x]$ ma pierwiastek w $K$ $\Ra$ jest nieskończone.
Każde ciało można domknąć (jednoznacznie co do $K$-izo-) do $a(K)$.

Domknięcie \wcht{rozdzielcze} $s(K)$: zbiór elementów rozdzielczych $K$ (dla charakterystyki zero ,,$s = a$'').
\wcht{Radykalne} $i(K)$: o elementach z $a(K)$ radykalnych nad $K$ (separable, inseparable).
\wcht{Pierwiastnikowe} $r(K)$: o elementach $a$ z $a(K)$ z ciągiem ciał $K = K_0 \subseteq \ldots \subseteq K_r$ ($a \in K_r$, zaś $K_i$ to ciało rozkładu wielomianu $x^{n_i} - a_i \in K_{i-1}[x]$, chyba że $\operatorname{char} K = p$, wtedy $p \nmid n_i$ albo bierzemy ciało rozkładu $x^p - x - a_i$).
Element $a(K)$ wyraża się przez \wcht{pierwiastniki} stopnia co najwyżej $n$ nad $K$, gdy ciąg wyżej można dobrać tak, by $(K_i : K_{i - 1}) \le n$, tworzą zbiór $r_n(K)$.
Gdy $L \subseteq r_n(K)$ jest skończonym rozszerzeniem $K$, to $(L : K)$ nie ma pierwszych dzielników większych niż $n$.

Teoria Galois -- 2.2

Rozwiązalność -- 2.3

Konstrukcje geometryczne -- 2.4
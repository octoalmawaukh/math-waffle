\wcht{Norma} \prawo{4.1.1} w ciele $\mathfrak K$ to funkcja $\varphi \colon \mathfrak K \to \R$, że $\varphi(a) \ge 0$, $\varphi(0
) = 0$, $\varphi(1) = 1$, $\varphi(ab) = \varphi(a)\varphi(b)$ i istnieje $C$, że $\varphi(a+b) \le C (\varphi(a) + \varphi(b))$.
Mamy: $\varphi(2) \le 2 \Lra C =1$ oraz $\varphi(2) \le 1 \Lra \varphi(a+b) \le \max \{\varphi(a), \varphi(b)\}$; każda norma ma równoważną z $C = 1$.
Jeśli $\operatorname{char} \mathfrak K = p \neq 0$, to wszystkie normy na $\mathfrak K$ są niearchimedesowe ($\varphi(2) \le 1$).
Normy \prawo{4.1.2} na pierścieniu wyznaczają jednoznacznie normy na ciele ułamków.
Dla EPZ (dziedziny) $P$ z ciałem ułamków $\mathfrak K$ i nierozkładalnego $p \in P$ mamy normę podszywającą się pod $p$-adyczną, przykład: $\mathfrak K[x]$ z $\mathfrak K(x)$ i normy $\varphi_f$.
Normy \prawo{4.1.3} na $\mathfrak K(x)$, trywialne na $\mathfrak K$: trywialna, normy $\varphi_f$ ($f$: parami niestowarzyszone, nierozkładalne wielomiany z $\mathfrak K[X]$) i $\varphi_{1/x}$.
Norma na skończonym $\mathfrak K$ jest trywialna.
%\wcht{Tw. Ostrowskiego}.
Żadna norma na $\R(x)$ obcięta do $\R$ nie jest ,,$x \mapsto |x|$''.
Każda niearchimedesowa norma $\varphi$ w ciele $\mathfrak K$ jest obcięciem pewnej normy $\overline \varphi$ na $\mathfrak K(x)$.
Brakująca wcześniej definicja normy $\varphi_{1/x}$: $0 \mapsto 0$, $f/g \mapsto \exp (\operatorname{deg} f - \operatorname{deg} g)$.
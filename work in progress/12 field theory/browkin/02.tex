% Istnieją \prawo{2.1.1} rozszerzenia ciała $\mathfrak K$, w których $f \in \mathfrak K[x]$ (dodatniego stopnia) ma pierwiastek lub jest produktem wielomianów liniowych (takie nazywa się \wcht{ciałem rozkładu}).
% \wcht{Twierdzenie o rozszerzaniu izo-}: jeśli $\varphi \colon \mathfrak K_1 \to \mathfrak K_2$ jest izo- ciał, $\overline \varphi \colon \mathfrak K_1[x] \to \mathfrak K_2[x]$ izo- pierścieni wielomianów (odpowiadający $\varphi$), $L_1$/$L_2$: ciało rozkładu $f_1 \in \mathfrak K_1[x]$ / $f_2 = \overline \varphi (f_1) \in \mathfrak K_2[x]$, to $\varphi$ przedłuża się do izo- $\psi \colon L_1 \to L_2$ przez $\overline \varphi(x) = x$.

% NWSR: \prawo{2.1.2} jeśli $L$ jest rozszerzeniem algebraicznym (lub skończonym) ciała $\mathfrak K$, to $L = \mathfrak K$; nierozkładalne elementy $\mathfrak K[x]$ są liniowe, każdy wielomian dodatniego stopnia z $\mathfrak K[x]$ ma pierwiastek w $\mathfrak K$.
% Zatem algebraicznie domknięte $\Ra$ nieskończone.
% \wcht{Domknięcie algebraiczne} $\mathfrak K$: minimalne ,,nadciało'' algebraicznie domknięte, $a(\mathfrak K)$.
% Rozszerzenie algebraiczne ciała $\mathfrak K$ można $\mathfrak K$-zanurzyć w $a(\mathfrak K)$ (i przedłużyć do $\mathfrak K$-auto- $\psi \colon a(\mathfrak K) \to a(\mathfrak K)$).
% Każde dwa domknięcia algebraiczne $\mathfrak K$ są $\mathfrak K$-izo-.

% \wcht{Domknięcie rozdzielcze} (\wcht{radykalne}: $\operatorname{char} \mathfrak K = p \neq 0$) $s(\mathfrak K) \subseteq a(\mathfrak K$ / $i(\mathfrak K)$: \prawo{2.1.3} zbiór elementów rozdzielczych (radykalnych) względem $\mathfrak K$, największe rozszerzenie rozdzielcze (radykalne) w $a(\mathfrak K)$.
% Jeżeli $\varphi \colon a(\mathfrak K) \to a(\mathfrak K)$ jest $\mathfrak K$-auto-, to $\varphi(s(\mathfrak K)) = s(\mathfrak K)$, $\varphi$ jest tożsamością na $i(\mathfrak K)$.
% Element $a \in a(\mathfrak K)$ \wcht{wyraża się przez pierwiastniki} względem $\mathfrak K$, jeżeli istnieje ciąg ciał $\mathfrak K = K_0 \subseteq \dots \subseteq \mathfrak K_r$, że $a \in \mathfrak K_r$, $\mathfrak K_i$ jest ciałem rozkładu wielomianu postaci $x^{n_i} - a_i \in \mathfrak K_{i-1}[x]$; tworzą \wcht{domknięcie pierwiastnikowe} $r(\mathfrak K)$ [dla $\operatorname{char} \mathfrak K = 0$, jeśli nie to trzeba zmienić drugi warunek na ,,\dots lub $p \nmid n_i$ albo wielomianu postaci $x^p - x- a_i \in \mathfrak K_{i-1}[x]$''].
% Uwaga: można przyjąć, że $n_i$ są pierwsze i $(\mathfrak K_i : \mathfrak K_{i-1}) \le n_i$.

% Element $a \in a(\mathfrak K)$ \wcht{wyraża się przez pierwiastniki stopni co najwyżej $n$} względem $\mathfrak K$, jeśli istnieje ciąg ciał $\mathfrak K = \mathfrak K_0 \subseteq \dots \subseteq \mathfrak K_r$, że $a \in \mathfrak K_r$, $(\mathfrak K_i : \mathfrak K_{i-1}) \le n$ i znów spełniony jest drugi warunek (z poprawką dla $\operatorname{char} \mathfrak K = p \neq 0$); tworzą zbiór $r_n(\mathfrak K)$.
% Jeśli skończone rozszerzenie $L$ ciała $\mathfrak K$ zawiera się w $r_n(\mathfrak K)$, to $(L:K)$ nie ma pierwszych dzielników $ > n$. (dla $n = 2$ ciekawe!).
% Fakty: $r(\mathfrak K) \subseteq s(\mathfrak K)$ to ciała, każdy element pierwiastnikowy względem $r(\mathfrak K)$ jest w $r(\mathfrak K)$, obraz $r(\mathfrak K)$ przez $\mathfrak K$-izo- $a(\mathfrak K) \to a(\mathfrak K)$ to $r(\mathfrak K)$ (podobnie (jak?) dla $r_n$ zamiast $r$).
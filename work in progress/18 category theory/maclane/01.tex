Obiekty \prawo{1.5} $a, b$ \wcht{izomorficzne}: strzałka $e \colon a \to b$ jest \wcht{odwracalna} w $C$  (istnieje $e' \colon b \to a$ w $C$, że $e'e = 1_a$, $ee' = 1_b$, ,,$e^{-1}$''). 
Strzałka $m \colon a \to b$ \wcht{mono-} w $C$: dla równoległych $f_1, f_2 \colon d \to a$ równość $m \circ f_1 = m \circ f_2$ pociąga $f_1 = f_2$.
W \textsc{Set} i \textsc{Grp} mono- to injekcje.
Strzałka $h \colon a \to b$ \wcht{epi-} w $C$: dla strzałek $g_1, g_2 \colon b \to c$ równość $g_1 \circ h = g_2 \circ h$ pociąga $g_1 = g_2$.
W \textsc{Set} są to surjekcje.
\wcht{Prawa odwrotność} dla $h$: $r \colon b \to a$, że $hr = 1_b$, sekcja.
\wcht{Lewa odwrotność}: analogicznie, retrakcja.
Strzałki z sekcjami $\Ra$ epi-, $\Leftarrow$ dla \textsc{Set}, ale nie \textsc{Grp}.
Strzałki z retrakcjami są mono-.
Jeżeli $gh=1_a$, to $g$ jest \wcht{rozdartym epi-}, $h$ rozdartym mono, zaś $f=hg$ jest idempotentna.
Do obiektu \wcht{terminalnego} (z \wcht{inicjalnego}) prowadzi po jednej strzałce z (do) każdego.
\wcht{Zerowy}: taki i taki.
Epi- mono- może się nie odwracać!
\wcht{Grupoid}: kategoria bez nieodwracalnych strzałek.
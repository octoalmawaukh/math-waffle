%Funkcja \prawo{6.1} $f \colon (U \subseteq_o \C) \to \C$ jest \wcht{holomorfem}, jeśli jest $\C$-różniczkowalna.
%Jeżeli $z(x, y) = x + \textrm{i}y$, to $\D{f}(a) = f'(a) \,\textrm{d}z$.

% 6.2
% 6.3

Jeżeli \prawo{6.4} $f$ jest holomorfem w $K_r^*(a)$, zaś $\kappa$ dowolnym koncentrycznym, dodatnio określonym konturem w $K_r^*(a)$, to $\textrm{Res}_a f$ jest \wcht{residuum}.
To współczynnik przy $(z-a)^{-1}$ w rozwinięciu Laurenta dla $f$; jednoznacznie określona $R \in \C$, że $f(z) - R / (z-a)$ ma w $K_r^*(a)$ pierwotną.
\wcht{Tw. o residuach}: dla $U \subseteq_o \C$, $S \subseteq U$ bez punktu skupienia w $U$ i holomorfa $f$ w $U \setminus S$ oraz $\gamma$, zero-homotopijnej zamkniętej drogi całkowania w $U$, która nie dotyka $S$ istnieje skończenie wiele $a \in S$, że $n(\gamma, a) \neq 0$ i khm-2.
\[
	\textrm{Res}_a f = \frac 1 {2\pi i} \int_{\kappa} f(z) \,\textrm{d}z \spk
	\int_\gamma f(z) \,\textrm{d}z = 2\pi i \sum_{a \in S} \textrm{Res}_a f \cdot n(\gamma, a)
\]
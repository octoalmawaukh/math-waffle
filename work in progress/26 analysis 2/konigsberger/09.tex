\wcht{Tw. o transformacji}: \prawo{9.1} funkcja $f$ na $V \subseteq_o \R^n$ jest całkowalna nad nim $\Lra$ nad $U \subseteq_o \R^n$ całkowalna jest $(f \circ T) \cdot |{\det T'}|$, gdzie $T$ to dyfeo- $U \to V$ (wtedy całki są równe).
Patrz na afiniczne dyfeomorfizmy.
Dla całkowalnej i ograniczonej funkcji $\rho$ na skorupie $K \subseteq^k \R^3$ o środku w $0$ i każdego $x \in \R^3$, $u(x)$ istnieje: $\rho$ obrotowo symetryczna $\Ra$ $u$ też.
\emph{Odwzorowania potęgowe i uogólnione sympleksy}.
\[
	u(x) = \int_K \frac{\rho(y) \, \D y}{\|x - y\|_2} \textrm{ (potencjał Newtona)}
\]




% % \marginpar{\rotatebox[origin=l]{90}{$\uparrow$ 9-1}}
% % Skorupa $K(I ) = \{x \in \R^n : \|x \| \in I\}$.
% % Gdy $I = (?, ?)$, to $P_n$ jest dyfeo- produktu $I \times \Pi$ na ,,rozciętą skorupę'' $K^*(I) = K(I) \setminus (S \times \R^{n-2})$, tu $(-\pi, \pi) \times (-\pi/2, \pi/2)^{n-2}$ i $S = \{(x_1, 0) : x_1 \le 0\}$, $C(\varphi_1, \dots, \varphi_{n-1}) = \prod_{k=1}^{n-1} \cos^{k-1}\varphi_k$
% % Funkcja $f$ na skorupie $K(I) \subseteq \R^n$ jest na niej całkowalna $\Lra$ funkcja $f(P_n(r, \varphi)) \cdot C(\varphi) r^{n-1}$ jest całkowalna na $I \times \Pi$. Wtedy równość.
% % \marginpar{\rotatebox[origin=l]{90}{$\uparrow$ 9-3-1}}
% % \wcht{Całkowanie obrotosymetrycznych funkcji}: $f$ funkcją na przedziale $I$; funkcja $x \mapsto f(\| x\|)$ jest całkowalna na skorupie $K(I) \subset \R^n$ $\Lra$ funkcja $r \mapsto f(r) r^{n-1}$ jest całkowalna na $I$, wtedy inna równość ($\kappa_n$ objętością kuli jednostkowej).
% % \hfill {\small $P_2(r, \varphi) = (r \cos \varphi, r \sin \varphi)$, $P_{n+1} (r, \varphi_1, \dots, \varphi_n) = (P_n(r, \varphi_1, \dots, \varphi_{n-1})\cdot \cos \varphi_n, r \sin \varphi_n )$}
% % \[
% % 	\int_{K(I)} f(x) \,\D x = \int_I \int_\Pi  f(P_n(r, \varphi)) \cdot C(\varphi) r^{n-1}\,\D \varphi \,\D r \spk
% % 	\int_{K(I)} f(\| x\|) \,\D x = n \kappa_n \int_I f(r) r^{n-1} \,\D r
% % \]

% % Niech $J \colon \R^2 \to \R^2$ będzie odwzorowaniem $J(u,v) = (u(1-v), uv) = (x,y)$.
% % \marginpar{\rotatebox[origin=l]{90}{$\uparrow$ 9-3}}
% % Wtedy w $\R_+^2$ jest $J^{-1}(x,y) = (x+y,y/(x+y))$; $J$ jest dyfeo- $S=R_+ \times (0,1)$ na $R_+^2$ i $W=(0,1)^2$ na $\interior (\Delta^2)$; $\det J'(u, v) = u$.
% % Tw. o transformacji daje: funkcja $f$ na $\R_+^2$ ($\interior (\Delta ^2)$) jest całkowalna $\Lra$ $(f \circ J) \cdot u$ jest całkowalna na $S$ (na $W$), wtedy khm.
% % \[
% % 	\int_{\R_+^2 / \Delta^2} f(x,y) \,\D{(x+y)} = \int_{S / W} f(u(1-v), uv) \cdot u \,\D{(u,v)}
% % \]

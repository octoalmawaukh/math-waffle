\wcht{Homotopia} ścieżek (ciągłych $I \to X$): $F \colon I \times I \to X$, że $F(s,t) = f_t(s)$ są ciągłe, $f_t(0) = x_0$, $f_t(1) = x_1$.
Wyznacza relację równoważności.
Złożenie: $(f \cdot g)(s) = f(2s)$ lub $g(2s - 1)$.
\wcht{Grupa podstawowa}: $\pi_1(X, x_0)$, rodzina klas homotopii $[f]$ \wcht{pętli} zaczepionych w $x_0$ ze składaniem.
Nie zależy od $x_0$, trywialna $\Lra$ przestrzeń jednospójna.
$\pi_1(S^1) = \Z$.
\wcht{P. nakryciowa} dla $X$: p. $X'$ z mapą $p \colon X' \to X$, że każdy $x \in X$ ma otoczenie $U$ w $X$, którego przeciwobraz jest unią rozłącznych, otwartych, homeo- przenoszonych na $U$.
Dla łukowo spójnych $X, Y$: $\pi_1(X\times Y)$ jest izo z $\pi_1(X) \times \pi_1(Y)$.
$\pi_1(S^n) = 0$ dla $n \ge 2$.


Jeśli $X$ jest unią łukowo spójnych, otwartych $A_\alpha$ (każdy zawierający punkt bazowy $x_0 \in X$), zaś każdy $A_\alpha \cap A_\beta$ też jest łukowo spójny, to homo- $\Phi: *_\alpha \pi_1(A_\alpha) \to \pi_1(X)$ jest ,,na''.
Jeśli $A_\alpha \cap A_\beta \cap A_\gamma$ też są łukowo spójne, to jądro $N = \ker \Phi$
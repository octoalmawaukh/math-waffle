\wcht{Spójna} \prawo{6.1} $X$: nie jest nietrywialną sumą ($\oplus$) otwartych $\Lra$ tylko $\varnothing, X$ są otwarnięte $\Lra$ ciągłe $X \to \mathfrak D_2$ są stałe.
Obraz spójnej jest spójny. % 6.1.3
$\mathcal T_{3.5}$, spójna, z dwoma punktami $\Ra$ mocy $\ge \mathfrak c$. % 6.1.4
Jeśli $C_s \subseteq X$ są spójne i istnieje $C_{s_0}$, który nie jest rozgraniczony z żadnym z $C_s$, to unia $\bigcup_s C_s$ jest spójna (Kuratowski, Knaster \datum{1921}). % 6.1.9
Tak jest np. gdy $\bigcap_s C_s \neq \varnothing$.
Jeśli $C$ jest spójny, to każdy $A$, że $C \subseteq A \subseteq \operatorname{cl} A$, też. 
Rozszerzenie $\beta X$ jest spójne $\Lra$ $X : \mathcal T_{3.5}$ jest spójna.
Niepusty $\prod_s X_s$ jest spójny $\Lra$ każda $X_s$ taka jest.
P. ilorazowe spójnych są spójne, ale ,,GCO'' niekoniecznie.

\wcht{Continuum}: zwarta, spójna.
Przekrój continuów skierowanych przez $\supseteq$ (np. dla zstępującego ciągu) jest continuum.
Eilenberg, Steenrod (\datum{1952}): ,,GSO'' continuów jest continuum. 
$(S^{n-1})^X$ ze zwarotwologią dla $X \subseteq \R^n$ jest spójna $\Lra$ $S^n \setminus X$ jest spójna; więc: $X_1, X_2 \subseteq \R^n$ są 
homeo- $\Ra$ $S^n \setminus X_1$ jest tak samo spójne jak $S^n \setminus X_2$ (uogólnienie tw. Jordana).
\wcht{Składowa}: unia spójnych otoczeń, domknięta.
Składową dla $\{x_s\}$ w $\prod_s X_s$ jest $\prod_s S_s$.
\wcht{Quasi-składowa}: przekrój otwarniętych otoczeń, domknięty.
Mamy $S_x \subseteq Q_x$, dla zwartych $X$ równość (Szura-Bura, \datum{1941}).
\wcht{Tw. Sierpińskiego} (\datum{1918}): jeśli continuum $X$ ma pokrycie $\{X_n \subseteq^a X\}_{n \ge 1}$ rozłącznymi, to tylko jeden z nich jest niepusty.

Funkcja \wcht{monotoniczna}: ciągła o spójnych włóknach (Whyburn, \datum{1934}).
Monotoniczne $\R \to \R$ jest niemalejące lub nierosnące.
Jeśli $X \to Y$ jest monotoniczne, ilorazowe, to przeciwobrazy spójnych (otwartych lub domkniętych) są spójne.
Monotoniczne surjekcja jest domknięta lub otwarta $\Ra$ przeciwobrazy spójnych są spójne.
Dla spójnych: $\mathfrak L$ $\Lra$ mocna parazwartość.
Dla spójnej metrycznej: każde dwa punkty można połączyć ciągiem $x_0, \dots, x_k$, że $d(x_i, x_{i+1}) < \varepsilon$; metryczna zwarta z tym warunkiem jest spójna (Cantor, \datum{1883}).



\wcht{Mocno zerowymiarowa}: \prawo{6.2} $X \neq \varnothing$, $\mathcal T_{3.5}$ i w funkwarte pokrycie $\{U_i\}_{i=1}^k$ można wpisać otwarte $\{V_i\}_{i=1}^m$ rozłącznych $\Ra$ \wcht{zerowymiarowa} (Sierpiński, \datum{1921}): $X\neq \varnothing$, $\mathcal T_1$ i ma otwarniętą bazę (jest $ \mathcal T_{3.5}$) $\Ra$ \wcht{,,dziedzicznie'' niespójna} (Hausdorff, \datum{1914}): składową $x$ jest $\{x\}$ (pociąga $\mathcal T_1$ i \wcht{punktokształtność}: continua są jednopunktowe) $\Leftarrow$ \wcht{ekstremalnie niespójna} (Stone, \datum{1937}): $\mathcal T_2$, domknięcia otwartych są otwarte.

Dla Lindelöfa: zero- $\Lra$ mocno-.
{\color{red}Dla $\mathcal T_3$: przeliczalna / ekstremalnie- $\Ra$ mocno-.}
Dla $\mathcal T_{3.5}$: mocno- $\Lra$ funkcyjnie oddzielalne $A, B$ mają otwarnięty $U$, że $A \subseteq U \subseteq X \setminus B$ $\Lra$ $\beta X$ jest mocno-.
Niepusta, ,,dziedzicznie''-, lokalnie zwarta $\Ra$ zero-. %6210 6211
Dla niepustych lokalnie zwartych, parazwartych: ,,dziedzicznie''- $\Lra$ zero- $\Lra$ mocno-.
Dla $\mathcal T_{3.5}$: $X, \beta X$ są ,,tak samo'' ekstremalnie-, mocno-.
Gdy $S, X_s$ są niepuste, to $\bigoplus_s X_s$ jest ,,dziedzicznie''- (zero-, mocno-) $\Lra$ wszystkie $X_s$ takie są; dla $\prod_s X_s$: to samo, ale nie mocno-, więc ,,GSO'' dla ,,dziedzicznie''- lub (zero-) jest taka (taka lub pusta).
Kostka Cantora $\mathfrak D_2^{\mathfrak m}$ jest uniwersalną dla zero- o ciężarze $\mathfrak m \ge \aleph_0$. %, więc 

 %Erdös: ,,dziedzicznie'' niespójna, metryczna i ośrodkowa, nie zero-; Dowker: zero-, $\mathcal T_4$, ale nie mocno-.

Brouwer (\datum{1910}): zero-, w sobie gęsta, metryczna, zwarta jest homeo- ze zbiorem Cantora.
Mazurkiewicz (\datum{1917}): gęsty i brzegowy zbiór $G_\delta$ w metrycznej ośrodkowej, zupełnej i zerowymiarowej jest homeo- z $\R \setminus \Q$.
Sierpiński (\datum{1920}): przeliczalna, metryzowalna, w sobie gęsta jest homeo- z $\Q$.
Aleksandrow, Urysohn (\datum{1928}): zerowymiarowa, metryczna ośrodkowa i metryzowalna zupełnie, gdzie tylko pusty jest otwarty i zwarty jest homeo- z $\R \setminus \Q$.
Gillman, Jerison (\datum{1960}): zwarta $X$ jest ekstremalnie- $\Lra$ $X = \beta Y$ dla gęstych podprzestrzeni $Y$ w $X$.

Pol (\datum{1974}): otwarta surjekcja z dziedzicznie niespójnej metrycznej ośrodkowej na niejednopunktową spójną.
Dziedziczna-, zero- i mocna- nie są ani doskonałych, ani otwartych, ale są otwarniętych.
Ekstremalna nie jest niezmiennikiem doskonałych, ale jest otwartych.
Ekstremalna $\mathcal T_3$ jest mocno-.
Dla $\mathcal T_2$, ekstremalnie- $\Lra$ rozłączne otwarte mają rozłączne domknięcia.
Suma $\bigoplus_s X_s$ jest ekstremalnie- $\Lra$ $X_s$ są.
,,GSO'' czy produkt ekstremalnie- nie musi taka być.

Funkcja $f \colon X \to Y$ jest \wcht{lekka} ($\Leftarrow$ \wcht{zerowymiarowa}; a dla zwartych włókien: $\Lra$) (Stoiłow, \datum{1928}), gdy jej włókna są dziedzicznie niespójne (puste lub zero-).
Każde doskonałe $X \to Y$ to jednoznaczne złożenie $X \to Z$ (doskonałe, monotoniczne, ,,na'') i $Z \to Y$ (doskonałe, zero-).

\wcht{Lokalna spójność} (dla $x \in X$, $U_x$ istnieje spójny $C \subseteq U_x$, że $x \in \interior C$ $\Lra$ otwarte składowe otwartych podprzestrzeni $\Ra$ quasiskładowe to składowe) jest niezmiennikiem ilorazowych (Whyburn, \datum{1952}).
\prawo{6.3}
$\bigoplus_s X_s$ jest lokalnie spójna $\Lra$ $X_s$ są, dla niepustych produktów: też, ale tylko skończenie wiele może być niespójnych.
Uzwarcenie $\beta X$ jest lokalnie spójne $\Ra$ $X$ pseudozwarta (Banaschewski, \datum{1956}).
Ośrodkowe continuum, którego nie \wcht{rozcinają} (domknięty $\{x\}$, że $X \setminus \{x\}$ niespójna) dokładnie dwa punkty jest homeo- z odcinkiem.
Każde continuum $X$, że $|X| > 1$, ma co najmniej dwa takie punkty (Wallace, \datum{1942}).
\wcht{Lokalnie łukowa spójność}: dla każdego $x$ oraz $U_x$ istnieje $V_x$, że dla każdego $y \in V$ istnieje ciągłe $f \colon I \to U$, że $f(0) = x$, $f(1) = y$; pociąga lokalną spójność (a ze spójnością także \wcht{łukową spójność}, która jest niezmiennikiem ciągłych i pociąga spójność) i jest niezmiennikiem ciągłych; dziedziczy się na otwarte.
Ośrodkowe continuum oraz linup jest homeo- z domkniętym odcinkiem.
Herrlich (\datum{1965}): ,,dziedzicznie''- linup $\Ra$ mocno-.
Metryzowalny ośrodkowy linup zanurza się w $\R$.
%\wcht{Przestrzenie projektywne, absoluty i 6.3.21}: 436.
Skończona, \prawo{6.1} abelowa jest sumą prostą \wcht{$p$-prymarnych} ($p$-grup abelowych) podgrup, \wcht{komponent} $\{x \in G : \exists_k p^k x = 0\}$.
\wcht{Niezależne} $x_1$, $\ldots$, $x_r \neq 0$ (w abelowej): $\sum_i m_i x_i = 0$ pociąga $m_i x_i = 0$ $\Lra$ $\langle x_1, \ldots, x_r \rangle = \langle x_1 \rangle \oplus \ldots \oplus \langle x_r \rangle$.
\wcht{Tw. o bazie} (Schering \datum{1868}, Kronecker \datum{1870}): każda skończona, abelowa jest sumą prostą cyklicznych (prymarnych).
Grupa jedności $\Z/p^n \Z$ jest cykliczna, rzędu $p^n - p^{n-1}$ (dla $p \neq 2$).

W \prawo{6.2} rozkładzie $p$-prymarnej $G$ na sumę prostą cyklicznych składników rzędu $p^{n+1}$ jest $d(p^nG / p^{n+1}G) - d(p^{n+1}G / p^{n+2}G)$, $U_p(n, G)$, gdzie $d$ to minimalna liczba generatorów, $mG = \{mg : g \in G\}$.
\wcht{Fundamentalne tw.} (Frobenius, Stickelberger \datum{1879}): $G, H$ są izo- $\Lra$ mają te same \wcht{dzielniki elementarne} ($U_p(n, G)$ razy liczone liczby $p^{n+1}$) dla $p \ge 2$.
Jeżeli $A \oplus C \cong B \oplus C$ lub $A \oplus A \cong B \oplus B$, to $A \oplus B$.
Jeśli $H \le G$, to $G$ zawiera izo-kopię $G/H$.
W sekcji 6.2 wszystkie grupy były skończone, abelowe.

Czy \prawo{6.3} skończenie generowana grupa $G$ o wykładniku $e$ jest skończona (\wcht{problem Burnside'a}, \datum{1902})?
Burnside: jeśli $G \le GL (n, \C)$, to tak.
Kontrprzykłady: Adian, Novikov (\datum{1968}, wzmocniony w \datum{1975}), Ivanov (\datum{1994}): istnieje $\infty$-wiele parami nieizo- grup skończenie generowanych o wykładniku $2^k m$ dla $k \ge 48$ i nieparzystej $m \ge 1$.
\emph{Dużo nudnej jak flaki z olejem algebry liniowej} (jak rozkład Jordana). %6.3 6.4

\wcht{Endo- normalny}: \prawo{6.5} $\varphi(x^a) = \varphi(x)^a$.
Nietrywialna grupa $G$ jest \wcht{nierozkładalna}: jeśli $G = H \times K$, to $H$ lub $K$ jest trywialna.
\wcht{ACC}/\wcht{DCC}: każdy wstępujący (zstępujący) ciąg podgrup normalnych zatrzymuje się.
Skończone grupy spełniają A+D, $\Z$ tylko A, $\Z(p^\infty)$ tylko D, $\Q$ zaś żadnego.
Jeśli produkt dwóch grup ($H \trk G$ i $G/H$) spełnia A+D, to czynniki ($G$) również.
Spełnianie któregokolwiek warunku pociąga bycie produktem prostym skończenie wielu nierozkładalnych grup.
\wcht{Endo- nilpotentny} $\varphi$: istnieje $k \ge 1$, że $\varphi^k = 0$.
\wcht{Lemat Fittinga} (\datum{1934}): jeśli grupa $G$ spełnia A+D i posiada normalny endo- $\varphi$, to $G = K \times H$, gdzie $\varphi(K) \le K$, $\varphi(H) \le H$, $\varphi \mid_K$ jest nilpotentny, zaś $\varphi \mid_H$: ,,na''.
\wcht{Tw. Krulla-Schmidta}: grupa $G$ spełnia A+D.
Jeśli $G = \prod_{i \le s} H_i = \prod_{i \le t} K_i$, $H$ i $K$ są nierozkładalne, to $s = t$.
Dla każdego $r$ pomiędzy $1$ i $s$ można tak przeindeksować czynniki, że $G = H_1 \times \ldots \times H_r \times K_{r+1} \times \ldots \times K_s$ (coś więcej niż jednoznaczność czynników z dokładnością do izo-).
Nierozkładalne są $\Q$, $\Z$, $\Z/p^n\Z$, $S_n$, $D_n$, $Q_n$, proste, nieabelowe rzędu $p^3$, $A_4$ czy dziwna rzędu 12.
Niech $G$ spełnia A+D.
Jeśli $G \times G \cong H \times H$, to $G \cong H$.
Nie istnieje właściwa podgrupa $H < G$, że $G \cong H$ lub $G/H \cong G$.
Jeśli $G \cong A \times B \cong A \times C$, to $B \cong C$ (to działa też dla skończonych $A$ oraz dowolnych $B, C$: Hirshon \datum{1969})
(6.6): Grupy operatorów pozwalają na uogólnienie tw. Jordana-Höldera, lematu Fittinga i tw. Krulla-Schmidta. % 6.6
Grupa, w \prawo{4.1} której każdy element ma rząd postaci $p^i$: \wcht{$p$-grupa}.
\wcht{Tw. Cauchy'ego}: jeśli $p$ dzieli $|G|$, to istnieje $g \in G$ rzędu $p$.
Weźmy skończoną $p$-grupę $G \neq \{e\}$.
Właściwe $H \le G$ spełniają $H \le \mathcal N_G(H)$;
$\mathcal Z(G) \neq \{e\}$.
Maksymalna podgrupa $G$ jest normalna oraz indeksu $p$.
Podgrup rzędu $p^s$ jest $r_s \in 1 + p\Z$.
Skończenie wiele skończonych grup ma $n$ klas sprzężoności.
\wcht{Równanie klasowe}: $|G| = \mathcal Z(G) + \sum_i [G : \mathcal C_G(x_i)]$.
Nieabelowa grupa $G$ rzędu $p^3$ ma $p$-elementowe centrum $\mathcal Z(G) = G'$, $G/\mathcal Z(G) \cong (\Z/p\Z)^2$.

\wcht{Sylowa, $p$-podgrupa}: \prawo{4.2} maksymalna $p$-podgrupa $P \le G$.
Jeśli $|G| = p^km$ i $p \nmid m$, to $|P| = p^k$.
\wcht{Tw. Sylowa} (\datum{1872}): $p$-podgrupy Sylowa są sprzężone dla $|G| < \infty$; ich liczba dzieli $|G|$ i przystaje do $1$ mod $p$.
% jest ich $r$ ($r \mid |G|$ $r \in 1 + p\Z$).
Skończona $G$ ma (dla jakiegoś $p$) jedyną $p$-podgrupę Sylowa $P \Lra P \trk G$.
\wcht{Argument Frattiniego}: jeśli $G$ jest skończona, zaś $P$ to $p$-podgrupa Sylowa $K \trk G$, to $G = K\mathcal N_G(P)$.

Grupa \prawo{4.3} rzędu $2p$ jest cykliczna lub diedralna, rzędu $pq$ ($p > q$ pierwsze): cykliczna lub $\langle a,b \rangle$, gdzie $b^p = a^q = 1$, $ab=b^ma$ i $m^q = 1$ mod $p$, ale $m \neq 1$ mod $p$.
Drugi przypadek nie zachodzi dla $q \nmid p - 1$.
\wcht{Kwaterniony} ($\langle a, b \rangle$ dla $a^4 = 1$, $b^2 = a^2$ i $aba = b$) i diedralna to jedyne nieprzemienne rzędu $8$.
\wcht{Diedralna}: $\langle a , b \mid a^2 = b^n = abab = 1 \rangle$.
Dla nieprzemiennych, skończonych $G$: każda podgrupa jest normalna $\Lra$ $G$ jest postaci $Q \times A \times B$ ($A, B$: przemienne, $A$ ma wykładnik $2$, $B$ nieparzysty rząd) (Dedekind \datum{1897} dla skończonych, Baer \datum{1933}).
\wcht{Uogólnione kwaterniony}, grupa dwucykliczna: $Q_n$ rzędu $4q = 2^n$ dla $n \ge 3$ generowana przez $a$ i $b$, że $a^{2q} = 1$, $aba = b$ i $a^q = b^2$.
Można pokazać ich istnienie przez zanurzenie w grupę $GL(2, \C)$: $A = [[0, \omega], [\omega, 0]]$, $B = [[0, 1], [-1, 0]]$, $\omega$ to $2^{n-1}$-szy pierwiastek jedności, $n \ge 3$.
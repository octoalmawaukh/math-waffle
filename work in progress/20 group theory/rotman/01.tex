Każda \prawo{1.X} \wcht{permutacja} (bijekcja $X \to X$) rozkłada się na \wcht{transpozycje} (2-cykle), gdy $|X| < \infty$.
\wcht{Półgrupa}: zbiór z łącznym \wcht{działaniem} ($X \times X \to X$).
\wcht{Grupa}: półgrupa z elementem neutralnym i odwrotnymi, może być \wcht{abelowa} (przemienna).
\wcht{Homomorfizmy}: \wcht{mono-} (1-1), \wcht{epi-} (,,na''), \wcht{endo-} ($G \to G$), \wcht{izo-} (odwracalny: mono- i epi-), \wcht{auto-} (izo-, endo-), odwzorowania $f$ między grupami, że $f(ab) = f(a)f(b)$ (zachowanie działania), przykład: $\operatorname{sgn} \colon S_n \to \{\pm 1\}$, który przypisuje $1$ dokładnie produktom parzyście wielu 2-cykli.
\wcht{Rząd} grupy: jej moc. 
Szok: $(\Z[x], +) \cong (\Q_+, \times)$.

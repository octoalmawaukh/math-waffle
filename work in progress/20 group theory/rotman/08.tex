\wcht{Grupa liniowa}: \prawo{8.2} $GL(F_q^m)$ to zbiór niesingularnych macierzy $m \times m$ nad ciałem $F_q$, z podgrupą specjalną ($SL$) macierzy unimodularnych ($\det A = 1$).
Mamy $GL(m, q) = SL(m, q) \rtimes F_q^\times$.
Centrum $GL$ to macierze postaci $\alpha E$ dla $\alpha \neq 0$, zaś $SL$: dla $\alpha^m = 1$ (tych drugich jest dokładnie $(m, q-1)$).
Sama grupa $GL$ ma $\prod_{i=0}^{m-1} (q^m - q^i)$ elementów.
Iloraz $SL$ przez centrum to \wcht{rzutowa grupa unimodularna} $PSL$.

Grupy $PSL(2, q)$ są \prawo{8.34} proste $\Lra q > 3$ (\wcht{tw. Jordana-Moore'a}), zaś $SL(2, 5)$ nie jest rozwiązalna.
Jeżeli ciało $K$ jest (charakterystyki $2$, każdy element ma pierwiastek kwadratowy) lub (nieskończone, charakterystyki $\neq 2$), to $PSL(2,K)$ też jest prosta.
Znane są wszystkie izomorfizmy grup $PSL$ z alternującymi ($A$):
$PSL_27 \cong PSL_32$, 
$PSL_24 \cong PSL_25 \cong A_5$,
$PSL_29 \cong A_6$,
oraz $PSL_42\cong A_8$, dodatkowo $PSL_23 \cong A_4$.
Komutant $GL$ to $SL$, chyba że $V = F_2^2$: wtedy $GL = SL \cong PSL_22 \cong S_3$, zaś $S_3' = A_3$.
\wcht{Tw. Jordana-Dicksona}: jeśli $V$ to $m$-wymiarowa przestrzeń wektorowa nad $K$ i $m \ge 3$, to grupa $PSL(V)$ jest prosta.
\wcht{Tw. Schottenfelsa} (\datum{1900}): $PSL_34$ i $A_8$ mają ten sam rząd i są proste, ale nie izo-.

Grupy \wcht{klasyczne}: ogólne \prawo{8.5} liniowe, symplektyczne, unitarne i ortogonalne.
Istnieje \prawo{12.X} grupa uniwersalna: skończenie prezentowana, zawierająca izo-kopię każdej skończenie prezentowanej jako podgrupę.
Istnieje też taka (skończenie prezentowana), która zawiera izo-kopie przeliczalnych abelowych.
\wcht{Tw. Adiana-Rabina} (\datum{1958}, Markov osiem lat wcześniej dla półgrup): jeśli $M$ jest własnością Markova, to nie istnieje proces decyzyjny, który określałby, czy dana skończona prezentacja przedstawia grupę o własności $M$.
\wcht{Własność Markova} $M$ skończenie prezentowanych grup: zachowywana przez izo-, istnieje $G_1$, która ma tę własność i $G_2$, której nie da się włożyć w grupę z własnością $M$.
Przykłady: rzędu 1, skończoność, skończony wykładnik, bycie $p$-grupą, przemienność, rozwiązalność, nilpotentność, torsyjność, beztorsyjność, wolność.
\wcht{Problem słowa}.
Tylko grupa uniwersalna jest skończenie prezentowana i bez własności Markova.
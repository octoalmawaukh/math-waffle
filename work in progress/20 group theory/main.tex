
\documentclass[a4paper, fleqn, 9pt]{extarticle}
\usepackage{etoolbox}
\usepackage{Alegreya, euler}

\usepackage{amssymb, mathrsfs}
\usepackage[left=1.5cm,top=2.5cm,right=1.5cm,bottom=1.5cm]{geometry}
\usepackage{multicol}
\usepackage{MnSymbol}

\usepackage{fancyhdr}
\makeatletter
\fancypagestyle{mypagestyle}
{\newpage \fancyfoot[C]{} \renewcommand{\footrulewidth}{0pt}}
\makeatother
\pagestyle{mypagestyle}
\headsep 5pt                 %% put this outside
\usepackage{lastpage}

\usepackage[usenames,dvipsnames]{color}
\usepackage{xcolor}

\newcommand{\prawo}[1]{\marginpar{\textbf{\underline{#1}}}}
\newcommand{\wcht}[1]{{\bf {#1}}}
\newcommand{\spk}{\, \bullet \,}
\newcommand{\datum}[1]{{\color{Purple} \bf {#1}}}

% ? - pożyteczne zbiory
\newcommand{\K}{\mathbb{K}}
\newcommand{\C}{\mathbb{C}}
\newcommand{\N}{\mathbb{N}}
\newcommand{\R}{\mathbb{R}}
\newcommand{\Q}{\mathbb{Q}}
\newcommand{\Z}{\mathbb{Z}}
\newcommand{\Qp}{\mathbb{Q}_p}
\newcommand{\Zp}{\mathbb{Z}_p}

% Logika
\newcommand{\Lra}{\Leftrightarrow}
\newcommand{\Ra}{\Rightarrow}

% Algebra
\newcommand{\trk}{\trianglelefteq}
\newcommand{\im}{\operatorname{im}}
\newcommand{\autgrp}{\operatorname{Aut}}
\newcommand{\inngrp}{\operatorname{Inn}}
\newcommand{\outgrp}{\operatorname{Out}}
\newcommand{\holmrf}{\operatorname{Hol}}























% Analiza
%\newcommand{\D}{\textrm{d}}
%\DeclareMathOperator{\grad}{grad}
%\DeclareMathOperator{\rot}{rot}
%\DeclareMathOperator{\dvrg}{div}

% Miara
%\newcommand{\Rr}{\overline \R}

% Algebra abstrakcyjna

%\newcommand{\krt}{\trianglerighteq}


% Topologia
%\DeclareMathOperator{\cl}{cl}
%\DeclareMathOperator{\interior}{int}

% Pstwo
%\DeclareMathOperator{\pstwo}{\mathcal{P}}
%\DeclareMathOperator{\esssup}{ess\,sup}
%\newcommand{\expected}{\mathbb{E}}
%\newcommand{\variance}{\mathbb{V}}
%\newcommand{\prawie}{\leadsto}
%\newcommand{\wgpstwa}{\rightarrowtail}
%\newcommand{\wgrozkladu}{\twoheadrightarrow}

% Liniowa
%\newcommand{\rank}{\operatorname{rank}}
%\newcommand{\chara}{\operatorname{char}}
%\newcommand{\pf}{\operatorname{Pf}}
%\newcommand{\ann}{\operatorname{Ann}}

% \DeclareMathOperator{\autom}{\mathcal A}
% \DeclareMathOperator{\centror}{\mathcal C}
% \DeclareMathOperator{\chara}{char}
% \DeclareMathOperator{\cl}{cl}
% \DeclareMathOperator{\covariance}{cov}
% \DeclareMathOperator{\diam}{diam}
% \DeclareMathOperator{\disexp}{Exp}
% \DeclareMathOperator{\esssup}{ess\,sup}
% \DeclareMathOperator{\grupapsl}{\mathrm{PSL}}
% \DeclareMathOperator{\holomorf}{\mathcal H}
% \DeclareMathOperator{\innerm}{\mathcal I}
% \DeclareMathOperator{\interior}{int}
% \DeclareMathOperator{\normor}{\mathcal N}
% \DeclareMathOperator{\outerm}{\mathcal O}
% \DeclareMathOperator{\rad}{rad}
% \DeclareMathOperator{\rot}{rot}
% \DeclareMathOperator{\zenter}{\mathcal Z}
% \newcommand{\Ab}{\catname{Ab}}
% \newcommand{\Aut}{{\textrm {Aut}}}
% \newcommand{\bigslant}[2]{{\raisebox{.2em}{$#1$}\left/\raisebox{-.2em}{$#2$}\right.}}
% \newcommand{\calka}[4] {\int_{\farbea{#1}}^{\farbeb{#2}} #3 \, \textrm{d}{#4}}
% \newcommand{\catname}[1]{{\normalfont\textbf{#1}}}
% \newcommand{\cf}{\text{cf }}
% \newcommand{\Ci}{\operatorname{Ci}}
% \newcommand{\col}{\mathit{Col}}
% \newcommand{\cov}{\text{cov }}
% \newcommand{\Cp}{\mathbb{C}_p}
% \newcommand{\CRng}{\catname{CRng}}
% \newcommand{\dotr}[1]{#1^{\cdot}} 
% \newcommand{\dvrg}{\textrm{div }}
% \newcommand{\dwumk}[2] {\left[{#1 \atop #2}\right]}
% \newcommand{\dwumo}[2] {\left\{{#1 \atop #2}\right\}}
% \newcommand{\dwump}[2] {\left\{\begin{matrix} #1\\ #2\\ \end{matrix} \right\}}
% \newcommand{\dwum}[2] {\left({#1 \atop #2}\right)}
% \newcommand{\Ei}{\operatorname{Ei}}
% \newcommand{\ex}{{\textrm E}}
% \newcommand{\Fp}{\mathbb{F}_p}
% \newcommand{\grad}{\text{grad }}
% \newcommand{\Grp}{\catname{Grp}}
% \newcommand{\hipergeo}[3] {\left(\left.{#1 \atope #2}\,\right|\,#3\right)}
% \newcommand{\igr}{{\textrm Y}}
% \newcommand{\iks}{{\textrm X}}
% \newcommand{\imaz}{\mathfrak{Im } }
% \newcommand{\imp}{\bf}
% \newcommand{\inj}{\hookrightarrow}
% \newcommand{\kalendarz}{\color{wordblu} \bf}
% \newcommand{\kcod}{\textrm{cod }}
% \newcommand{\kdom}{\textrm{dom }}
% \newcommand{\khom}{\textrm{hom}}
% \newcommand{\kid}[1]{\textrm{id}_{#1} }
% \newcommand{\kkcod}{\textrm{cod }}
% \newcommand{\kkdom}{\textrm{dom }}
% \newcommand{\kkid}[1]{\textrm{id}_{#1} }
% \newcommand{\krt}{\trianglerighteq}
% \newcommand{\K}{\mathbb{K}}
% \newcommand{\li}{\operatorname{li}}
% \newcommand{\lk}{\mathit{lk}}
% \newcommand{\mydelta}{{\color{wordred}\delta}}
% \newcommand{\mysigma}{{\color{wordred}\sigma}}
% \newcommand{\oprot}{\text{rot }}
% \newcommand{\pf}{\text{Pf }}
% \newcommand{\pocho}[2] {{#1}^{\overline{#2}}}
% \newcommand{\pochu}[2] {{#1}^{\underline{#2}}}
% \newcommand{\PSL}{\text{PSL}}
% \newcommand{\rank}{\text{rank }}
% \newcommand{\reel}{\mathfrak{Re }\, }
% \newcommand{\rhn}[1] {\left \lsem {#1} \right \rsem}
% \newcommand{\Span}{\mathit{span}}
% \newcommand{\sqrr}[1] {#1^{1/2}}
% \newcommand{\stda}{\color{wordblu}}
% \newcommand{\stdb}{\color{wordblu}}
% \newcommand{\summ}{\sum_{m=1}^\infty}
% \newcommand{\sumn}{\sum_{n=1}^\infty}
% \newcommand{\surj}{\twoheadrightarrow}
% \newcommand{\Top}{\catname{Top}}
% \newcommand{\tostar}{\ensuremath{\mathaccent\star\to}}
% \newcommand{\unif}{\rightrightarrows}

\usepackage{graphicx}
\relpenalty=10000
\binoppenalty=10000
\interlinepenalty=10000

\newenvironment{enumx}{\begin{enumerate} \setlength{\itemsep}{0pt} \setlength{\parskip}{0pt} \setlength{\parsep}{0pt}}{\end{enumerate}}
\newenvironment{itemx}{\begin{itemize} \setlength{\itemsep}{0pt} \setlength{\parskip}{0pt} \setlength{\parsep}{0pt}}{\end{itemize}}


\usepackage[polish]{babel}
\usepackage[utf8]{inputenc}
\usepackage[T1]{fontenc}
\selectlanguage{polish}

\begin{document}
\setlength{\belowdisplayskip}{2pt}
\setlength{\belowdisplayshortskip}{2pt}
\setlength{\abovedisplayskip}{2pt}
\setlength{\abovedisplayshortskip}{2pt}

	
\renewcommand{\footrulewidth}{0.4pt}
\fancyhead[LE,LO]{Teoria grup (MSC 20)}
\fancyhead[RO,RE]{leon.aragones@gmail.com, lipiec 2016}
\fancyfoot[RF]{Rotman, styczeń 1984}

Obiekty \prawo{1.5} $a, b$ \wcht{izomorficzne}: strzałka $e \colon a \to b$ jest \wcht{odwracalna} w $C$  (istnieje $e' \colon b \to a$ w $C$, że $e'e = 1_a$, $ee' = 1_b$, ,,$e^{-1}$''). 
Strzałka $m \colon a \to b$ \wcht{mono-} w $C$: dla równoległych $f_1, f_2 \colon d \to a$ równość $m \circ f_1 = m \circ f_2$ pociąga $f_1 = f_2$.
W \textsc{Set} i \textsc{Grp} mono- to injekcje.
Strzałka $h \colon a \to b$ \wcht{epi-} w $C$: dla strzałek $g_1, g_2 \colon b \to c$ równość $g_1 \circ h = g_2 \circ h$ pociąga $g_1 = g_2$.
W \textsc{Set} są to surjekcje.
\wcht{Prawa odwrotność} dla $h$: $r \colon b \to a$, że $hr = 1_b$, sekcja.
\wcht{Lewa odwrotność}: analogicznie, retrakcja.
Strzałki z sekcjami $\Ra$ epi-, $\Leftarrow$ dla \textsc{Set}, ale nie \textsc{Grp}.
Strzałki z retrakcjami są mono-.
Jeżeli $gh=1_a$, to $g$ jest \wcht{rozdartym epi-}, $h$ rozdartym mono, zaś $f=hg$ jest idempotentna.
Do obiektu \wcht{terminalnego} (z \wcht{inicjalnego}) prowadzi po jednej strzałce z (do) każdego.
\wcht{Zerowy}: taki i taki.
Epi- mono- może się nie odwracać!
\wcht{Grupoid}: kategoria bez nieodwracalnych strzałek.

\fancyfoot[LF]{strona \thepage { }z \pageref{LastPage}} %[od 1 do 113]}

Każdy ciąg zstępujących \prawo{2.X} przedziałów $[a_n, b_n]$ długości dążącej do zera (\wcht{gnieżdżący się}) wyznacza pewną liczbę rzeczywistą.
Przykłady: pierwiastek z $a_0b_0$ i $a_{n+1} = H(a_n, b_n)$, $b_{n+1} = A(a_n, b_n)$; średnia arytmetyczno-geometryczna.
\wcht{Ciąg liczbowy}: odwzorowanie $\N \to \C$.
Jeśli $a_n > 0$ i $a_{n+1}/a_n \to a$, to $n$-ty pierwiastek z $a_n$ też dąży do $a$.
\wcht{Dzielenie mnożeniem}: $x_{n+1} = x_n(2-ax_n)$ zbiega kwadratowo do $1/a$ dla $0 < a x_0 < 2$.
Podobnie można szukać pierwiastka z $a$ ($x_n$ zbiega kwadratowo, $y_n$: sześciennie).
\wcht{Tw. o kanapce}: jeśli $a_n\le x_n\le b_n$, a przy tym $a_n$ oraz $b_n$ mają wspólną granicę $s$, to również $x_n$ dąży do tej liczby.
\[
	x_{n+1} = \frac{1}{2}\left(x_n+\frac{a}{x_n}\right) \spk
	y_{n+1} = \frac{y_n^3 + 3ay_n}{3y_n^2+a} \hfill
	b_{n+1}-a_{n+1} \le_{\text{HA}} \frac{(b_n-a_n)^2}{4a} \spk
	b_{n+1}-a_{n+1} \le_{\text{GA}} \frac{(b_n-a_n)^2}{8a}
\]


% Równanie \prawo{3.1} $ax^2+bxy+cy^2+dx+ey=k$ (współczynniki z $\Z$) dla $|a|+|b|+|c|>0$ ma wyróżnik $\Delta=b^2-4ac$.
% Jeśli $\Delta=0$, to sprowadza się do liniowego lub kwadratowej kongruencji; jeśli nie, to do $AX^2+BY^2=C$ z $A,B,C\in\Z$.
% Jeśli $ABC\neq0$ i znamy rozwiązanie $(x_0,\pm y_0)\in\Q^2$, to pozostałe dostajemy z khm-1 ($\lambda\in\Q$); brak rozwiązań w $\Q\Lra$ w $\R$ lub ,,któregoś spośród: $a \mid x^2-bc$, $b \mid x^2-ca$, $c \mid x^2-ab$''.
% \wcht{Euklides}: wszystkie naturalne rozwiązania $x^2+y^2=z^2$ z $(x,y)=1$, $2\mid y$, są dane przez $(m^2-n^2,2mn,m^2+n^2)$ dla względnie pierwszych $m>n$.
% \wcht{Pell}: $x^2-dy^2=1$, ciekawe dla niekwadratu $d>0$.
% \wcht{Lagrange}: jeśli $(x_1,y_1)$ jest rozwiązaniem z $y_1>0$ i minimalnym $x_1>0$, to pozostałe spełniają $x_n+y_nd^{1/2}=(x_1+y_1d^{1/2})^n$; jest ich $\infty$-wiele oraz $x_{n+1}=x_1x_n+dy_1y_n$, $y_{n+1}=x_1y_n+y_1x_n$, $x_{n+1}=2x_1x_n-x_{n-1}$, $y_{n+1}=2x_1y_n-y_{n-1}$.
% \wcht{Lemat Dirichleta}: każdej $a\in\R$ i $N\in\N$ odpowiada $m/n\in\Q$ (skrócona), że $1\le n\le N$ i $|a-m/n| \le 1/(nN)$.
% \[
% 	\left(x_0-2\cdot\frac{Ax_0+\lambda By_0}{A+B\lambda^2},y_0-2\lambda\cdot\frac{Ax_0+\lambda By_0}{A+B\lambda^2}\right)
% \]
% {\color{Red} Stosunkowo szybką metodę testowania... strona 88.}

% %Równanie \prawo{3.2} $x^n+y^n = z^n$ nie ma rozwiązań dla $n \ge 3$ (Wiles, \datum{1995}). \\
% %Krzywe eliptyczne.
% {
% Nam \prawo{3.2} dui \color{gray} ligula, fringilla a, euismod sodales, sollicitudin vel, wisi. Morbi auctor lorem non justo. Nam lacus libero, pretium at, lobortis
% vitae,ultricieset,tellus. Donecaliquet,tortorsedaccumsanbibendum,eratligulaaliquetmagna,vitaeornareodiometusami. Morbiacorci
% et nisl hendrerit mollis. Suspendisse ut massa. Cras nec ante. Pellentesque a nulla. Cum sociis natoque penatibus et magnis dis parturient
% montes, nascetur ridiculus mus. Aliquam tincidunt urna. Nulla ullamcorper vestibulum turpis. Pellentesque cursus luctus mauris.
% }

Zbiór \wcht{funkcji arytmetycznych} \prawo{4.1} ($\N \to \C$) %($\varphi$, $\omega$, $\sigma_k$, $\Omega$, $\mu$).
z dodawaniem i \wcht{splotem Dirichleta} $f*g \colon n \mapsto \sum_{d \mid n} f(d) g(n : d)$ jest przemiennym pierścieniem bez dzielników zera, $n \mapsto [n=1]$ jest jedynką.
Funkcja $f$ odwraca się $\Lra f(1) \neq 0$.
Zbieżność absolutna khm-1-a dla $f, g \in \mathbb A$ oraz pewnego $z \in \C$ pociąga to samo dla khm-1-b.
\wcht{Splot Abela}: $f \times g \colon n \mapsto \sum_{k=0}^n f(k)g(n-k)$, określony na $\mathbb A_0$ ($\N_0 \to \C$).
Ma jedynkę $n \mapsto [n = 0]$; $f$ się odwraca $\Lra f(0) \neq 0$.
\wcht{Splot unitarny}: $f \circ g \colon n \mapsto \sum_* f(d) g(n/d)$, sumowanie po $d \mid n$, że $(d, n/d) = 1$.
\wcht{Sumowanie Abela}: gdy $a_i, b_i \in \C$, $A(m) = a_1 + \ldots + a_m$ i $c_m = b_{m+1} - b_m$, to $\sum_{i=1}^n a_i b_i = A(n) b_n - \sum_{m=1}^{n-1} A(m) c_m$.
\wcht{Wzór Eulera-MacLaurina}: \emph{Analiza 1}.
Dla $x \ge 2$ i $c > -1$, $\{f(n)\}$ jest skrótem $\sum_{n \le x} f(n)$ (tylko tu!): $\{n^c\} = x^{1+c}/(1+c) + O(x^c)$, $\{1/n\} = \log x + \gamma + O(1/x)$, wzór Stirlinga (\emph{Kombinatoryka}).
\wcht{Tw. Césaro}: jeśli $g = 1 * f$ i szeregi $\sum_{n=1}^\infty g(n) x^n$, $\sum_{n=1}^\infty f(n) x^n / (1-x)$ są absolutnie zbieżne w $x$, to mają równe sumy.
Pierścień arytmetycznych funkcji ze splotem Cauchy'ego jest izo- z ,,formalnym'' $\C[X]$, ze splotem Dirichleta: $\C[X_1, X_2, \ldots]$.
\[
	\left[\sum_{n=1}^\infty \frac{f(n)}{n^z} \right] \cdot \left[\sum_{n=1}^\infty \frac{g(n)}{n^z} \right] = \sum_{n=1}^\infty \frac{(f*g)(n)}{n^z} \spk
	\left\{\frac{1}{n \log n}\right\} = \log \log x + C_1 + O \left( \frac{1}{x \log x}\right) \spk
	\left\{\frac{\log n}{n}\right\} = \frac{\log^2 x}{2} + O \left(\frac{\log x}{x}\right)
\]

Funkcja \wcht{addytywna} \prawo{4.2} spełnia $f(mn) = f(m) + f(n)$ dla $(m,n) = 1$, \wcht{w pełni}: dla wszystkich; \wcht{multiplikatywna}: $f(mn) = f(m) f(n)$; te ostatnie ze splotem Dirichleta są grupą.
Wniosek Bella: splot multi- z niemulti- nie jest multi-.
Przykłady: $d$, $\sigma$, $\sigma_k$.
\wcht{Wzór Möbiusa}: $f$, $g$ są arytmetyczne, $\sum_{d \mid n} f(d) = g(n) \Lra \sum_{d\mid n} \mu (d) g(n/d) = f(n)$ (Dedekind, Liouville \datum{1857}).
Khm-3: Walfisz (\datum{1963}).
Jeśli funkcja $f$ spełnia ,,$\lim_n f(2n+1) - f(n)$ istnieje'' lub ,,$f(n+1) - f(n) \to 0$, $f$ addytywna'', to $f(n) = c \log n$ (Mauclaire, Erdös).

Dla każdego $\delta > 0$ prawdą jest $d(n) = o(n^\delta)$.
Dla $x \ge 2$, $\sum_{n \le x} d(n) = x \log x + (2\gamma - 1) x + O(x^{1:2})$.
Jeżeli funkcja $f$ jest multiplikatywna, $S = \sum_{n \ge 1} f8n)$ zbiega absolutnie, to $\prod_p \sum_{k \ge 0} f(p^k)$ też, do tego samego.
\wcht{Tw. von Sternecka}: $(1 * (f \circ g)) = (1 * f)(1 * g)$, przy czym $(f \circ g)(n) = \sum f(r)g(s)$, $[r, s] = n$.
\[
	\limsup_n \frac{\varphi(n)}{n} = 1 \spk
	\liminf_n \frac{\varphi(n)}{n} = 0 \spk
	\sum_{n \le N} \varphi(n) = 3 \frac{N^2}{\pi^2} + O(N [\log N]^{2:3} [\log \log N]^{4:3}) \spk
	\sum_{n=1}^\infty \frac{\mu(n)}{n} = 0
\]


% % \wcht{Gęstość górna, dolna}: \prawo{4.3} $d^* = \limsup_{x \to \infty} |A \cap [1,x]| / x$ ($d_* = \liminf \dots$).
% % Gęstość unii rozłącznej dwóch jest sumą gęstości, ale nie jest miarą.
% % Jeśli $a_k$ jest ciągiem naturalnych parami względnie pierwszych, zaś $A$ zbiorem naturalnych niepodzielnych przez żadną z $a_k$, to $A$ ma gęstość: gdy $\sum_{i=1}^\infty 1/a_i$ rozbiega, to $d(A) = 0$, jesli nie, to $d(A) = \prod_{i=1}^\infty (1-1/a_i)$.
% % Wniosek: $\mathbb P$ ma gęstość zero.


% % \wcht{Górna/dolna wartość średnia}: $M^*(f) = \limsup_{x \to \infty} \sum_{n \le x} f(n) /x$ (inferior).
% % \wcht{Tw. van der Corputa}: jeśli $g$ ma wartość średnią, zaś $H = \sum_{n=1}^\infty h(n) / n$ abso-zbiega, to $g*h$ też ma wartość średnią, $M(g) \cdot H$.
% % \wcht{Aproksymanta} $g$ dla $f$: dla każdego $\varepsilon > 0$ $\{n \colon |f(n) - g(n)| > \varepsilon g(n)\}$ jest gęstości zero.
% % \wcht{Tw. Bircha}: jeśli $f$ jest multiplikatywna, dodatnia i nieograniczona, ma aproksymantę niemalejącą $\Ra$ $f(n) = n^c$.
% % \wcht{Wniosek Erdösa}: jeśli $f$ jest multiplikatywna, dodatnia i monotoniczna, to $f(n) = n^c$; jeśli jest addytywna i monotoniczna, to $g(n) = c \log n$.

% \wcht{Charakter} \prawo{4.4} grupy $G$ (skończonej, abelowej): homo- $G \to \{z \in \C^\times : |z| = 1\}$, tworzą \wcht{grupą dualną}, $\widehat G$.
% Jeżeli $g$ generuje $\Z_n$, to $\widehat G \cong G$, zaś jej elementy to $\varphi_j(g^r) = \exp(2 \pi i r j / n)$ dla $0 \le j, r \le n-1$.
% Gdy $H = G_1 \oplus G_2$, to $\widehat H \cong \widehat G_1 \oplus \widehat G_2$.
% Jeśli $g \in \Z_n$, dla każdego $\phi \in \widehat \Z_n$ jest $\phi(g) = 1$, to $g$ jest jednością, zaś $|\widehat G| = \varphi(n)$.
% \wcht{Charakter Dirichleta}: $\chi(n) = \psi(n \textrm{ mod } k)$, gdy $(n, k) = 1$ i $0$ w.p.p. ($\psi$ to charakter $\Z/k\Z$). 
% Własności: $\chi(n+k) = \chi(n)$, $\chi(n) = 0 \Lra (n, k) \neq 1$ oraz $\chi(mn) = \chi(m)\chi(n)$.
% Funkcja arytmetyczna z tymi własnościami jest charakterem mod $k$.
% Charakter pochodzący od $\psi = 1$: $\chi_0$, \wcht{główny}.
% Symbol Legendre'a, Jacobiego ($(\frac{x}{k}) = \prod_{j=1}^r (\frac{x}{p_j})^{\alpha_j}$ dla $x \in \N$, gdy $2 \nmid k = \prod_{j=1}^r p_i^{\alpha_i}$).

% Jeśli $m \mid n$ i $\chi$ jest charakterem mod $m$, to wzór $\chi_1(x) = \chi(x \textrm{ mod } m)$ (gdy $(x, n) = 1$) lub $0$ (w.p.p) określa charakter mod $n$.
% Mówi się, że $\chi_1$ jest indukowany przez $\chi$.
% \wcht{Przewodnik} charakteru $\chi$ mod $k$: minimalne $d$, że istnieje charakter $\chi_1$ mod $d$ indukujący $\chi$.
% \wcht{Pierwotny} charakter: $d = k$.
% Charakter $\chi$ mod $k$ jest pierwotny $\Lra$ dla każdego $d \mid k$, $d < k$, istnieje $n \in 1 + d\Z$, że $(n, k) = 1$ i $\chi(n) \neq 1$.

% Jeśli $a \in \Z$, $\zeta_k$ to pierwotny $k$-ty pierwiastek jedności, to $\tau_a (\chi) = \sum_{n=1}^k \chi(n) \zeta_k^{an}$ (gdzie $\chi$ jest charakterem mod $k$) jest \wcht{sumą Gaußa}.
% \wcht{Tw. Polyi-Winogradowa}: jeśli $k \ge 3$, zaś $\chi$ pierwotnym charakterem modulo $k$, to dla $x \ge 1$ jest: $\left|\sum_{n \le x} \chi (n) \right| \le \sqrt{k} \log k$.

% \newpage








% Wniosek: jeśli $p > 2$ jest pierwsza, zaś $n_2(p)$ najmniejszą dodatnią nieresztą kwadratową modulo $p$, to $n_2(p) \le 1 + \sqrt{p} \log p$.

\wcht{Forma Pfaffa} \prawo{5.1} (1-forma) odwzorowanie $\omega \colon (U \subseteq_o \R^n) \to \textrm{L}(\R^n, \C)$; \wcht{rzeczywista}: zawsze $\omega(x)[\R^n] \subseteq \R$.
Dyferencjały różalnych $f \colon U \to \C$, $\sum_{i=1}^n \partial_i f(x)h_i = \D f (x) h$, to przykłady 1-form.
Rzeczywiste 1-formy odpowiadają polom wektorów, polu $v \colon U \to \R^n$ przypisujemy formę $\omega_v$, $\omega_v(x) h = \langle v(x) \mid h \rangle$.
Każda forma liniowa $\omega(x)$ ,,pochodzi'' od $v_\omega(x)$: $\omega(x)h = \langle v_\omega(x) \mid h \rangle$, $x \mapsto v_w(x)$ jest szukanym polem.
Dyferencjał dostanie gradient: $v = \grad f$ gdy $\omega_v = \textrm{d}f$.
Niech $a_i(\xi) = \omega(\xi) e_i$, wtedy $\omega(\xi) h = \sum_{i=1}^n a_i(\xi) \textrm{d}x_i (\xi)h$, w skrócie $\omega = \sum_i a_i \textrm{d}x_i$.
Funkcje $a_i$ to \wcht{współczynniki} względem $\D x_i$.

Forma \prawo{5.2} $\omega$ na $U$ jest \wcht{całkowalna wzdłuż $\gamma$}, gdy istnieje $I$, że każdy $\varepsilon > 0$ ma $\delta > 0$, że rozkład $Z$ odcinka $[a, b]$ drobniejszy niż $\delta$ pociąga dla każdego $Z'$ nierówność $|S(Z, Z') - I| < \varepsilon$; rozkład $Z$: $a = t_0 < \dots < t_r = b$; wtedy $Z' = \{t_k' \in [t_{k-1}, t_k] : 1 \le k \le r\}$.
Nie można całkować wzdłuż każdej drogi, $\gamma = (\gamma_1, \ldots, \gamma_n) \colon [a, b] \to \R^n$ jest \wcht{całkodrogą}, gdy istnieją ,,Regel-'' $\dot \gamma_i$, których $\gamma_i$ to pierwotne.
Wzdłuż całkodrogi $\gamma$ każda ciągła 1-forma $\omega = \sum_{i=1}^n a_i \D x_i$, $\omega$ jest całkowalna i khm-3.
\[
	S(Z, Z') = \sum_{k=1}^r \omega(\gamma(t_k'))(\gamma(t_k) - \gamma(t_{k-1})) \spk
	I = \int_\gamma \omega \spk
	\int_\gamma \omega = \int_a^b \omega(\gamma(t)) \dot \gamma (t) \,\D t = \int_a^b \sum_{i=1}^n a_i(\gamma(t)) \cdot \dot{\gamma}_i (t) \,\D t
\]

Forma \prawo{5.3} Pfaffa $\omega = \sum_{i = 1}^n f_i \D x_i$, która ma na $U \subseteq_o \R^n$ \wcht{potencjał} (\wcht{f. pierwotną}: różalną $f \colon U \to \C$, że $\omega = \D f$, tzn. $f_k = \partial_k f$) jest \wcht{dokładna}.
Jeśli $f$ to potencjał na $U$ ciągłej 1-formy $\omega$, to całka z $\omega$ wzdłuż każdej całkodrogi $\gamma$ w $U$ od $A$ do $B$ wynosi $f(B) - f(A)$.
Ciągła 1-forma $\omega$ na obszarze $U \subseteq_o \R^n$, którą można całkować niezależnie od drogi, ma pierwotną: $f(x) = \int_a^x \omega$ dla $x \in U$.

Ciągła \prawo{5.4} 1-forma $\omega$ na $U \subseteq_o \R^n$ jest \wcht{lokalnie dokładna} (\wcht{zamknięta}): każdy $x$ ma $U_x$, gdzie istnieje pierwotna $f$ dla $\omega$: $\omega \mid_{U_x} = \D f$.
Dla formy $\omega = \sum_{i=1}^n f_i \D x_i$ (ciągle różalnej) tw. Schwarza daje \wcht{warunek całkowalności} (konieczny dla zamkniętości): $\partial_i f_k = \partial_k f_i$ dla $i, k \le n$.
\wcht{Gwiezdny zbiór}: $X \subseteq \R^n$, o ile istnieje $a \in X$, że wszystkie odcinki $[a,x]$ leżą w $X$.
\wcht{Lemat Poincarégo}: ciągle różalna 1-forma na gwiezdnym zbiorze spełniająca warunek całkowalności
 ma pierwotną.
Niech $n = 3$.
Warunek całkowalności zmienia się w \wcht{bezrotacyjność}: $\rot v = (\partial_2 v_3 - \partial_3 v_2,$ $\partial_3 v_1 - \partial_1 v_3, \partial_1 v_2 - \partial_2 v_1)$, symbolicznie $\nabla \times u$, ma się zerować (dla różalnych pól).
Na gwiezdnych zbiorach jest też wystarczający.

\wcht{Homotopia} \prawo{5.5} dwóch krzywych $\gamma_i \colon [a,b] \to X$ o wspólnych końcach $A, B$: ciągłe $H \colon [a,b] \times [0,1] \to X$, że $H(\cdot, i) = \gamma_i$ dla $i \in \{0, 1\}$ oraz $H(a, s) = A$, $H(b,s) = B$.
Całka z lokalnie dokładnej 1-formy $\omega$ w $U \subseteq_o \R^n$ po homotopijnych drogach o tych samych końcach ma tę samą wartość.
\wcht{Wolna homotopia} zamkniętych krzywych: nie musi trzymać końców.
Indeks zaczepienia nie zależy wolnohomotopijnie od krzywej.
%\wcht{Tw. Brouwera o punkcie stałym}: każda ciągła z $B_\ge(0,1)$ w siebie ma punkt stały.
\wcht{Jednospójność}: łukowo spójny $X \subseteq \R^n$, gdzie każda zamknięta krzywa ściąŋa się do punktu.
Zamknięta 1-forma na jednospójnym obszarze $U \subseteq_o \R^n$ ma tam całki niezależne od drogi oraz pierwotną.
Dla $n = 3$: ciągle różalne pole wektorów bez rotacji jest gradientowe.

W \prawo{6.2} tym rozdziale żyjemy w $(\Omega, \mathcal M, \pstwo)$, wszystkie $\sigma$-ciaua zawierają się w $M$.
Mamy \wcht{warunkowa wartość oczekiwaną} (khm-1) dla $\pstwo_A(B)$ równego $\pstwo (B \mid A)$ i $X$ o skończonej nadziei.
Jeżeli $\pstwo(A) > 0$, to khm-2.
Jeżeli $\{A_i\}$ to przeliczalne rozbicie $\Omega$ i $\pstwo(A_i) > 0$, zaś z-losowa $X$ jest caukowalna, to khm-3.
Jeśli $\Omega = \bigcup_i B_i$, $\pstwo(B_i) > 0$ i $\mathcal G = \sigma(B_i : i \in I)$, to $\expected (X \mid \mathcal G)(\omega) = \sum_{i \in I} \expected (X \mid B_i) [\omega \in B_i]$. 
Tak zefiniowana z-losowa jest $\mathcal G$-mierzalna; dla $B \in \mathcal G$ mamy khm-4.
\[
	\expected (X \mid A) := \int_\Omega X \,\D \pstwo_A = \int_A \frac{X}{\pstwo(A)} \,\D \pstwo \spk
	\expected X = \sum_{i=0}^\infty \expected (X \mid A_i) \cdot \pstwo (A_i) \spk
	\int_B X \,\D\pstwo = \int_B \expected (X \mid \mathcal G) \,\D\pstwo
\]

\wcht{Warunkowa wartość oczekiwana} \prawo{6.3} całkowalnej z-losowej $X$ pod warunkiem $\sigma$-ciaua $\mathcal F \subseteq \mathcal M$ to $\mathcal F$-mierzalna z-losowa $\expected (X \mid \mathcal F)$, że dla $A \in \mathcal F$ całki z ,,$X \, \D \pstwo$'', ,,$\expected (X \mid \mathcal F) \, \D \pstwo$'' nad $A$ są równe.
Zawsze istnieje, jednoznacznie z dokładnością do zdarzeń o p-stwie zero.
\wcht{Wielkie twierdzenie}: z-losowe $X, X_i$ mają skończoną nadzieję, $\mathcal G \subseteq \mathcal F \subseteq \mathcal M$ to $\sigma$-ciaua.
\wcht{Nierówność Jensena}: dla wypukłej $\varphi \colon \R \to \R$, z-losowych $X$, $\varphi(X)$ z $L^1(\Omega, \mathcal M, \pstwo)$ i $\sigma$-ciaua $\mathcal F \subseteq \mathcal M$ mamy $\varphi (\expected(X \mid \mathcal F)) \le \expected (\varphi(X) \mid \mathcal F)$ p.n.
Jeżeli z-losowa $X$ spełnia $\expected |X| < \infty$, zaś $Y$ ma wartości w $\R^n$, to istnieje borelowska $h \colon \R^n \to \R$, że $\expected (X \mid Y) = h(Y)$.
Warwaroczem z-losowej $X$ pod warunkiem $\{Y = y\}$ nazywamy $h(y)$.

\wcht{4.A}: dla $\mathcal F$-mierzalnej $X$: $\expected (X\mid \mathcal F) = X$ p.n.
\wcht{4.B}: dla $X \ge 0$: $\expected (X \mid \mathcal F) \ge 0$ p.n.
\wcht{4.C}: $|\expected(X \mid \mathcal F)| \le \expected (|X| \mid \mathcal F)$ p.n.
\wcht{4.D}: $\expected (\alpha X_1 + \beta X_2 \mid \mathcal F)$ jest równe $\alpha \cdot \expected (X_1 \mid \mathcal F) + \beta \cdot \expected (X_2 \mid \mathcal F)$ p.n.
\wcht{4.E}: $X_n \uparrow X$ implikuje $\expected (X_n \mid \mathcal F) \uparrow \expected(X \mid \mathcal F)$ p.n. 
\wcht{4.G}: $\expected X = \expected( \expected (X \mid \mathcal F))$ p.n.
\wcht{4.F}: $\expected (X \mid \mathcal G)$, $\expected (\expected (X \mid  \mathcal F) \mid  \mathcal G)$ oraz $\expected (\expected (X \mid  \mathcal G) \mid \mathcal F)$ są równe sobie p.n.
\wcht{4.H}: dla niezależnych $\mathcal F$ i $\sigma(X)$: $\expected(X \mid \mathcal F) = \expected X$ p.n.
\wcht{4.I}: dla ograniczonej oraz $\mathcal F$-mierzalnej z-losowej $Y$, $\expected (XY \mid \mathcal F) = Y \expected (X \mid \mathcal F)$.

\wcht{Warunkowy Fatou}: dla $X_n \ge 0$ mamy $\expected (\liminf X_n \mid \mathcal F) \le \liminf \expected (X_n \mid \mathcal F)$.
\wcht{Levi}: gdy $|X_n (\omega)| \le Y (\omega)$, $\expected Y < \infty$ oraz $X_n \to X$ p.n., to $\lim_n \expected (X_n \mid \mathcal F) = \expected (X \mid \mathcal F)$ p.n.
\wcht{Wariancja}: $\variance (X \mid \mathcal F) := \expected ((X - \expected (X \mid \mathcal F))^2 \mid \mathcal F)$, gdy $\expected X^2 < \infty$, wtedy $\variance X = \expected \variance (X \mid \mathcal F) + \variance \expected (X \mid \mathcal F)$.
\wcht{Fubini}: $\sigma$-ciauo $\mathcal F \subseteq \mathcal M$, p. mierzalna $(E, \Sigma, \mu)$, $X \in L^1 (E \times \Omega, \Sigma \times \mathcal F, \mu \times \pstwo)$.
Wtedy khm-1, khm-2.
\wcht{Niezależność} $\sigma$-ciau $\mathcal F_1, \ldots, \mathcal F_n; \mathcal G \subseteq M$: $\pstwo (\bigcap_{i=1}^n A_i \mid \mathcal G) = \prod_{i=1}^n \pstwo (A_i \mid \mathcal G)$ dla każdego $A_i \in \mathcal F_i$.
$\mathcal F, \mathcal H$ są wnz względem $\mathcal G$ $\Lra$ dla każdego $H \in \mathcal H$, $\pstwo (H \mid \mathcal F \vee \mathcal G) = \pstwo (H \mid \mathcal G)$ p.n.
\[
	\expected \left | \int_E \expected (X_s \mid \mathcal F) \mu(\D s) \right| < \infty \spk
	\expected \left[\left. \int_E X_s \mu(\D s) \right\mid \mathcal F \right] = \int_E \expected(X_S \mid \mathcal F) \mu (\D s)
\]

\wcht{P-stwo warunkowe} \prawo{6.4} $A \in \mathcal M$ pod warunkiem $Y = y$: $\pstwo(A \mid Y = y) := \expected (1_A \mid Y= y)$. 
Gdy $(X,Y)$ ma ciągły rozkład o gęstości $g$, to khm-1 i khm-2 dla tych borelowskich $\varphi$, że $\expected |\varphi(x)| < \infty$ (gdy mianownik się zeruje, kładziemy $0$ po prawej).
\wcht{Uogólniony Bayes} $\mathcal G \subseteq \mathcal F$: $\sigma$-ciało, $B \in \mathcal G$, $A \in \mathcal F$, $\pstwo(A) > 0$ i ,,$\pstwo(A\mid \mathcal G) = \expected(\mathbb I_A \mid \mathcal G)$ dają khm-3.
\wcht{Abstrakcyjny}: $P, Q$ miarami probabilistycznymi na $(\Omega, \mathcal F)$, że gęstość $\D Q / \D P = Z >0$ istnieje, $\mathcal G \subseteq \mathcal F$, $X$: z-losowa $Q$-caukowalna; wtedy: $\expected_Q X = \expected_P XZ$ i khm-4 jest równe $\expected_Q (X \mid \mathcal G)$.
\[
	\pstwo (X\in B \mid Y) = \frac{\int_B g(x, Y) \, \D x}{\int_\R g(x,Y) \, \D x} \spk
	\expected (\varphi(x) \mid Y) = \frac{\int_\R \varphi(x) g(x, Y) \, \D x}{\int_\R g(x,Y) \, \D x} \spk
	\pstwo (B \mid A) = \frac{\int_B \pstwo (A \mid \mathcal G) \, \D \pstwo}{\int_\Omega \pstwo(A \mid \mathcal G) \, \D \pstwo} \spk
	\frac{\expected_P(XZ \mid \mathcal G)}{\expected_P (Z \mathcal G)}
 \]

\wcht{P-stwo} $B$ \prawo{6.5} pod warunkiem $\sigma$-ciaua $\mathcal F$: $\mathcal F$-mierzalna z-losowa $\pstwo (B \mid \mathcal F) := \expected (1_B \mid \mathcal F)$ o wartościach w $[0,1]$.
Khm-1 ($A \in \mathcal F$); jeśli $B_n$ są rozłączne parami, to khm-2.
\wcht{Regularny rozkład warunkowy} względem $\mathcal F$: funkcja $\pstwo_{\mathcal F} \colon \mathcal M \times \Omega \to [0,1]$, że $\pstwo_{\mathcal F} (B, \cdot)$ jest wersją $\expected(1_B, \mathcal F)$, zaś $\pstwo_{\mathcal F}(\cdot, \omega)$ to rozkłady p-stwa na $\mathcal M$.
Dla całkowalnej z-losowej $X$: khm-3. 
,,Regrowar'' nie musi istnieć, lecz istnieje on dla z-losowej $X$ pod warunkiem $\sigma$-ciaua $\mathcal F$ (funkcja $\pstwo_{X \mid \mathcal F}$ na $\mathfrak B(\R) \times \Omega$, że $\pstwo_{X \mid \mathcal F}(B, \omega) = \pstwo (X \in B \mid \mathcal F)(\omega)$, zaś $\pstwo_{X \mid \mathcal F}(\cdot, \omega)$ to rozkłady p-stwa na $\mathfrak B(\R)$).
\[
	\pstwo (A \cap B) = \int_A \pstwo (B \mid \mathcal F) \, \D \pstwo \spk
	\pstwo \left(\left. \bigcup_{n=1}^\infty B_n \right \mid \mathcal F \right) = \sum_{n=1}^\infty \pstwo (B_n \mid \mathcal F) \textrm{ p.n.} \spk
	\expected (X \mid \mathcal F)(\omega) = \int_{\Omega} X (\tilde{\omega}) \pstwo_{\mathcal F} (\D{} \tilde{\omega}, \omega) \textrm{ p.n.}
\]

% \fancyfoot[LF]{strona \thepage { }z \pageref{LastPage} [od 114 do 186]}

\wcht{Rozbicie} zbioru $A \subseteq \R^n$ o subtelności $\delta$: rodzina rozłącznych i mierzalnych $A_1, \dots, A_r$, że $\bigcup_k A_k = A$ i ich średnica jest $\le \delta$.
% 7-8
Fakt: dla ciągłej $f \colon A \to \C$ na zwartym $A \subset \R^n$ i każdego $\varepsilon > 0$ istnieje $\delta > 0$, że dla rozbić $A_1, \dots, A_r$ zbioru $A$ o subtelności $\delta$ i dowolnego wyboru $\xi_k \in A_k$ zachodzi khm.
\[
	\left| \int_A f(x) \,\D x - \sum_{k=1}^r f(\xi_k) \cdot v(A_k) \right| \le \varepsilon
\]

Jeśli \prawo{8.1} $z_n \to z \in \C$, to \prawo{\ldots} mamy \prawo{8.10} khm-1, \wcht{eksponensę} $\C \to \C$.
Rozwiązaniem układu $f(z + w) = f(z) f(w)$, $\lim_{z \to 0} \frac 1z [f(z) - 1] = c$ ($\C \to \C$) jest funkcja $\exp (cz)$.
Eksponensa jest nieujemna i rosnąca na $\R$, rośnie szybciej od dowolnego wielomianu.
\wcht{Logarytm} to funkcja odwrotna do eksponensy, rośnie wolniej od pierwiastków.
Potęgowanie: $x^y = \exp (y \ln x)$.
W $B_s$ i $L$, $x \in (-1, 1)$.
Mamy: $\lim_{s \to 0} \frac 1s [B_s(z) - 1] = L(z)$ i $B_s B_t = B_{s+t}$.
\wcht{Trygonometria}: $2i \sin z = \exp iz - \exp -iz$, $2 \cos x = \exp iz + \exp - iz$.
Najmniejsze dodatnie miejsce zerowe dla $\cos \colon \R \to \R$ to $\pi / 2$.
\wcht{Logarytm} $w = |w| \exp i \varphi$ to $\ln |w| + i \varphi$ (cięcie wzdłuż $(-\infty, 0]$).
Jeśli $\Re w_1$, $\Re w_2 > 0$, to $\ln w_1 w_2 = \ln w_1 + \ln w_2$.
\[
	\exp z := \sum_{k=0}^\infty \frac {z^k}{k!} = \lim_{n \to \infty} \left(1 + \frac{z_n}n \right)^n \spk
	B_s(x) = \sum_{n=0}^\infty {s \choose n} \cdot x^n = (1+x)^s \spk
	L(x) = \sum_{n=1}^\infty \frac{(-1)^{n-1}}{n} x^n = \log ( 1 +x)
\]

\wcht{Hiperboliczne}: \prawo{8.12} wykresem $\cosh^2 x - \sinh^2 x = 1$ jest hiperbola.
Krzywa łańcuchowa: $a \cosh (x/a)$, łańcuch wiszący na dwóch punktach.
\wcht{Reguła Osborna}: wziąć tożsamość trygonometryczną dla całkowitych potęg $\sin x$, $\cos x$, zamienić je na funkcje hiperboliczne i odwrócić znak iloczynów $4k+2$ funkcji $\sinh$.
Dla zespolonych $x$, trzeba się trochę namęczyć z odwrotnymi hiperbolicznymi.
\[
	\sinh x = \frac{e^x - e^{-x}}{2} \spk
	\cosh x = \frac{e^x + e^{-x}}{2} \spk
	\operatorname{asinh}\ x = \ln(x + \sqrt{x^2 + 1}) \spk
	\operatorname{acosh}\ x = \ln(x + \sqrt{x^2 - 1}) \spk
    \operatorname{atanh}\ x = \frac{1}{2} \ln\frac{1+x}{1-x}
\]


% \fancyfoot[LF]{strona \thepage { }z \pageref{LastPage} [od 187 do 470]}

%Rozdział dziewiąty: grupy Mathieu.

\wcht{Algebra zespolona}: \prawo{10.1} $\C$-liniowa przestrzeń z porządnym mnożeniem: łącznym, rozdzielnym, $\alpha(xy)= (\alpha x)y = x(\alpha y)$.
\wcht{Algebra Banacha}: dodatkowo $\mathscr B$ z normą taką, że $\|xy\| \le \|x\|\|y\|$ i $\|e\| = 1$ (neutralny).
Na każdej $\C$-algebrze, $\mathscr B$ z jednostkowym $e \neq 0$ i ciągłym z dwóch stron mnożeniem można zadać normę, która indukuje wyjściową topologię, zaś $A$ zamienia w algebrę Banacha.
Tu: \prawo{10.2} $\mathscr B$-algebra $A$, $x\in  A$, $\|x\| <1$, wtedy: $e-x$ odwraca się, $\|(e-x)^{-1} - e-x\|\le \|x\|^2 / (1-\|x\|)$; $|\phi(x)| < 1$ dla każdego $\C$-homo- $\phi$ na $A$.
\wcht{Tw. Gleasona-Kahane-Żelazka}: [$\phi$:  funkcjonał na $\mathscr B$-algebrze $A$, że $\phi(e) = 1$ i $\phi(x) \neq 0$ dla odwracalnych $x \in A$] $\Ra$ [$\phi(xy) = \phi(x)\phi(y)$].
\hfill $\mathscr B$-algebra: algebra Banacha!

Dla \prawo{10.3} algebry Banacha $A$, $(A^* \subseteq_o A) \to (A^* \subseteq_o A)$, $x \mapsto x^{-1}$, to homeo- ,,na''.
\wcht{Zbiór rezolwenty}: dopełnienie \wcht{widma} $\sigma(x)$ (zbioru $\lambda \in \C$, że $\lambda e-x$ się nie odwraca, niepusty i zwarty).
\wcht{Promień spektralny}: $\rho(x) = \sup\{ |\lambda| : \lambda \in \sigma(x)\}$, granica $\|x^n\|^{1/n}$.
\wcht{Tw. Gelfanda-Mazura}: jeśli $A = A^* \cup \{0\}$, to $A$ oraz $\C$ są izometrycznie izo-.
Jeśli dla algebry Banacha $A$ istnieje $M < \infty$, że $\|x \| \cdot \|y\| \le M\|xy\|$, to $A$ jest izometrycznie izo- z $\C$.
Jeśli $A$ to domknięta podalgebra algebry Banacha $B$ i $e_B \in A$, $x \in A$, to $\sigma_A(x)$ jest sumą $\sigma_B(x)$ i rodziny ograniczonych składowych jego dopełnienia.

Dla \prawo{10.4} $f \colon Q \to A$ (ciągłej ze zwartej, $\mathcal T_2$ z Bo-miarą $\mu$ w $A$, algebrę Banacha) całka istnieje (bo $A$ jest przestrzenią Banacha), zaś dla każdego $x \in A$ mamy khm-1, khm-2.
Niech $R(\lambda) = P(\lambda) + \sum_{m,k} c_{m,k} (\lambda - \alpha_m)^{-k}$ będzie wymierną z biegunami w $\alpha_n$ (skończenie wiele składników, $P$: wielomian).
Gdy $x \in A$ i $\sigma(x)$ nie zawiera żadnego bieguna $R$, to $R(x) := P(x) + \sum_{m,k} c_{m,k} (x - \alpha_m e)^{-k}$.
Jeśli $U \subseteq_o \C$ zawiera $\sigma(x)$, $R$ jest holo- na $U$ i $\Gamma$ otacza $\sigma(x)$ w $U$, to khm-3.
\wcht{Tw. o odwzorowaniu spektralnym}: jeśli liniowy operator $T$ jest ograniczony na zespolonej p. Banacha $X$, $\sigma(T) \subseteq U \subseteq_o \C$ i $f$ jest analityczna na $U$, to holofunkcyjny rachunek jest w stanie zdefiniować $f(T)$, że $f(\sigma(T)) = \sigma(f(T))$, a dla widma punktowego tylko $f(\sigma_p(T)) \subseteq \sigma_p(f(T))$, z równością np. gdy $f$ nie jest stała na żadnej składowej spójności $U$.  % w wersji ze stacka.
\[
	x \int_Q f \,\D \mu = \int_Q x f(p) \,\D \mu(p) \spk
	\left(\int_Q f \,\D \mu \right) x = \int_Q f(p)x \,\D \mu(p) \spk
	R(x) = \frac{1}{2\pi i} \int_\Gamma R(\lambda) (\lambda e - x)^{-1} \,\textrm{d}\lambda
\]

Niech \prawo{10.5} $G = A^*$ \prawo{10.6} dla algebry Banacha $A$, niech $G_1$ będzie składową spójności $e$ w $G$: jest otwarta, normalna, generowana przez $\exp A$.
Jeżeli $A$ jest przemienna, to $G_1 = \exp A$, zaś $G/G_1$ jest beztorsyjna.
%\wcht{Podprzestrzeń niezmiennicza} \prawo{10.6} operatora $T \in \mathcal B(X)$: $M \subseteq X$, że $T(M) \subseteq M$.
\wcht{Tw. Łomonosowa}: w $\C$-przestrzeni Banacha wymiaru $\infty$, wszystkie $S \in \mathcal B(X)$ komutujące ze zwartym $T \in \mathcal B(X) \setminus \{0\}$ mają podprzestrzenie niezmiennicze: $M \subseteq^a X$ (nietrywialną) że $S(M) \subseteq M$.

Domknięcie \prawo{11.1} ideału $J$ w przemiennej $\mathscr B$-algebrze $A$ jest ideałem; niech $\Delta$ będzie zbiorem wszystkich zespolonych homo- $A$.
Maksymalne ideały w $A$ to dokładnie jądra $h \in \Delta$; $x \in A$ jest odwracalny $\Lra$ nie leży w żadnym właściwym ideale.
\wcht{Lemat Wienera}: jeśli $f$ jest funkcją na $\R^n$ bez miejsc zerowych, $f(x) = \sum_* a_m e^{im\cdot x}$, $\sum_* |a_m| <\infty$ (sumy po $m \in \Z^n$), to $1/f(x)$ jest ,,też tej postaci''.

Tu $A$ jest \prawo{11.2} przemienna.
\wcht{Topologia Gelfanda} na $\Delta$: najsłabsza, przy której wszystkie $\widehat x(h) = h(x)$ (dla $h \in \Delta$, $x \in A$) są ciągłe.
Algebra $A$ \wcht{półprosta}: trywialny \wcht{radykał} (przekrój ideałów maksymalnych).
Fakt: ,,przestrzeń ideałów maksymalnych'' $\Delta$ jest zwarta i $\mathcal T_2$.
\wcht{Transformacja} $x \mapsto \widehat x$ ma jądro, radykał $A$.
Dla $x \in A$ obrazem $\widehat x$ jest widmo $\sigma(x)$.
Każdy izo- między półprostymi $\mathscr B$-algebrami przemiennymi jest homeo-.
Sama transformacja Gelfanda jest izo- ($\|x\| = \|\widehat x\|_\infty$) $\Lra$ $\|x^2\|= \|x\|^2$.

\wcht{Inwolucja}: \prawo{11.3} $x \mapsto x^*$ (z zespolonej algebry $A$ w siebie), jeśli $(x+y)^* = x^* + y^*$, $(ux)^*  =\overline{u}x^*$, $(xy)^* = y^*x^*$ i $x^{**} = x$ ($u \in \C$).
\wcht{Hermitowski}: $x = x^*$.
Jeśli $\mathscr B$-algebra jest przemienna i półprosta, to inwolucje na $A$ są ciągłe.
\wcht{$C^*$-algebra}: $\mathscr B$-algebra z inwolucją, że $\|xx^*\| = \|x\|^2$.
\wcht{Tw. Gelfanda-Najmarka}: jeśli $A$ jest przemienną $C^*$-algebrą z p. ideałów maksymalnych $\Delta$, to transformacja Gelfanda jest izometrycznym izo- $A$ na $C(\Delta)$ i $h(x^*)$ jest sprzężeniem $h(x)$.
Jeśli $A$ jest $\mathscr B$-algebrą z inwolucją, $x = x^* \in A$ i $\sigma(x)$ nie zawiera rzeczywistych $\lambda \le 0$, to istnieje $y \in A$, że $y = y^*$ i $y^2 = x$. % nie dopisuj przemienności!

Jeśli \prawo{11.4} $A$ jest $\mathscr B$-algebrą i $xy=yx$, to $\sigma(x+y) \subseteq \sigma(x) + \sigma(y)$ oraz $\sigma(xy) \subseteq \sigma(x) \sigma(y)$.
W algebrze $A$ z inwolucją, $x$ jest \wcht{normalny}: $xx^*$ to $x^*x$; $B \subseteq A$ jest normalny: komutuje i $x \in S \Ra x^* \in S$.
Jeśli $B$ jest maksymalny, to także domkniętą przemienną podalgebrą i $\sigma_B(x) = \sigma_A(x)$ dla $x \in B$.
W $\mathscr B$-algebrze z inwolucją $x \ge 0$ oznacza $x = x^*$ i $\sigma(x) \subseteq [0, \infty)$.
Własności $C^*$ algebry $A$: hermitowskie elementy mają rzeczywiste widma, jeśli $x \in A$ jest normalny, to $\rho(x) = \|x\|$, $\rho(y y^*) = \|y\|^2$, $u,v \ge 0$ $\Ra$ $u+v \ge 0$, $yy^* \ge 0$ i $e+yy^*$ odwraca się w $A$ ($u, v, y \in A$).

\wcht{Funkcjonał dodatni}: \prawo{11.5} $F$ na $\mathscr B$-algebrze $A$ z inwolucją, że $F(x x^*) \ge 0$.
Wtedy $F(x*)$ jest sprzężeniem $F(x)$, $|F(xy^*)|^2 \le F(xx^*) F(yy^*)$,  $|F(x)|^2 \le F(e) F(xx^*) \le F(e)^2 \rho(xx^*)$, $|F(x)| \le F(e) \rho(x)$ dla normalnych $x \in A$, $F$ jest ograniczonym funkcjonałem.
Mamy  $\|F\| = F(e)$ dla przemiennych $A$; jeśli inwolucja spełnia $\|x^*\| \le \beta \|x\|$, to $\|F\| \le \beta^{1/2} F(e)$.
Jeśli $A$ jest przemienną $\mathscr B$-algebrą z p. ideałów maksymalnych $\Delta$ i z inwolucją, która jest ,,symetryczna'' ($u(x^*)$ to sprzężenie ${u(x)}$ dla $u \in \Delta$), zaś $K$ zbiorem wszystkich dodatnich funkcjonałów $F$ na $A$, że $F(e) \le 1$; $M$ zbiorem dodatnich regularnych Bo-miar $\mu$ na $\Delta$, że $\mu(\Delta) \le 1$, to wzór khm-1 ustala bijekcję między wypukłymi $K, M$, która ,,trzyma'' punkty ekstremalne; zatem multiplikatywne funkcjonały liniowe na $A$ to dokładnie punkty ekstremalne $K$.
Jeśli $F \in K$, to NWSR: $F$ ekstremalny w $K$; $F(xy) = F(x)F(y)$; $F(xx^*) = F(x)F(x^*)$ dla $x, y \in A$.
\hfill Khm: $F(x) = \int_\Delta \widehat{x} \,\textrm{d}\mu \spk (x \in A)$

\wcht{Enefe} \prawo{12.1} na regularnym hiperstworze $M \subset \R^n$ to unormowane ciągłe pole wektorów $\nu \colon M \to \R^n$, które w każdym $x \in M$ jest prostopadłe do przestrzeni stycznej $T_xM$.
Enefe na poziomicy to unormowany gradient, na śladzie włożenia $\gamma \colon (\Omega \subseteq_o \R^{n-1}) \to \R^n$ w $\gamma (u)$ to zewnętrzny produkt $\partial_1 \gamma(u) \wedge \ldots \wedge \partial_{n-1} \gamma(u)$ (+ unormowanie).
Dla spójnych $M$, enefe są dwa lub wcale: na wstędze Möbiusa ich nie ma.
Pole wektorów $F \colon M \to \R^n$ jest \wcht{całkowalne} nad orientowalną $M$, gdy $x \mapsto \langle F(x) \mid \nu(x) \rangle$ jest, a to daje nowy rodzaj całki.

\wcht{Regularny brzegopunkt} \prawo{12.2} ($G \subseteq \R^n$): $a \in \partial G$, gdy ma otoczenie $U$ i $\mathscr C^1$-funkcję $q \colon U \to \R$, że $q'(x) \neq 0$ na $U$ i $G \cap U = \{x \in U : q(x) \le 0\}$.
\wcht{Singularny}: nieregularny.
\wcht{Regularny} (gładki) \wcht{brzeg}: $\partial_r G$; singularny: $\partial_s G = \partial G \setminus \partial_r G$.
\wcht{$\mathscr C^1$-wielościan}: $G \subseteq^k \R^n$ o $n$-zerowym singularnym brzegu.
%\wcht{Lokalna normalność brzegu}: każdy regularny brzegopunkt $a$ dla $\mathscr C^1$-wielościanu $G$ posiada kostkowe otoczenie $Q$, że: (po przenumerowaniu współrzędnych) $Q$ to produkt otwartej kostki $Q'$ w $\R^{n-1}$ i otwartego odcinka $I$, poza tym istnieje $\mathscr C^1$-funkcja $h \colon Q' \to I$, że albo (i) $G \cap Q =  \{(x', x_n) \in Q' \times I : x_n \ge h(x')\}$ albo (ii) $G \cap Q  = \{(x', x_n) \in Q' \times I : x_n \le h(x')\}$, przy czym $\partial G \cap Q = \Gamma$, wykres $h$.
Na regularnym brzegu $\mathscr C^1$-wielościanu $G$ istnieje dokładnie jedno ciągłe enefe $\nu$ (\wcht{zewnętrzne}), że dla każdego $x \in \partial_r G$ i małych $t > 0$ jest $x + t \nu(x) \in \R^n\setminus G$ i $x - t \nu(x) \in G$.
%Gdy dla $a \in \partial_r G$ wybierze się $U$ i $q$, to $\nu(x)$ dla $x \in \partial_r G \cap U$ zadane jest przez $\grad q(x) / \|\grad q(x) \|$.
Pole wektorów $F \colon \partial_r G \to \R^n$ jest \wcht{całkowalne nad $\partial G$} ($G \subseteq \R^n$: $\mathscr C^1$-wielościan z zewnętrznym enefe $\nu$): $\langle F \mid \nu \rangle$ jest całkowalna nad gładkim brzegiem $\partial_r G$ (khm).
\[
	\int_{\partial G} F \, \overrightarrow{\D S} = \int_{\partial_r G} \langle F, \nu \rangle \,\D S
\]

\wcht{Tw. całkowe Gaußa} (\datum{1840}): \prawo{12.4} jeżeli ciągłe pole wektorów na $G$ ($\mathscr C^1$-wielościanie w $\R^n$ o brzegu, który jest mierzalnym hiperstworem) jest ciągle różalne w jego wnętrzu i ma całkowalną dywergencję w tym wnętrzu, to khm-1.
%\emph{Jeśli $F$ to pole prędkości płynącej cieczy nieściśliwej, to lewa strona opisuje wydajność w $G$ zawartych źródeł i depresji (?), zaś prawa: łączny strumień płynący przez brzeg $G$}.
Wniosek: dla $\mathscr C^1$-wielościanu $G \subseteq \R^2$ ze skończonym $P \subseteq \partial G$ i parami rozłącznymi, mierzalnymi podrozmaitościami $M_1, \ldots, M_q \subseteq \partial_r G$ wymiaru $1$, że $(\partial G) \setminus P \subseteq \bigcup_k M_k$, a przy tym każdy $M_k$ to ślad włożenia $\gamma_k \colon (0,1) \to \R^2$ z taką orientacją, że $v(\gamma_k(t)) = -D(\dot \gamma_k(t) : \|\dot \gamma_k(t)\|)$ ($D$: obrót $(x, y) \mapsto (-y, x)$, $\gamma_k$: zewnętrzne enefe).
Dla ciągłej 1-formy $u \D x + v \D y$ na $G$ z ciągłą pochodną we wnętrzu $G$, nad którym $v_x - u_y$ jest całkowalne, mamy khm-2.
%Mamy $\mathscr C^1$-wielościan $G \subseteq \R^2$ oraz skończony $P \subseteq \partial G$ i parami rozłączne, mierzalne podrozmaitości  wymiaru jeden, że $(\partial G) \setminus P \subseteq \bigcup_k M_k$, przy czym $M_k$ musi być śladem włożenia $\gamma_k \colon (0,1) \to \R^2$ zorientowanego tak, że dla zewnętrznej ,,ene'' w zachodzi: $\nu(\gamma_i(t)) = - D (\dot \gamma_i(t) : \|\dot \gamma_i(t)\|)$ (uwaga: $D$ jest operatorem obrotu, $(x,y) \mapsto (-y, x)$). 
\wcht{Powierzchniowzór Leibniza}: pole powierzchni takiego $G$ zadane jest ,,brzegową całką'' khm-3.
\[
 	\int_G \operatorname{div} F \,\D x = \int_{\partial G} F \,\overrightarrow{\D S} \spk
 	%\nu(\gamma_k(t)) = -\textrm{D}\left(\frac{\dot{\gamma}_k(t)}{\|\dot{\gamma}_k(t)\|}\right) \spk
 	\int_G(v_x - u_y) \,\D (x,y) = \sum_{k=1}^q \int_{\gamma_k} u \,\D x + v \,\D y =: \int_{\partial G} u \D x + v \D y \spk
 	v_2(G) = \int_{\partial G} \frac{x \, \D y -y \,\D x}{2}
\]

\wcht{Wzory Greena} (\datum{1828}): \prawo{12.6} jeśli $G \subseteq \R^n$ jest $\mathscr C^1$-wielościanem, o brzegu: mierzalnym hiperstworze, to dla $f$, $g$ z $\mathscr C^2(G)$ zachodzi khm-1 oraz 2, gdzie $\partial_\nu h := \langle \nu, \grad h\rangle$ jest ablatywem $h$ w kierunku zewnętrznego enefe na $G$.
\wcht{,,Średniość'' harmofunkcji}: jeśli $h \colon (U \subseteq_o \R^n) \to \R$ jest harmofunkcją, to dla każdej w $U$ zawartej kuli $K_{\le r}(a)$ khm-3, przy czym $\omega_n$ to powierzchnia jednostkowej sfery wymiaru $n-1$.
%Ciągła $h \colon U \to \R$ na otwartym $U$ jest harmofunkcją dokładnie wtedy, gdy khm-3 zachodzi dla każdej $K_{\le r}(a) \subset U$.
Warunek ten charakteryzuje harmofunkcje.
Wynika stąd \wcht{reguła maksimum}: harmofunkcja $h \colon (U \subseteq_o \R^n) \to \R$ osiągająca swoje maksimum jest stała.
\[
	\int_G \langle \grad f, \grad g \rangle \,\D x = \int_{\partial G} f \partial_\nu g \,\D S - \int_G f \Delta g \,\D x \spk
	\int_G (f \Delta g - g \Delta f) \,\D x = \int_{\partial G} (f \partial_\nu g - g \partial_\nu f) \,\D S \spk
	h(a) = \int_{\partial K_r(a)}  \frac{h \,\D S}{\omega_n r^{n-1}} 
\]

%\newpage

Ile jest grup rzędu $n$?
Relacja pomiędzy pierwszymi $p$ i $q$: $p \mid q - 1$ lub ,,odwrotnie''.
\begin{itemx}
	\item {\color{Blue} Jedna}, dokładnie gdy $n = 1, p$ lub $\prod_i p_i$ ($p_i$ parami różne), że między $p_i$ nie ma relacji.
	\item {\color{Blue} Dwie}, dokładnie gdy $n = p_1 p_2 \cdot \ldots \cdot p_t q^2$ (bez żadnych relacji) i $p_i \nmid q + 1$ albo $n = p_1\cdot \ldots \cdot p_t$ (z jedną relacją).
	\item {\color{Blue} Trzy}, gdy $n = pq^2$ (brak relacji między $p$, $q$ i $p \mid q + 1$).
	\item {\color{Blue} Cztery/pięć}, gdy $n = 4p > 12$ (piąta pojawia się tylko dla $p = 4k+1$).
	\item {\color{Blue} Pięć}, gdy $n = p^3$:
	\begin{itemx}
	\item $D_8$ (symetrie kwadratu) i $Q$ (kwaterniony) dla $p = 2$.
	\item $UT(3, p)$ (macierze $3 \times 3$ nad $\mathbb F_p$, zera pod przekątną, jedynka na niej) i $(\Z/p^2\Z) \rtimes (\Z/p\Z)$ dla $p \ge 3$.
	\end{itemx} 
	\item {\color{Blue} Piętnaście}, gdy $n = p^4$ i $p \neq 2$ (czternaście dla $p = 2$).
	\begin{itemx}
	\item $\langle a,b,c \mid a^4 = b^2 = c^2 = [a,b] = [b,c] = e, ca = abc\rangle$, $\langle a, x \mid a^8 = x^2 = e, xa = a^5x\rangle$, $\Z_4 \rtimes \Z_4$, $D_8 \times \Z_2$, $Q_8 \times \Z_2$, centralny produkt $D_8$ i $\Z_4$ (klasy dwa).
	\item $D_{16}$, półdiedralna $SD_{16}$, $Q_{16}$ (klasy trzy).
	\end{itemx}
	\item {\color{Blue} $2p + 61 + 2(p - 1, 3) + (p-1, 4)$} dla $p \ge 5$, $51$ dla $p = 2$ i $67$ dla $p = 3$.
	\item {\color{Blue} $3p^2 + 39p + 344 + 24(p-1,3) + 11(p-1, 4) + 2(p-1, 5)$} dla $p \ge 5$, $267$ dla $p = 2$ i $504$ dla $p = 3$.
\end{itemx}

\end{document}

(Kaplansky). An element $a$ in a ring $R$ has a left quasi-inverse if there exists an element $b \in R$ with $a+b-ba= 0$.
Prove that if every element in a ring $R$ except $1$ has a left quasi-inverse, then $R$ is a division ring.
Hint: R\{1} is a group under a, b -> a+b-ba
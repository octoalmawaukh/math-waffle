\wcht{Problem momentów}: czy znając momenty miary można odtworzyć miarę?
\wcht{Warunek Carlemana} (\datum{1922}): dla miary na $\R$ wystarczy, żeby zachodziło $\sum_{n \ge 1} m_{2n}^{-1:2n} = +\infty$.
Ciąg $m_n$ jest ciągiem momentów $\Lra$ macierze Hankla $(H_n)_{ij} = m_{i+j}$ są dodatnio półokreślone.
\wcht{Warunek Kreina} (\datum{1945}): jeżeli całka z $(- \ln f(x)) : (1 + x^2)$ nad $\R$ jest skończona, to istnieją różne miary o tych samych momentach, przy czym $\mu$ ma być absolutnie ciągłą miarą z $\D \mu(x) = f(x) \,\D x$.
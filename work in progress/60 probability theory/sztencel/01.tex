\wcht{P. probabilistyczna}: \prawo{1.X} $(\Omega, \mathcal F, \pstwo)$: zbiór, $\sigma$-ciauo i p-stwo.
\wcht{Aksjomaty Kołmogorowa} (\datum{1933}): p-stwo to unormowana miara na $\sigma$-ciele.
\wcht{Wzór włączeń/wyłączeń}.
Tw. o ciągłości: $A_n$ jest monotonicznym ciągiem zdarzeń $\Ra \pstwo\left(\lim_n A_n \right) = \lim_n \pstwo(A_n)$.
\wcht{Paradoks Bertranda}: p-stwo, że losowa cięciwa okręgu jest dłuższa niż bok wpisanego trójkąta równobocznego?
$1/4$: p-stwo geometryczne. 
$1/3$: kąt środkowy oparty na cięciwie musi być większy niż $2\pi/3$.
$1/2$: odległość środka cięciwy od środka koła nie przekracza $1/2$. 
\wcht{Paradoks \prawo{2.X} urodzin}.
Na to, by $p_k \le 1/2$ wystarcza, by $k(k-1) \ge 730\ln 2$, czyli $k_{365} \ge 23$.
\wcht{Paradoks kawalera de Mere}: łatwiej otrzymać choć jedną 1 przy czterech rzutach kostką czy choć raz dwie 1 na obu kostkach przy 24 rzutach ($n$ jedynek, $4 \cdot 6^{n-1}$ rzutów)?
\wcht{Igłę Buffona} o długości $l$ upuszczono na płaszczyznę pokrytą równoległymi prostymi (odległymi od siebie o $t$).
Igła przetnie linię z p-stwem ,,całka z $4/(t\pi)$ po prostokącie $0 \le \theta \le \pi/2$, $0 \le x \le (l/2) \sin \theta$'', czyli $\frac{2l}{t\pi}$ (krótka igła: $l \le t$).
Dla długiej: $0 \le x \le \min(t/2, \frac{l}{2} \sin \theta)$, wynik poniżej.
\hfill $d < 10^{18}$.
\[
	\prod_{t=1}^{k-1} \left(1-\frac{t}{365}\right) \le \exp\left(\frac{k-k^2}{730}\right) \spk
	k_d = \left\lceil \sqrt{2d\ln 2}
	    + \frac{3-2\ln 2}{6}
	    + \frac{9-4\ln^2 2}{72\sqrt{2d\ln 2}}
	    - \frac{2\ln^2 2}{135d}\right\rceil \spk
	\frac{2}{\pi} \arccos \frac{t}{l} + \frac{2l}{\pi t} \left[1-\sqrt{1-(t/l)^2}\right]
\]
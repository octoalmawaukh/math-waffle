Standardowy \prawo{13.1} \wcht{proces Wienera}: proces taki, że $W_0 = 0$, przyrosty są niezależne (jeśli $0 < t_1 < \dots < t_n$, to z-losowe $W_{t_1}$, $W_{t_2} - W_{t_1}$, itd. są nz), dla $t, u \ge 0$ jest $W_{t+u} - W_u \sim \mathcal N(0,1)$ i trajektorie $t \mapsto W_t(\omega)$ są ciągłe.
Zamiast warunku z ,,$\mathcal N$'' wystarczy dla $X_t \ge 0$ sprawdzić: $\expected X_1 = 0$, $\expected X_1^2 = 1$, $t \ge s$ pociąga $X_t - X_s \sim X_{t-s}$.
\wcht{PWL} dla Wienera: $\lim_t W_t/t = 0$.
Dla filtracji $\mathcal F_t = \sigma (W_s : s \le t)$ (gdzie $t \ge 0$) dostajemy martyngał $(W_t, \mathcal F_t)$ z ciągłym czasem.
\wcht{Levy}: jeśli proces $X_{t \ge 0}$ jest martygnałem o ciągłych trajektoriach, zaś $X_t^2 - t$ jest martyngałem, to $X_t$ jest procesem Wienera.
\wcht{Most Browna}: proces $U_t = W_t - tW_1$ dla $0 \le t \le 1$.
Proces $\exp [\alpha W_t - \alpha^2 t/2]$ jest martyngałem.

Prawie \prawo{13.X} wszystkie trajektorie procesu Wienera są nigdzie nier-lne.
\wcht{Zasada odbicia}: jeśli $(W_t)_{t \ge 0}$ jest procesem Wienera, to dla $r \ge 0$ jest $\pstwo (\sup_{s \le t} W_s > r) = 2 \pstwo (W_t > r)$.
\wcht{Tw. Dynkina-Hunta}: jeśli $\tau$ to moment stopu względem filtracji $\mathcal F_t = \sigma(W_s : s \le t)$, $\pstwo (\tau < \infty) = 1$, to proces $W_t^\circ = W_{\tau+t} - W_\tau$, $t \ge 0$, jest procesem Wienera nz od $\mathcal F_\tau$, tzn. $\sigma$-ciaua $\sigma(W_t^\circ : t \ge 0)$ i $\mathcal F_\tau$ są nz.

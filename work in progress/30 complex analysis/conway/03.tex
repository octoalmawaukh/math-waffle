\wcht{Holomorf}: {r--lna} \prawo{3.2} (,,jak w $\R^n$'').
Szereg potęgowy różniczkuje się w kole zbieżności bardzo ($\mathscr C^\infty$).
\wcht{Obszar}: otwarty, spójny.
\wcht{Gałąź logarytmu}: $f \colon G \to \C$ (z obszaru), że $z = \exp f(z)$, np. $G = \C \setminus \{z \le 0\}$, $z = |z| e^{i \theta}$, $-\pi <\theta < \pi$, niech $f(re^{i \theta}) = \log r + i\theta$, holo-.
\wcht{Harmofunkcja}: ciągłe drugie pochodne i $u_{xx} + u_{yy} = 0$.
Gdy $u, v \colon G \to \R$ ($G$: obszar) mają ciągłe czosnkowe, to $f \colon G \to \C$, $f(z) = u(z) + iv(z)$ jest holo- $\Lra$ spełnia \wcht{równania Cauchy'ego-Riemanna}: $u_x = v_y$, $u_y = -v_x$..
Harmo- $u \colon G \to \R$ ($G = \C$ lub otwarty dysk) mają harmo- sprzężenia ($v \colon G \to \R$, że $f=u+iv$ jest holo- na $G$).

Holo- \prawo{3.3} jest kątowierna tam, gdzie $f'(z_0)\neq 0$.
\wcht{Homografia} ($z \mapsto (az+b)/(cz+d)$ i $ad\neq bc$) nie zmienia \wcht{dwustosunku} i przenosi okręgi na okręgi.
Dwustosunek w $\R$ $\Lra$ punkty na okręgu.
\wcht{Symetria} $z, z^* \in \C_\infty$ względem okręgu $\Gamma$ przez $z_2, z_3, z_4$: $(z^*, z_2, z_3, z_4) = \overline{z, z_2,z_3,z_4}$.
\wcht{Reguła symetrji}: homografia $T$ przerzuca okrąg $\Gamma_1$ na $\Gamma_2$ $\Ra$ $\Gamma_1$-symetryczne punkty przerzuca na symetryczne-$\Gamma_2$.
Okrąg $\Gamma$ ma \wcht{orientację} $(z_1, z_2, z_3)$, gdy $z_j$ są w $\Gamma$.
\wcht{Reguła orientacji}: jeśli $\Gamma_1, \Gamma_2$ są okręgami w $\C_\infty$, $T$ homografią, że $T(\Gamma_1) = \Gamma_2$ i $(z_1, z_2, z_3)$ orientacją dla $\Gamma_1$, to $T$ przenosi prawą stronę $\Gamma_1$ na prawą stronę $\Gamma_2$ względem orientacji $(T z_1, Tz_2, Tz_3)$.
%\[
	%f'(a) = \lim_{h \to 0} \frac{f(a+h)- f(a)}{h}
%\]


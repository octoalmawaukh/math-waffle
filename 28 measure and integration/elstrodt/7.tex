\wcht{Miara znakowana}: \prawo{7.1} $\sigma$-addytywna funkcja $\nu \colon \mathcal A \to \R_\infty$, że $\nu(\varnothing) = 0$, nie przyjmuje jednocześnie $\pm \infty$.
Skończona: $\nu[\mathcal A] \subseteq \R$.
Dla każdej znakowanej miary istnieje  \wcht{rozkład Hahna} (\datum{1921}) $X = P \sqcup N$ na zbiór dodatni (jeśli $A \subseteq P$, to $\nu(A) \ge 0$) i ujemny (przez analogię).
Rozkład jest jednoznaczny z dokładnością do $\mu$-zerowych $Q$ ($A \subseteq Q$ pociąga $\nu(A) = 0$).
Wariacje: $\nu^+(A) = \nu(A \cap P)$, $\nu^-(A) = - \nu(A \cap N)$, $\|\nu\| = \nu^+ + \nu^-$.
Dwie miary znakowane $\nu, \rho$ są \wcht{singularne} do siebie, gdy $X = A \sqcup B$, że $A$ jest $\nu$-zerowy, zaś $B$: $\rho$-zerowy.
\wcht{Rozkład Jordana}: $\nu = \nu^+ - \nu^-$ na singularne; jest on ,,minimalny'': jeśli $\nu = \rho - \sigma$ (i choć jedna jest skończona), to $\nu^+ \le \rho$, $\nu^- \le \sigma$; ponadto $\nu^+ \perp \nu^-$.
Skończone miary znakowane na $\mathcal A$ jest uporządkowaną, wektorową p. Riesza.

Gdy \prawo{7.2} $\mu, \nu$ to znakowane (lub zespolone) miary na $\mathcal A$, to $\nu$ jest $\mu$-\wcht{absolutnie ciągła}, gdy $\mu$-zerowe są $\nu$-zerowe, $\nu << \mu$.
Wystarcza istnienie (quasicałkowalnej) gęstości dla $\nu$ względem $\mu$.
\wcht{Tw. Radona-Nikodyma}: jeśli $\mu$ jest $\sigma$-skończoną miarą, zaś $\nu << \mu$ znakowaną miarą na $\mathcal A$, to $\nu$ ma gęstość względem $\mu$ (istnieje półcałkowalna $f \colon X \to \R_\infty$, że $\nu = f \odot \mu$, $\mu$-p.w. jednoznacznie);
Jeśli $\nu$ jest miarą, to można założyć $f \ge 0$.
\wcht{Tw. o rozkładzie Lebesgue'a}: jeśli $\mu$ jest $\sigma$-skończoną miarą, zaś $\nu$ znakowaną $\sigma$-skończoną miarą, to $\nu = \rho + \sigma$ rozkłada się jednoznacznie na znakowane $\rho, \sigma$ (są $\sigma$-skończone), że $\rho << \mu$, $\sigma \perp \mu$ i $\rho$ ma półcałkowalną gęstość $f \colon X \to \R$ względem $\mu$ [$\rho, \sigma$ są skończone $\Lra$ $\nu$ jest].

\wcht{Tw. pokryciowe Vitaliego} \prawo{7.4} (\datum{1908}): jeśli $A \subseteq \R$ ma skończoną zew-Le-miarę, zaś $\mathcal F$ to \wcht{pokrycie Vitaliego} [rodzina przedziałów, że każdy $x \in A$ można przykryć dowolnie krótkim $I \in \mathcal F$], to dla każdego $\varepsilon > 0$ istnieją rozłączne $I_1, \dots, I_n \in \mathcal F$, że $\eta(A \setminus \bigsqcup_{k=1}^n I_k) < \varepsilon$.
%Prawe i lewe (górne) liczby ablatywne w $x$ dla $f \colon I \to \R$: $D^\pm f(x) = \limsup_{h \to 0^+} [\mp f(x) \pm f(x \pm h)]$, dolne: to samo, ale z $\liminf$. - potrzebne tylko w dowodzie
\wcht{Tw. Lebesgue'a} (\datum{1904}): monotonicznie rosnąca $f \colon [a,b] \to \R$ jest $\lambda$-p.w. różniczkowalna.
Kładąc $f' = 0$ w nieróżniczkowalnościach, $f' \in \mathcal L^1$ i całka z niej nie przekracza $f(b) - f(a)$.
Ograniczona wariacja $\Ra$ $\lambda$-p.w. różniczkowalna.
%Mamy prawe i lewe (górne) \wcht{liczby ablatywne} w $x \in I \subseteq \R$ (przedział, funkcja $I \to \R$); dolne dla $\liminf$ miast $\limsup$.
Inny wniosek (Fubini, \datum{1915}): jeśli $(f_n)$ jest ciągiem monotonicnie rosnących (malejących) funkcji na $[a,b]$, że szereg $F(x) = \sum_{n=1}^\infty f_n(x)$ zbiega, to można różniczkować wyraz po wyrazie $\lambda$-p.w.

\wcht{Tw. o gęstości} (Lebesgue, \datum{1904}): $\lambda$-p.w. punkty $A \subseteq \R$ są \wcht{punktami gęstości} (khm-1), tzn. $\eta (A \setminus D(A)) = 0$.
Jeśli $A \in \mathfrak L^1$, to $D(A) \in \mathfrak L^1$ i $\lambda(A \triangle D(A)) = 0$.
Funkcja $F \colon [a,b] \to \mathbb K$ \wcht{absolutnie ciągła} (Vitali, \datum{1905}): każdy $\varepsilon > 0$ ma $\delta > 0$, że khm-2 dla $a \le \alpha_1 < \beta_1 \le \ldots \le \alpha_n < \beta_n \le b$, takich że $\sum_k \beta_k - \alpha_k < \delta$, pociąga ciągłość i ograniczoną wariację.
\wcht{Tw. Vitaliego} (\datum{1905}): absolutnie ciągła o $\lambda$-p.w. pochodnej zero jest stała.
\wcht{Hauptsatz dla Le-całki} (Lebesgue, Vitali \datum{1904/1905}): jeśli $f \colon [a,b] \to \overline {\mathbb K}$ jest Le-całkowalna, to khm-3 ($a \le x \le b$) jest absolutnie ciągła i $F' = f$ $\lambda$-p.w.
Funkcja jest całką nieoznaczoną $\Lra$ absolutnie ciągła.
\[
	\lim_{h \to 0^+} \frac{\eta(A \cap [x-h, x+h])}{2h} = 1 \spk
	\sum_{k=1}^n |F(\beta_k) - F(\alpha_k)| < \varepsilon \spk
	F(x) := \int_a^x f(t) \,\textrm{d}t \spk
\]

Funkcja $J \to \R$ (z przedziału) jest absolutnie ciągła, gdy jest taka obcięta do dowolnego domkniętego przedziału.
\wcht{Singularna}: jest ciągła, rosnąca, $\beta$-p.w. z pochodną zero.
Jeśli $f \colon \R \to \R$ jest rosnąca, prawo-ciągła, to $\mu_f << \beta \Lra F$ absolutnie ciągła, wtedy $\mu_f = f' \odot \beta$.
Rosnąca, ciągła $f \colon \R \to \R$ jest singularna $\Lra$ $\mu_f$ jest bezatomowa i $\mu_f \perp \beta$.
Prawo-ciągła, rosnąca $ \colon \R \to \R$ rozkłada się na $F_a + F_s + F_d$ (też rosnące, prawo-ciągłe): sumę absolutnie ciągłej, singularnej i skokowej; $\mu_a + \mu_s$ jest bezatomowa, $\mu_d$ czysto atomowa. (\wcht{mieszanka Lebesgue'a}, \datum{1904})
Miary na przestrzeniach topologicznych:
\begin{enumx}
\item miary Borela i Radona: 
\begin{itemx}
\item twierdzenia o regularności
\item umiarkowane miary Borela
\item polskie przestrzenie
\item twierdzenie Łuzina
\end{itemx}
\item twierdzenie Riesza: 
\begin{itemx}
\item twierdzenie o przedłużaniu
\item przestrzenie lokalnie zwarte i zupełnie regularne
\item nośniki miar
\item ciągłe formy liniowe na $C_0(X)$
\end{itemx}
\item miara Haara: 
\begin{itemx}
\item grupy topologiczne
\item lewoniezmiennicze formy liniowe i miary
\item istnienie i jednoznaczność
\item zastosowania
\item niezmiennicze i względnie nieziennicze miary na przestrzeniach klas abstrakcji
\end{itemx}
\item słaba zbieżność i zwartość: 
\begin{itemx}
\item skończone miary na przestrzeniach metrycznych
\item słaba zbieżność ciągów miar
\item twierdzenie o zbitce
\item twierdzenia Helly'ego, Braya, Prochorova
\item transformata Laplace'a
\item metryka Prochorova
\end{itemx}
\end{enumx}